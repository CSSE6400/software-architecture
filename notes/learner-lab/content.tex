\title{AWS Academy: Learner Lab}
\author{Richard Thomas}
\date{\week{2}}

\maketitle

\section{Introduction}
In this course we use Amazon Web Services (AWS),
as an example of a cloud based Infrastructure as a Service (IaaS) and Platform as a Service (PaaS) environment.
AWS offers a wide range of services that can be used to develop software systems.
For example, they offer virtual compute resources, database storage options, and networking to tie it all together.
Services are offered on a pay as you go model, meaning you only pay for the seconds you use a service.

AWS Academy provides a collection of resources to learn how to deploy and manage infrastructure on AWS.
You have been enrolled in three AWS Academy courses, as well as AWS Academy Learner Lab.
See the \link{Introduction to AWS Academy notes}{https://csse6400.uqcloud.net/handouts/aws-academy.pdf}
for information about how to accept your invitation to enrol and login to AWS Academy.

\link{AWS Academy Learner Lab}{https://awsacademy.instructure.com/courses/152632}
will be used in practical sessions from week 4 onwards.
You have a second \link{Learner Lab}{https://awsacademy.instructure.com/courses/152633}
to use for the Cloud Infrastructure assignment, so that you start with a fresh budget for the assignment.
You have a budget of \textbf{\$50} USD for \emph{each} Learner Lab.
Once you have spent a Lab's budget, you will not be able to use that Learner Lab.


\section{Learner Lab}
In this section, you will learn how to access and use
\href{https://awsacademy.instructure.com/courses/152632}{AWS Academy Learner Lab}.

\begin{enumerate}

\item Login to the Practicals \href{https://awsacademy.instructure.com/courses/152632}{Learner Lab [152632]}.
      If the login takes you to your AWS Academy Dashboard, then open the AWS Academy Learner Lab [152632].
      This will open the course home page.

\vspace{3mm}
\hspace{-5mm}
\includegraphics[trim=0 200 0 0,clip,width=1.023\textwidth]{images/lab-homepage}

\newpage
\item Navigate to the \texttt{Modules} tab and select the link for ``Launch AWS Academy Learner Lab''.
      You will need to accept the AWS Academy Learner Lab terms and conditions to be able to launch Learner Lab.

      You should also open and browse the ``AWS Academy Learner Lab Student Guide'' and
      ``Learn how to effectively use the AWS Academy Learner Lab'' links,
      which provide information about using the lab environment.

\vspace{3mm}
\hspace{-5mm}
\includegraphics[width=1.023\textwidth]{images/modules-page}

\newpage
\item You should now see the Learner Lab interface.
   \begin{itemize}
      \item The AWS text, near the top left of the window, with the (currently) red circle is the link to open the AWS console.
      \item You can see your budget usage in the same row near the top of the window.
            Note that the budget is not updated in real-time.
            Do not rely on the value it displays, if you are getting close to the \$50 budget limit.
      \item The \texttt{00:00} is a countdown of time remaining for your lab.
            A lab can only remain active for 4 hours, after which it will close,
            unless you press ``Start Lab'' again before the 4 hours expires.
            Once the lab is started, \texttt{00:00} will change to \texttt{04:00}.
      \item ``AWS Details'' will become important later but are not needed now.
      \item The ``Readme'' button will re-open the text panel currently on the right of the terminal interface.
      \item The Readme text has a lot of important information,
            including what AWS services are available in the Learner Labs environment, please read it.
      \item The terminal interface is an environment with the SSH keys required to connect to AWS instances semi-automatically (we will use this today).
   \end{itemize}

\vspace{3mm}
\hspace{-15mm}
\includegraphics[width=1.08\textwidth]{images/lab-interface}

\vspace{3mm}
\notice{If you get an error message saying ``labs.vocareum.com refused to connect.'',
        ensure that your browser is not in the incognito mode.
        If you still encounter this error, try using a different browser.}

\newpage
\item Start the lab by clicking on ``Start Lab''. It will take a few moments to get ready%
      \footnote{Sometimes it can take several minutes for the lab to start.}.
      The red circle will turn yellow as the lab is starting, and green once it has started.
      Click on the AWS text with the green circle when it is available.
      This will open the AWS Console in a new browser tab.
      (You may need to enable pop-ups from awsacademy.)
      If you end up working for a company which uses AWS, welcome to your new home.
 
\vspace{2mm}
\hspace{-5mm}
\includegraphics[width=1.023\textwidth]{images/aws-console}

\end{enumerate}

\vspace{2mm}
\noindent
Explore the AWS Academy Learner Lab environment, including viewing the course content in the \texttt{Modules} tab.
The \link{week 4 practical worksheet}{https://csse6400.uqcloud.net/practicals/week04.pdf}
provides an introduction to using AWS infrastructure.
You could work through that material at your own pace,
if you wish to get an early start on stage 2 of the Cloud Infrastructure assignment.
