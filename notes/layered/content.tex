\title{Layered Architecture}
\maketitle

\section{Introduction}
In the beginning there was the big ball of mud.
The ball of mud is an architectural style identified by it's lack of architectural style \cite{ballofmud}.
In a ball of mud architecture, all components of the system are allowed to communicate.
If your GUI code wants to ask the database a question, it will write an SQL query and ask it.
Likewise, if the code which primarily talks to the database decides your GUI needs to be updated a particular way, it will make it so.

The ball of mud style is a challenging system to work under.
Modifications can come from any direction at any time.
Akin to a program which primarily uses global variables,
it is hard, if not impossible,
to understand everything that is happening or could happen.

The first architectural style we will investigate is the layered architecture.
Layered architecture (also called multi-tier or tiered architecture) 
partitions software into specialized components and restricts how those components can communicate with each other.
The intention of a layered architecture is to create superficial boundaries between software components.
Often component boundaries aren't enforced by the technology but by architectural policy.

% The isolated components of a layered architecture are normally technically partitioned rather than domain partitioned.

\begin{figure}[h]
\centering
\begin{tikzpicture}[component/.style={draw, anchor=center, text width=120pt}]
    \node [component](P) at (0,0)  {Presentation Layer};
    \node [component] at (0,-1)  {Business Layer};
    \node [component] at (0,-2)  {Persistence Layer};
    \node [component](D) at (0,-3)  {Database Layer};

    \node[draw, fit=(P) (D)](hardware) {};
\end{tikzpicture}
\caption{The traditional specialized components of a layered architecture.}
\label{fig:traditional-layered}
\end{figure}

\section{Standard Form}

The traditional components of a layered architecture are seen in Figure \ref{fig:traditional-layered}.
This style of layered architecture is the four-tier architecture.
Here, our system is composed of a presentation layer, business layer, persistence layer, and database layer.

The presentation layer takes data and formats it in a way that is sensible for humans.
For command line applications, the presentation layer would accept user input and print formatted messages for the user.
For traditional GUI applications, the presentation layer would use a GUI library to communicate with the user.

\todo{explain business, persistence, and database}

One of the key benefits afforded by a well designed layered architecture is each layer should be interchangeable.
A typical example is an application which starts as a command line application can be adapted to a GUI application by replacing the presentation layer.



\section{Deployment Variations}

While the layered architecture is popular with monolithic applications, as it allows monoliths to simulate physical isolation,
a layered architecture does not have to be monolithic.

Each layer can be physically deployed to separate binaries on different systems.
The most common variant of distributed deployment is separating the database layer.
Since databases have well defined contracts and are language independent, the database layer is a natural first choice for physical separation.

\begin{figure}[h]
    \centering
    \begin{tikzpicture}[component/.style={draw, anchor=center, text width=120pt}]
        \node [component](P) at (0,0)  {Presentation Layer};
        \node [component](B) at (0,-1)  {Business Layer};
        \node [component](Pe) at (0,-2)  {Persistence Layer};
        \node [component](D) at (0,-3.5)  {Database Layer};
    
        \node[draw, fit=(P) (Pe)](hardware1) {};
        \node[draw, fit=(D)](hardware2) {};

        \draw (hardware1) -- (hardware2);
    \end{tikzpicture}
    \caption{Traditional layered architecture with a separately deployed database.}
    \label{fig:layered-db-separated}
\end{figure}

Of course, in a well designed system, any layer of the system could be physically separated.
The presentation is another common target.
Physically separating the presentation layer gives users the ability to only install the presentation layer and allow communication to
other software components to occur via network communication.

\begin{figure}[h]
    \centering
    \begin{tikzpicture}[component/.style={draw, anchor=center, text width=120pt}]
        \node [component](P) at (0,0)  {Presentation Layer};
        \node [component](B) at (0,-1.5)  {Business Layer};
        \node [component](Pe) at (0,-2.5)  {Persistence Layer};
        \node [component](D) at (0,-4)  {Database Layer};
    
        \node[draw, fit=(P)](hardware1) {};
        \node[draw, fit=(B) (Pe)](hardware2) {};
        \node[draw, fit=(D)](hardware3) {};

        \draw (hardware1) -- (hardware2);
        \draw (hardware2) -- (hardware3);
    \end{tikzpicture}
    \caption{Traditional layered architecture with a separately deployed database and presentation layer.}
    \label{fig:layered-db-pres-separated}
\end{figure}

This deployment form is very typical of web applications.
The presentation layer is deployed as a HTML/JavaScript application which makes network requests to the remote business/persistence layer.
The business/persistence layer then validates requests and makes any appropriate database updates.


\section{Closed/Open Layers}