There are a few tools that support C4. Some to consider are \link{Structurizr}{https://www.structurizr.com/},
\link{C4-PlantUML}{https://github.com/plantuml-stdlib/C4-PlantUML}, \link{Archi}{https://www.archimatetool.com/},
\link{IcePanel}{https://icepanel.io/}, or \link{Gaphor}{https://gaphor.org/}.

\begin{description}
    \item[Structurizr]
        was developed by Simon Brown as a tool to support generating C4 diagrams from textual descriptions.
        UQ students may register for free access to the paid version of the \link{Structurizr Cloud Service}{https://structurizr.com/help/academic}.
        You must use your \texttt{student.uq.edu.au} or \texttt{uq.net.au} email address when you register to get free access.
        Structurizr is an \link{open source tool}{https://github.com/structurizr/}.
        You can use a domain specific language to describe a C4 model, or you can embed the details in Java or .Net code.
    \item[C4-PlantUML] extends the UML modelling tool PlantUML to support C4.
    \item[Archi] is an open source visual modelling tool that
        \link{supports C4}{https://www.archimatetool.com/blog/2020/04/18/c4-model-architecture-viewpoint-and-archi-4-7/} and ArchiMate models.
    \item[IcePanel] is a cloud-based visual modelling tool that supports C4. There is a limited free license for the tool.
    \item[Gaphor] is an open source visual modelling tool that supports UML and C4.
\end{description}

\subsection{Textual vs Visual Modelling}
The tools described above include both graphical and textual modelling tools.
Graphical tools, such as Archi and Gaphor, allow you to create models by drawing them.
This approach is often preferred by \link{visually oriented learners}{https://vark-learn.com/strategies/visual-strategies/}.
Text-based tools, such as C4-PlantUML and Structurizr, allow you to create models by providing a textual description of the model.
This approach is often preferred by \link{read/write oriented learners}{https://vark-learn.com/strategies/readwrite-strategies/}.

Despite preferences, there are situations where there are advantages of using a text-based modelling tool.
Being text, the model can be stored and versioned in a version control system (e.g. git).
For team projects, it is much easier for everyone to edit the model and ensure that you do not destroy other team members' work.
It is also possible to build a tool pipeline that will generate diagrams and embed them into the project documentation.

Text-based modelling tools, such as Structurizr or C4-PlantUML, use a
\link{domain specific language}{https://opensource.com/article/20/2/domain-specific-languages} (DSL) to describe the model.
These tools require that you learn the syntax and semantics of the DSL.
The following sources of information will help you learn the Structurizr DSL:
\begin{itemize}[nosep]
    \item \link{language reference manual}{https://github.com/structurizr/dsl/blob/master/docs/language-reference.md},
    \item \link{language examples}{https://github.com/structurizr/dsl/tree/master/docs/cookbook}, 
    \item \link{on-line editable examples}{https://structurizr.com/dsl}, and
    \item \link{off-line tool}{https://github.com/structurizr/cli}.
\end{itemize}

\subsection{Example Diagrams}
You may find the Sahara eCommerce C4 model useful as an example of a number of features of the Structurizr DSL.
You are able to download the C4 model of the Sahara eCommerce example, from the course website.
The \link{C4 model}{https://csse6400.uqcloud.net/resources/c4_model.zip}
was created using the \link{Structurizr}{https://www.structurizr.com/} DSL.
