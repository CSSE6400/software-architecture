The important thing is that you want to use a modelling tool, not a drawing tool.
Many drawing tools provide UML templates, and some also support C4.
The issue with drawing tools is that they do not know what the elements of the diagram mean.
If the name of an operation in a class is changed in a drawing tool, you will need to manually change it wherever it is referenced in other diagrams
(e.g. in sequence diagrams).
A modelling tool will track the information that describes the model, so that a change to a model element in one place,
will be replicated wherever that element appears in other diagrams.

There are many tools that support UML.
In a commercial project using UML on a large system, the cost of professional UML tools is negligible and is quickly recovered by the automation they provide.
There are a number of free UML tools. Some to consider are \link{Astah}{https://astah.net/products/free-student-license/},
\link{ModelIO}{https://www.modelio.org/}, or \link{PlantUML}{https://plantuml.com/}.
\link{Visual Paradigm}{https://www.visual-paradigm.com/} is not as recommended, as their free cloud-based tool is only a drawing tool, and not a modelling tool.

\begin{description}
    \item[Astah]
        is a commercial product that supports visual modelling in many notations. They provide a free UML tool for students.
    \item[ModelIO] is an open source visual UML modelling tool.
    \item[PlantUML] Is an open source text-based descriptive language that generates UML diagrams.
        \link{PlantText}{https://www.planttext.com/} is an online tool supporting it.
    \item[Visual Paradigm] is a commercial product that supports visual modelling in many notations.
        They provide a simple free cloud-based drawing tool that supports UML and some limited aspects of C4, but it lacks full modelling support.
\end{description}

\noindent
There are fewer tools that support C4. Some to consider are \link{Structurizr}{https://www.structurizr.com/},
\link{C4-PlantUML}{https://github.com/plantuml-stdlib/C4-PlantUML}, \link{Archi}{https://www.archimatetool.com/},
\link{IcePanel}{https://icepanel.io/}, or \link{Gaphor}{https://gaphor.org/}.

\begin{description}
    \item[Structurizr]
        was developed by Simon Brown as a tool to support generating C4 diagrams from textual descriptions.
        UQ students may register for free access to the paid version of the \link{Structurizr Cloud Service}{https://structurizr.com/help/academic}.
        You must use your \texttt{student.uq.edu.au} or \texttt{uq.net.au} email address when you register to get free access.
        Structurizr is an \link{open source tool}{https://github.com/structurizr/}.
        You can use a domain specific language to describe a C4 model, or you can embed the details in Java or .Net code.
    \item[C4-PlantUML] which extends PlantUML to support C4.
    \item[Archi] is an open source visual modelling tool that
        \link{supports C4}{https://www.archimatetool.com/blog/2020/04/18/c4-model-architecture-viewpoint-and-archi-4-7/} and ArchiMate models.
    \item[IcePanel] is a cloud-based visual modelling tool that supports C4. There is a limited free license for the tool.
    \item[Gaphor] is an open source visual modelling tool that supports UML and C4.
\end{description}