\documentclass{slide}

%\usepackage{pgfpages}
%\setbeameroption{show notes on second screen}

\title{Distributed Computing III}
\subtitle{
    \textit{Murphy was an optimist.}\\
    \vspace{3mm} \hspace{2cm} \textit{--- O'Toole's Commentary}\\
    \vspace{8mm} CSSE6400
}

\author{Richard Thomas}
\date{\week{9}}

\begin{document}

\maketitle

\questionanswer{What communication faults may occur?}{
    \begin{itemize}[<+(1)->]
        \item Message not delivered	\note<2>{Lost in transit.}
        \item Message delayed		\note<3>{Network delay or receiver overloaded,\\but message will be processed later.}
        \item Receiver failed		\note<4>{Receiver software has crashed or node has died.}
        \item Receiver busy			\note<5>{Receiver temporarily not replying\\(e.g. garbage collection has frozen other processes).}
        \item Reply not received		\note<6>{Request was processed but reply lost in transit.}
        \item Reply delayed			\note<7>{Reply will be received later.}
    \end{itemize}
}

\questionanswer{How do we detect faults?}{
    \begin{itemize}[<+(1)->]
        \item No listener on port -- RST or FIN packet
        		\note<2>{Assumes node is running \& reachable.\\OS should close or refuse connection.\\
        				 Error packet may be lost in transit.}
        \item Process crashes -- Monitor report failure
        		\note<3>{Assumes node is running \& reachable.\\Most reliable.}
        \item IP address not reachable -- unreachable packet
        		\note<4>{Router has to determine address is not reachable,\\which is no easier than for your application.}
        \item Query switches
        		\note<5>{Need permissions to do this.\\Will only have this in your own data centre.}
        \item Timeout
        		\note<6>{UDP reduces network transmission time guarantee\\-- does not perform retransmission.}
    \end{itemize}
}

\questionanswer{What to do if fault is detected?}{
    \begin{itemize}
        \item Retry
        \item Restart
    \end{itemize}
}
\note[itemize]{
    \item How many retries? How often?
    \item Exponential backoff with jitter
    \item How long to wait to restart?
    \item Too long reduces responsiveness.
    \item Unacknowledged messages need to be sent to other nodes \\-- reducing performance.
    \item Too short may prematurely declare nodes dead.
    \item May lead to contention -- two nodes processing the same request.
    \item May lead to cascading failure -- load is sent to other nodes, slowing them down so they are then declared dead \dots.
}

\definition{Idempotency}{
    Repeating an operation does not change receiver's state.
}
\note[itemize]{
    \item Idempotent consumer pattern
    \item Tag messages with an ID, so repeated messages can be ignored
    \item Or, redo messages that do not change state (e.g. queries)
}

\begin{frame}{Byzantine Generals Problem}
    \begin{columns}[onlytextwidth,T]
      \column{30mm}
        \vspace{2mm}
        \includegraphics[width=30mm]{images/byzantine-general.jpg}  % From https://pixabay.com/photos/konstantinos-paleologos-emperor-4618859/

      \column{\dimexpr\linewidth-30mm-5mm}
        {\LARGE
        \begin{itemize}
            \item $n$ generals need to agree on plan
%            \vspace{2mm}
            \item Can only communicate via messenger
            \item Messenger may be delayed or lost
            \vspace{2mm}
            \item Some generals are traitors
            \begin{itemize}
                \vspace{1mm}
                \Large\item Send dishonest messages
                \vspace{1mm}
                \Large\item Pretend to have not received message
                \vspace{1mm}
                \Large\item Send messages pretending to be another general
            \end{itemize}
        \end{itemize}
        }
    \end{columns}
\end{frame}
\note{Link analogy to Byzantine faults}

\definition{Byzantine Faults}{
    Nodes in a distributed system may `lie'. \\--- Send faulty or corrupted messages or responses.
}
\note[itemize]{
    \item A message that causes the receiver to fail.
    \item Incorrect responses (e.g. they have finished processing a message but haven't).
    \item Can be due to faults or malicious hosts.
    \item Difficult to deal with all possible variations of these faults.
}

\questionanswer{Can we design a system to be Byzantine fault tolerant?}{
    Yes, but, it is \emph{challenging}.
}
\note[itemize]{
    \item Most systems don't attempt to
    \item Some need to (e.g. safety critical systems, blockchain, \dots)
    \item Refer to CSSE3012 Safety Critical guest lecture.
}

\begin{frame}{Limited Fault Tolerance}
    \vspace{1mm}
    {\LARGE
        \begin{itemize}
            \item<1-> Validate format of received messages
            \begin{itemize}
                \Large\item Need strategy to handle \& report errors
            \end{itemize}
            \vspace{2mm}
            \item<2-> Santise inputs
            \begin{itemize}
                \Large\item Assume any input from external sources may be malicious
            \end{itemize}
            \vspace{2mm}
            \item<3-> Retrieve data from multiple sources
            \begin{itemize}
                \Large\item If possible
                \Large\item e.g. Multiple NTP servers
            \end{itemize}
        \end{itemize}
    }
\end{frame}

\point[Assumption]{
    If all nodes are part of our system, we may assume there are no Byzantine faults.
    \begin{itemize}
        \item Santise user input
        \vspace{1mm}
        \item Byzantine faults may still arise
        \begin{itemize}
            \LARGE\item Logic defects
            \begin{itemize}
                \Large\item Same code is usually deployed to all replicated nodes, defeating easy fault tolerance solutions
            \end{itemize}
        \end{itemize}
    \end{itemize}
}

\definition{Poison Message}{
    A message that causes the receiver to fail.
}
\note[itemize]{
    \item Could literally cause the receiver to crash
    \item Often the receiver just cannot process the message and aborts processing
}

\point{Normal Message Flow}
% Sequence of slides animating a poison message.
% First 3 slides are an example of a message being queued and processed.
% Slides 4-8 are an example of a poison message blocking the queue.

% Message being queued and processed normally.

\image[trim=40 90 40 90,clip,width=\textwidth]{diagrams/poison-message-01.png}
\note[itemize]{
    \item Normal message/event is sent from a Producer.
}

\image[trim=40 90 40 90,clip,width=\textwidth]{diagrams/poison-message-02.png}
\note[itemize]{
    \item Normal message/event is queued (in Message Queue).
}

\image[trim=40 90 40 90,clip,width=\textwidth]{diagrams/poison-message-03.png}
\note[itemize]{
    \item Normal message/event is dequeued and processed by a Consumer.
}

\begin{frame}
    \begin{columns}[onlytextwidth,c]
      \column{0.5\textwidth}
        \huge Poison Message

      \column{0.5\textwidth}
        \includegraphics[width=0.5\textwidth]{images/poison-bottle.png}
        % From https://pixabay.com/vectors/poison-bottle-toxic-chemistry-4662212/
    \end{columns}
\end{frame}
\note[itemize]{
    \item Receiver can't process message.
    \item Always fails -- Not due to transient failure.
    \item Failed messages are retried.
    \item Returned to front of queue -- Preserve message order.
    \item Next receiver fails to process message -- Infinite loop.
    \item Blocks sending of following messages.
}

% Next 5 slides are an example of a poison message blocking the queue.

\image[trim=40 90 40 90,clip,width=\textwidth]{diagrams/poison-message-04.png}
\note[itemize]{
    \item This set of slides is an example of a poison message blocking the queue.
    \item Poison message is at head of queue.
blocking issue.
}

\image[trim=40 90 40 90,clip,width=\textwidth]{diagrams/poison-message-05.png}
\note[itemize]{
    \item This set of slides is an example of a poison message blocking the queue.
    \item Poison message is dequeued by a Consumer.
}

\image[trim=40 90 40 90,clip,width=\textwidth]{diagrams/poison-message-06.png}
\note[itemize]{
    \item This set of slides is an example of a poison message blocking the queue.
    \item Consumer fails (crashes).
    \item Poison message is added back to the head of the queue (re-try).
}

\image[trim=40 90 40 90,clip,width=\textwidth]{diagrams/poison-message-07.png}
\note[itemize]{
    \item This set of slides is an example of a poison message blocking the queue.
    \item Next Consumer dequeues poison message and fails (crashes).
}

\image[trim=40 90 40 90,clip,width=\textwidth]{diagrams/poison-message-08.png}
\note[itemize]{
    \item This set of slides is an example of a poison message blocking the queue.
    \item Poison message is added back to the head of the queue again (re-try).
    \item Infinite loop ...
    \item \textbf{Comment} that poison messages block processing regardless of how they're delivered.
    \item A message queue or service isn't the key blocking point.
    \item Async messages sent directly to a consumer requires it to queue them as they're processed, leading to the same blocking issue.
}

\questionanswer{What causes a message to be poisonous?}{
    \begin{itemize}
        \item<2-> Content is invalid
            \begin{itemize}
                \Large\item e.g. Invalid product id sent to purchasing service
                \vspace{1mm}
                \Large\item Error handling doesn't cater for error case
            \end{itemize}
        		\note<2>{Invalid content may be
        				 \begin{itemize}
        				 	\item corrupted data,
        				 	\item old version of data structure,
        				 	\item incorrect data, or
        				 	\item malicious data.
        				 \end{itemize}
        				}
            \vspace{2mm}
        \item<3-> System state is invalid
            \begin{itemize}
                \Large\item e.g. Add item to shopping cart that has been deleted
                \vspace{1mm}
                \Large\item Logic doesn't handle out of order messages
                \begin{itemize}
                    \large\item Insidious asynchronous faults
                \end{itemize}
            \end{itemize}
         	\note<3>{Invalid state may be
        				 \begin{itemize}
        				 	\item events out of order (e.g. delete then update),
        				 	\item logic error making state invalid, or
        				 	\item external corruption of persistent state.
        				 \end{itemize}
        				}
   \end{itemize}
}

\begin{frame}{Detecting Poison Messages}
    \vspace{1mm}
    {\LARGE
        Retry counter -- with limit
        \begin{itemize}
            \Large\item Where is counter stored?
            \begin{itemize}
                \large\item Memory -- What if server restarts?
                \vspace{1mm}
                \large\item DB -- Slow
                \vspace{2mm}
                \large\item Must ensure counter is reset, regardless of how message is handled
                \begin{itemize}
                    \item e.g. Message is manually deleted
                \end{itemize}
            \end{itemize}
        \end{itemize}
        \pause
        \vspace{3mm}
        Message service may have a timeout property
        \begin{itemize}
            \Large\item Message removed from queue
            \begin{itemize}
                \large\item Pending messages get older while waiting for poison message
                \vspace{1mm}
                \large\item Transient network faults may exceed timeout
            \end{itemize}
        \end{itemize}
    }
\end{frame}

\begin{frame}{Detecting Poison Messages}
    \vspace{1mm}
    {\LARGE
        Monitoring service
        \begin{itemize}
            \Large\item Trigger action if message stays at top of queue for too long
            \vspace{2mm}
            \Large\item Can check for queue errors
            \begin{itemize}
                \large\item No messages are being processed
                \vspace{1mm}
                \large\item Restart message service
            \end{itemize}
        \end{itemize}
    }
\end{frame}

\begin{frame}{Handling Poison Messages}
    \vspace{1mm}
    {\LARGE
        Discard message
        \begin{itemize}
            \Large\item System must not require guarantee of message delivery
            \Large\item Suitable when message processing speed is most important
        \end{itemize}
        \pause
        \vspace{3mm}
        Always retry
        \begin{itemize}
            \Large\item Requires mechanism to fix message
            \begin{itemize}
                \large\item Often requires manual intervention
            \end{itemize}
            \Large\item Suitable when message delivery is most important
            \Large\item Very long delays in processing
        \end{itemize}
    }
\end{frame}

\begin{frame}{Handling Poison Messages}
    \vspace{1mm}
    {\LARGE
        Dead-letter queue
        \begin{itemize}
            \Large\item Long transient failures result in adding many messages
            \begin{itemize}
                \large\item e.g. Network failure
            \end{itemize}
            \Large\item Requires manual monitoring and intervention
            \Large\item System must not require strict ordering of messages
            \Large\item Suitable when message processing speed is important
        \end{itemize}
    }
\end{frame}

\begin{frame}{Handling Poison Messages}
    \vspace{1mm}
    {\LARGE
        Retry queue
        \begin{itemize}
            \Large\item Transient failures also added
            \Large\item Use a previous strategy to deal with poison messages
            \Large\item System must not require strict ordering of messages
            \vspace{4pt}
            \Large\item Suitable when message processing speed is very important
            \begin{itemize}
                \large\item Main queue is never blocked
                \vspace{2pt}
                \large\item Receivers need to process from two message queues
            \end{itemize}
        \end{itemize}
    }
\end{frame}

\definition{Poison Pill Message}{
    Special message used to notify receiver it should no longer wait for messages.
}
\note{Emphasise that this is \textbf{different} to a poison message}

\questionanswer{Why use a poison pill message?}{
    Graceful shutdown of system.
}
\note[itemize]{
    \item Implementation is challenging with multiple producers and/or consumers.
    \item It must be the last message received by \textit{all} consumers.
}

\questionanswer{How to order asynchronous messages?}{
    \begin{itemize}
        \item Timestamps?
        \begin{itemize}
            \vspace{1mm}
            \LARGE\item Can't keep clocks in sync
            \vspace{1mm}
            \LARGE\item Limited clock precision
        \end{itemize}
    \end{itemize}
}
\note[itemize]{
    \item Trying to sync with NTP is unreliable
    \item Network delays during sync
    \item Clock drift between syncs
    \item Finite precision -- two events may end up with the same timestamp, if they occur in quick succession
}


\section{Data Issues}

\point[Consistency]{
    \vspace{4mm}
    \begin{description}
        \item[Eventual Consistency] weak guarantee
        \vspace{2mm}
        \item[Linearisability] strong guarantee
        \vspace{2mm}
        \item[Causal Ordering] strong guarantee
    \end{description}
}

\point[Eventual Consistency]{
\vspace{4mm}
\begin{itemize}
    \item Allows stale reads
    \vspace{5mm}
    \item May be appropriate for some systems
    \begin{itemize}
        \LARGE\item e.g. Social media updates\footnote{See \href{https://csse6400.uqcloud.net/slides/distributed2.pdf}{Distributed II slides 45 - 50}.}
    \end{itemize}
\end{itemize}
}

\point[Linearisability]{
\vspace{3mm}
\begin{itemize}
    \item<1-> Once value is written, all reads see same value
    \begin{itemize}
        \LARGE\item<1-> Regardless of replica read from
        	\note<1>{Abstraction over replicated database.\\Used when uniqueness needs to be guaranteed.}
    \end{itemize}
    \vspace{2mm}
    \item<2-> Single-leader replication
    \begin{itemize}
        \LARGE\item<2-> Read from leader
        \LARGE\item<2-> Read from synchronous follower
        	\note<2>{SLR -- defeats most performance benefits.\\Maintains reliability benefits.}
    \end{itemize}
    \item<3-> Multi-leader replication can't be linearised
    \vspace{2mm}
    \item<4-> Leaderless replication
    \begin{itemize}
        \LARGE\item<4-> Lock value on quorum \emph{before} writing
        	\note<4>{Leaderless -- similar performance cost to SLR.}
    \end{itemize}
\end{itemize}
}

\point[Causal Order]{
\vspace{4mm}
\begin{itemize}
    \item<1-> Order is based on causality
    \begin{itemize}
        \LARGE\item<1-> What event needs to happen before another
        \LARGE\item<1-> Allows concurrent events
    \end{itemize}
    \note<1>{
		\begin{itemize}
		    \item Linearisation defines a \highlight{total} order.
		    \item Causal ordering defines a \highlight{partial} order.
		    \item e.g. Git repo history with branching as \highlight{causal order}.
		    \item Not as strict as linearisability, so less performance cost.
	    \end{itemize}
	}
    \vspace{3mm}
    \item<2-> Single-leader replication
    \begin{itemize}
        \LARGE\item<2-> Record sequence number of writes in log
        \LARGE\item<2-> Followers read log to execute writes
    \end{itemize}
    \item<3-> Lamport timestamps
    \note<3>{Next slide is an example of Lamport timestamps.}
\end{itemize}
}

\image[trim=60 189 440 35,clip,height=\textheight]{../../notes/distributed3/diagrams/lamport-timestamp-seq.png}
\note[itemize]{
    \item Each node has an id, \& counts number of operations.
    \item Timestamp is a tuple (\emph{counterValue}, \emph{nodeID}).
    \item \emph{nodeID} guarantees every timestamp is unique, even if they have the same counter value.
    \item Every node stores the maximum counter value seen so far.
    \item Maximum is passed in every request to another node.
    \item If a node receives a request or response with a maximum counter value greater than its own counter value,
          it increases its own counter to the new maximum.
    \item \textbf{Message 4:} InventoryUpdater sends its current maximum value (\textbf{1}).
    		  ProductDB2’s counter is \textbf{2}, so it increases it to \textbf{3} \& returns to InventoryUpdater.
    		  InventoryUpdater records \textbf{3} as its new maximum value.
    		  \textbf{Message 5:} Updates ProductDB1's max value \& then counts operation.
}

\definition{Consensus}{
    A set of nodes in the system agree on some aspect of the system’s state.
}
\note[itemize]{
	\item Abstraction to make it easier to reason about system state.
	\item e.g. Selecting Leader DB if leader fails, or some Followers think it has failed.
}

\point[Consensus Properties]{
\vspace{3mm}
\begin{description}
    \item[Uniform Agreement] All nodes must agree on the decision
    \item[Integrity] Nodes can only vote once
    \vspace{1mm}
    \item[Validity] Result must have been proposed by a node
    \item[Termination] Every node that doesn't crash must decide
\end{description}
}
\note[itemize]{
    \item \textit{Uniform Agreement} and \textit{Integrity} are key
    \vspace{3mm}
    \item \textit{Validity} avoids nonsensical solutions
    \begin{itemize}
        \Large\item e.g. Always agreeing to a null decision
    \end{itemize}
    \vspace{3mm}
    \item \textit{Termination} enforces fault tolerance
    \begin{itemize}
        \Large\item Requires making progress towards a solution
    \end{itemize}
}

\definition{Atomic Commit}{
    All nodes participating in a distributed transaction need to form consensus to complete the transaction.
}
\note{Based on transaction atomicity from ACID\\(Atomicity, Consistency, Isolation, Durability)}

\point[Two-Phase Commit]{
\vspace{4mm}
\begin{description}
    \item[Prepare] Confirm nodes can commit transaction
    \vspace{2mm}
    \item[Commit] Finalise commit once consensus is reached
    \begin{itemize}
        \LARGE\item Abort if consensus can't be reached
    \end{itemize}
\end{description}
}
\note{Two-Phase commit example is on next slide.}

\image[trim=55 70 420 30,clip,height=\textheight]{../../notes/distributed3/diagrams/2phase-commit-seq.png}
\note[itemize]{
    \item Transaction ID used to track writes
    \item Prepare does all steps of a commit, \textit{aside} from confirming it -- It cannot be revoked by participant
    \item Commit intent is recorded in log before sending to participants
    \item Even if a participant fails, commit can proceed when it recovers
    \item Comment on \highlight{performance costs}
}

\begin{frame}{Distributed Systems Timing Assumptions}
    \vspace{1mm}
    {\LARGE
    \begin{itemize}
        \item<1-> Synchronous System
        \vspace{-3pt}
        \begin{itemize}
            \Large\item Not realistic due to faults above
            \Large\item Minimal performance benefit
        \end{itemize}
        \vspace{2mm}
        \item<2-> Partially Synchronous System
        \vspace{-3pt}
        \begin{itemize}
            \Large\item Assumes important message order is preserved
            \Large\item Assumes most faults are rare \& transient
            \Large\item Error handling to catch faults
        \end{itemize}
        \vspace{2mm}
        \item<3-> Asynchronous System
        \vspace{-3pt}
        \begin{itemize}
            \Large\item No timing assumptions
            \vspace{-1pt}
            \Large\item Important message order managed by application
            \Large\item Difficult \& limited design
        \end{itemize}
    \end{itemize}
    }
\end{frame}

\begin{frame}{Distributed Systems Node Failure Assumptions}
    \vspace{1mm}
    {\LARGE
    \begin{itemize}
        \item<1-> Crash Stop
        \begin{itemize}
            \Large\item Node fails and never restarts
        \end{itemize}
      	\note<1>{Cloud-based system that kills crashed nodes.}
        \vspace{3mm}
        \item<2-> Crash Recovery
        \begin{itemize}
            \Large\item Node fails and restarts
            \begin{itemize}
                \large\item Requires persistent memory for recovery close to prior state
            \end{itemize}
        \end{itemize}
        	\note<2>{\begin{itemize}
        			 	\item Any system that allows nodes to be restarted.
        			 	\item May lose some steps in memory for \highlight{non-critical} tasks.
        			 \end{itemize}
        			}
        \vspace{2mm}
        \item<3-> Arbitrary Failure
        \begin{itemize}
            \Large\item Nodes may perform spurious or malicious actions
            \begin{itemize}
                \large\item Byzantine faults
            \end{itemize}
        \end{itemize}
        	\note<3>{Nodes may send faulty, corrupt or malicious messages.}
    \end{itemize}
    }
\end{frame}

\point{
\begin{itemize}
    \item Distributed systems are hard to build
    \vspace{4mm}
    \item Large systems have to be distributed
    \begin{itemize}
        \LARGE\item Monoliths can't scale to millions of users
    \end{itemize}
    \vspace{4mm}
    \item Use environments, tools \& libraries
    \begin{itemize}
        \LARGE\item Leaverage others' experience
    \end{itemize}
    \vspace{4mm}
    \item CSSE7610 Concurrency: Theory \& Practice
    \begin{itemize}
        \LARGE\item Prove correctness of concurrent \& distributed systems
    \end{itemize}
\end{itemize}
}

\end{document}
