\documentclass{slide}

\usepackage{pgfpages}
\usepackage{languages}
\usepackage{changepage}

%\setbeameroption{show notes on second screen}

\hypersetup{
    colorlinks=true,
    linkcolor=violet,
    filecolor=purple,      
    urlcolor=blue,
    citecolor=black,
}

\title{Architectural Decision Records}
\subtitle{Software Architecture}
\author{Richard Thomas \& Guangdong Bai}
\date{\week{2}}

\begin{document}

\maketitle
\note[enumerate]{
    \item ADRs are the `why' of the architecture
    \item Good if obvious approach is wrong
    \item Mechanism for preserving assumptions of an architecture
}


\begin{frame}{Developer Reaction to Reading Software Architecture Documentation}

\begin{figure}
    \href{https://pixabay.com/vectors/computer-internet-unhappy-user-1295358/}{\includegraphics[width=0.9\textwidth]{images/frustration.png}}
\end{figure}

\end{frame}
\note[enumerate]{
    \item ADRs follow on from views
    \item But separated from views to make them more accessible
}


\questionanswer{How do you know why certain decisions were made in the architectural design?}
{\highlight{Architectural Decision Records (ADRs)}}


\begin{frame}{Record Decisions that Influence}

\Large{
\begin{itemize}
    \item \highlight{Structure} of the architecture
    \item Delivery of \highlight{quality attributes}
    \item \highlight{Dependencies} between important parts of the architecture
	\vspace{1mm}
    \item \highlight{Interfaces} between important parts of the architecture
    \begin{itemize}
        \large{\item or external interfaces}
    \end{itemize}
    \item \highlight{Principles} about implementation techniques or platforms
    \hfill
    \item More "why certain decisions were made" than "what are in structure, interfaces, etc."
\end{itemize}
}
\note{
    \begin{description}
        \item[Structure] e.g. \highlight{plugin} architecture: build core platform, expecting others to add to it; or
                         \highlight{microservices} architecture: allow scaling and small teams
        \item[Quality attributes] e.g. need to be \highlight{highly scalable} so using cloud services
        \item[Dependencies/Interfaces] picked dependency \textit{Y}, i.e. SMS API because \textit{rationale}.
                                       ADRs don't record details of interface, just \textit{rationale}.
        \item[Principles]
            \begin{itemize}
                \item use Haskell because we get computer science grads
                \item use on-prem because government org \& regulations
                \item use language \textit{X} because of obscure dependency
            \end{itemize}
    \end{description}
}

\end{frame}


\begin{frame}

\begin{figure}
    \href{https://decodenatura.com/bad-comments-and-how-to-fix-them/}{\includegraphics[width=0.9\textwidth]{images/warning-comment.png}}
\end{figure}

\end{frame}
\note[enumerate]{
    \item Think of ADRs as warnings
    \item Why is this not a good ADR? (\textit{No rationale})
}


\questionanswer{Why ADRs?}
{My code will defeat the architectural design,\\
\highlight{if} I do not know why it was designed that way.}
\note[enumerate]{
    \item I will break things if I don't know why it works a certain way
    \item I'll just \textit{fix that}
}


\begin{frame}{Why ADRs}

\LARGE{
\begin{itemize}
    \item \highlight{Team} alignment, ensuring all stakeholders are on the same page regarding architectural choices
      \vspace{1mm}
      \begin{itemize}
        \Large{\item Onboarding team members sync up on past decisions}
      \end{itemize}
    \item \highlight{Traceability} allowing team to track how architecture \highlight{evolves}
    \item \highlight{Justification} to defend \highlight{decisions} when questioned later
    \item Preserve \highlight{knowledge} for future reference
\end{itemize}
}

\end{frame}


\point[ADRs]{Record a \highlight{single} decision}
\note[enumerate]{
	\item Keeps it short.
	\item More likely to write it (and read it).
}


\point[ADRs]{Are \highlight{never} deleted\\
Mark as \highlight{superseded} and link to new decision}
\note[enumerate]{
	\item Preserves bad decisions --- less likely to repeat them
	\vspace{2mm}
	\item Superseded decision:
            \begin{enumerate}
                \large\item it was a bad idea
                \item it was a good idea at the time, but conditions have changed
            \end{enumerate}
}


\questionanswer{Where are ADRs documented?}
{Each decision is a separate file in the project repository%
\footnote{See the \texttt{adrs} directory in the \href{https://csse6400.uqcloud.net/resources/c4_model.zip}{C4 Model}
          on the course website.}}
\note[itemize]{
    \item C4 allows you to link to ADRs
    \item Useful to note as ADRs will be required for the project
}


\begin{frame}{ADR Management Tools}

\LARGE{
\begin{itemize}
    \item \href{https://github.com/thomvaill/log4brains}{Log4brains}\footnote{\url{https://github.com/thomvaill/log4brains}}
     \begin{itemize}
        \Large{\item Generates ADRs from Markdown files}
	    \Large{\item Web-based UI for browsing ADRs}
    \end{itemize}
    \vspace{0.4em}
    \item \href{https://github.com/npryce/adr-tools}{adr-tools}\footnote{\url{https://github.com/npryce/adr-tools}}
    \begin{itemize}
        \Large{\item Light-weight command-line tool}
	    \Large{\item Generates ADR templates automatically}
    \end{itemize}
\end{itemize}
}

\end{frame}


\begin{frame}{ADR Template \cite{nygard-adr}}

\Large{
\begin{description}
    \item[Title] Short phrase describing the decision
    \item[Date] When the decision was made
    \item[Status] Current status of the decision
    \begin{itemize}
        \large{\item proposed, accepted, deprecated, superseded, rejected}
    \end{itemize}
    \item[Summary] Summarise the decision and its rationale
    \item[Context] Describe the facts that influence the decision
    \item[Decision] Explain how the decision will solve the problem
    \item[Consequences] Impact of the decision
    \begin{itemize}
        \large{\item what's easier to do}
        \large{\item what's harder to do}
    \end{itemize}
\end{description}
}

\end{frame}
\note{
    \begin{description}
        \item[Title] Descriptive, easy to find, part of the filename
        \item[Status] Rejected is an option --- prevents repeating proposals
        \item[Summary] Describe context
    \end{description}
}

\begin{frame}{ADR Example}

\begin{figure}
    \href{https://github.com/CSSE6400/software-architecture/blob/main/notes/views/c4_model/adrs/0001-independent-business-logic.md}
        {\includegraphics[height=\textheight]{images/adr-example.png}}
\end{figure}

\end{frame}
\note[itemize]{
    \item From Sahara example in the Views lecture/notes
    \item Functional requirement: Ability to change mode of interaction with the system
    \item Independent business logic enables this
    \item Format of the summary is a why statement --- summarise the context and decision
}

\point[Reading...]{``Architectural Decision Records'' Notes \cite{adr-notes}}


\references{articles,books,ours}

\end{document}
