\documentclass{slide}

% Comment out animate package for GitHub commited version.
\usepackage{animate}

% Only needed if final slide with manually created table is used.
\usepackage{colortbl}

\usepackage{changepage}
\usepackage{setspace}
% From setspace use \setstretch. \setstretch requires \\ at end of paragraphs.

%\usepackage{pgfpages}
%\setbeameroption{show notes on second screen}

\title{Deployment Strategies}
\subtitle{CSSE6400}
\author{Richard Thomas}
\date{\week{12}}


\begin{document}

\maketitle

\definition{Deployment Strategy}{How a software system is made available to clients.}

\begin{frame}{Deployment Strategies}
  \vspace{1pt}
  {\huge
    \begin{itemize}
        \item Branching Strategies
        \item Recreate Deployment
        \item Rolling Deployment
        \item Blue/Green Deployment
        \item Canary Deployment
        \item A/B Deployment
        \item Shadow Deployment
    \end{itemize}
  }
\end{frame}
\note{There isn't any one perfect deployment strategy.}

\definition{Branching}{Copying the trunk to allow separate and parallel development.}
\note[itemize]{
    \item Branches deviate from the trunk.
    \item A few different branching strategies.
}

\begin{frame}{Branching Strategies}
  \vspace{1pt}
  {\huge{\setstretch{1.25}
    \begin{itemize}
        \item GitHub Flow
        \item GitLab Flow
        \item Release Branches
    \end{itemize}
  }}
\end{frame}
\note{Branching strategies supporting deployment strategies.}

\begin{frame}{GitHub Flow \cite{github-flow}}
    \vspace{1pt}
    \begin{columns}
    \column{0.45\textwidth}
      {\LARGE
        \vspace{-5mm}
        \begin{itemize}
            \item {\setstretch{0.95} Main is always deployable\\}
            \item Create branch
            \item Make changes
            \item Create pull request
            \item Resolve issues
            \item Merge pull request
            \item Delete branch
        \end{itemize}
      }
    \column{0.55\textwidth}
        \centering
        \includegraphics[height=0.93\textheight]{diagrams/github-flow.jpeg}
    \end{columns}
\end{frame}
\note[itemize]{
    \item Supports CI \& CD.
    \item Expects there is a single deployable version\\ (e.g. cloud / web systems).
}

\begin{frame}{GitLab Flow \cite{gitlab-flow}}
    \vspace{1pt}
    \begin{columns}
    \column{0.65\textwidth}
      {\LARGE
        \begin{itemize}
            \item Supports deployment windows
            \begin{itemize}
                \Large\item Merge to production
                \vspace{0.5mm}
                \Large\item Deploy when allowed
            \end{itemize}
            \vspace{2mm}
            \item Production branch
            \begin{itemize}
                \Large\item Plus alpha, beta, \dots
            \end{itemize}
            \vspace{1.5mm}
            \item Still have
            \begin{itemize}
                \Large\item Feature branches
                \vspace{0.5mm}
                \Large\item Pull requests
            \end{itemize}
        \end{itemize}
      }
    \column{0.35\textwidth}
        \centering
        \includegraphics[height=0.9\textheight]{diagrams/gitlab-flow.png}
    \end{columns}
\end{frame}
\note[itemize]{
    \item Deployment windows examples
    \begin{itemize}
        \Large\item App store approval
        \Large\item Server availability
        \Large\item Support availability
    \end{itemize}
}

\begin{frame}{Release Branches \cite{gitlab-flow}}
    \vspace{1pt}
    \begin{columns}
    \column{0.6\textwidth}
      \vspace{-5mm}
      {\LARGE
        \begin{itemize}
            \item {\setstretch{0.95} Supports multiple versions of system\\}
            \item Feature development in main
            \item Released versions are branches
            \vspace{3mm}
            \item Bug fixes in main
            \begin{itemize}
                \Large\item Cherry-pick into branches
            \end{itemize}
        \end{itemize}
      }
    \column{0.4\textwidth}
        \centering
        \includegraphics[height=0.9\textheight]{diagrams/release-branches.png}
    \end{columns}
\end{frame}
\note[itemize]{
    \item Cherry-pick: commit is copied from one branch to another, but the branches aren't merged.
}

\begin{frame}{Recreate Deployment \cite{deployment-strategies}}
    \vspace{1pt}
    \centering
    % \animategraphics output only works in Adobe Acrobat.
    % Replace with \includegraphics of the merged image for slides published via GitHub.
%    \animategraphics[trim=10 115 10 100,clip,controls,buttonsize=1em,width=1.05\linewidth]{6}{diagrams/recreate/fnum}{1}{20}
    \includegraphics[height=\textheight]{diagrams/recreate.png}
\end{frame}
\note[itemize]{
    \item Shutdown version 1.
    \item Deploy version 2.
    \item Requires downtime.
}

\begin{frame}{Recreate Deployment}
    \vspace{1pt}
    \begin{columns}[t]
    \column{0.5\textwidth}
      \huge Pros
      {\LARGE
        \begin{itemize}
            \item Easy
            \vspace{2mm}
            \item Renewed state
            \begin{itemize}
                \Large\item App reinitialised
	            \vspace{2mm}
                \Large\item {\setstretch{0.9} Persistent storage consistent with system version\\}
            \end{itemize}
        \end{itemize}
      }
    \column{0.5\textwidth}
      \huge Cons
      {\LARGE
        \begin{itemize}
            \item Downtime
        \end{itemize}
      }
    \end{columns}
\end{frame}
\note{Renewed state means app is reinitialised and db table structure is consistent with system version.}

\begin{frame}{Rolling Deployment \cite{deployment-strategies}}
    \vspace{1pt}
    \centering
    % \animategraphics output only works in Adobe Acrobat.
    % Replace with \includegraphics of the merged image for slides published via GitHub.
    \animategraphics[trim=10 115 10 100,clip,poster=14,controls,buttonsize=1em,width=1.05\linewidth]{6}{diagrams/rolling/fnum}{1}{27}
%    \includegraphics[height=\textheight]{diagrams/rolling/fnum15.png}
\end{frame}
\note[itemize]{
    \item Slowly roll out new version.
    \item Pool of instances of \textbf{V1} behind load balancer.
    \item Deploy an instance of \textbf{V2}.
    \item Add \textbf{V2} instance to pool.
    \item Remove one \textbf{V1} instance from pool.
    \item Continue until \textbf{V2} is fully deployed, replacing \textbf{V1}.
}

\begin{frame}{Rolling Deployment}
    \vspace{1pt}
    \begin{columns}[t]
    \column{0.5\textwidth}
      \huge Pros
      {\LARGE
        \begin{itemize}
            \item Fairly easy
            \vspace{1mm}
            \item {\setstretch{0.95} Slow release of new version\\}
            \begin{itemize}
                \Large\item Observe issues
                \vspace{0.7mm}
                \Large\item Rollback
            \end{itemize}
            \vspace{1mm}
            \item {\setstretch{0.95} Stateful instances can finish gracefully\\}
            \vspace{1mm}
            \begin{itemize}
                \Large\item {\setstretch{0.9} Instance is killed when inactive\\}
            \end{itemize}
        \end{itemize}
      }
    \column{0.5\textwidth}
      \huge Cons
      {\LARGE
        \begin{itemize}
            \item Time
            \item Support multiple APIs
            \item {\setstretch{0.95} Support different versions of persistent data structure\\}
            \item {\setstretch{0.95} No control over traffic to different versions\\}
        \end{itemize}
      }
    \end{columns}
\end{frame}

\begin{frame}{Blue-Green Deployment \cite{deployment-strategies}}
    \vspace{1pt}
    \centering
    % \animategraphics output only works in Adobe Acrobat.
    % Replace with \includegraphics of the merged image for slides published via GitHub.
    \animategraphics[trim=10 115 10 100,clip,poster=9,controls,buttonsize=1em,width=1.05\linewidth]{6}{diagrams/blue-green/fnum}{1}{14}
%    \includegraphics[height=\textheight]{diagrams/blue-green/fnum10.png}
\end{frame}
\note[itemize]{
    \item \textbf{V2} deployed alongside \textbf{V1}, including same number of instances.
    \item \textbf{V2} tested in production environment.
    \item Load balancer switched to use \textbf{V2} instances
    \item Shutdown \textbf{V1} instances.
}

\begin{frame}{Blue-Green Deployment}
    \vspace{1pt}
    \begin{columns}[t]
    \column{0.5\textwidth}
      \huge Pros
      {\LARGE
        \begin{itemize}
            \item {\setstretch{0.95} Instant release of new version\\}
            \item Fast rollback if necessary
            \vspace{1mm}
            \item {\setstretch{0.95} Only one version `live' at any time\\}
            \begin{itemize}
                \Large\item No versioning conflicts
            \end{itemize}
        \end{itemize}
      }
    \column{0.5\textwidth}
      \huge Cons
      {\LARGE
        \vspace{3mm}
        \begin{itemize}
            \item Expensive
            \begin{itemize}
                \Large\item Double the infrastructure
            \end{itemize}
            \vspace{1mm}
            \item {\setstretch{0.95} Stateful instance version switch difficult\\}
            \begin{itemize}
                \Large\item Can't kill instance in middle of a transaction
            \end{itemize}
        \end{itemize}
      }
    \end{columns}
\end{frame}

\begin{frame}{Canary Deployment \cite{deployment-strategies}}
    \vspace{1pt}
    \centering
    % \animategraphics output only works in Adobe Acrobat.
    % Replace with \includegraphics of the merged image for slides published via GitHub.
    \animategraphics[trim=10 115 10 100,clip,poster=13,controls,buttonsize=1em,width=1.05\linewidth]{4}{diagrams/canary/fnum}{1}{15}
%    \includegraphics[height=\textheight]{diagrams/canary/fnum14.png}
\end{frame}
\note[itemize]{
    \item Gradually shift traffic from \textbf{V1} to \textbf{V2}.
    \item Traffic usually split by percent (e.g. 90/10, 80/20, ...).
    \item Allows a trial deployment to see what happens.
}

\begin{frame}{Canary Deployment}
    \vspace{1pt}
    \begin{columns}[t]
    \column{0.5\textwidth}
      \huge Pros
      {\LARGE
        \begin{itemize}
            \item {\setstretch{0.95} New version released to subset of users\\}
            \item {\setstretch{0.95} Can monitor perform- ance and error rates\\}
            \item Easy and fast rollback
        \end{itemize}
      }
    \column{0.5\textwidth}
      \huge Cons
      {\LARGE
        \begin{itemize}
            \item Slow
            \item Implies poor testing
        \end{itemize}
      }
    \end{columns}
\end{frame}
\note{Canary is commonly used to see if something works or will fail in production.}

\begin{frame}{A/B Deployment \cite{deployment-strategies}}
    \vspace{1pt}
    \centering
    % \animategraphics output only works in Adobe Acrobat.
    % Replace with \includegraphics of the merged image for slides published via GitHub.
    \animategraphics[trim=10 115 10 100,clip,poster=last,controls,buttonsize=1em,width=1.05\linewidth]{6}{diagrams/a-b/fnum}{1}{19}
%    \includegraphics[height=\textheight]{diagrams/a-b/fnum19.png}
\end{frame}
\note[itemize]{
    \item Actually it's A/B Testing.
    \item Both versions are deployed and usage evaluated, usually via analytics.
    \item Long Term: Deploy version that has best usage result.
}

\begin{frame}{A/B Deployment}
    \vspace{1pt}
    \begin{columns}[t]
    \column{0.5\textwidth}
      \huge Pros
      {\LARGE
        \begin{itemize}
            \item {\setstretch{0.95} Multiple versions run in parallel\\}
            \vspace{1mm}
            \item {\setstretch{0.95} Full control over traffic distribution\\}
        \end{itemize}
      }
    \column{0.5\textwidth}
      \huge Cons
      {\LARGE
        \begin{itemize}
            \item {\setstretch{0.95} Needs intelligent load balancer\\}
            \vspace{1mm}
            \item {\setstretch{0.95} Debugging a version is difficult\\}
            \begin{itemize}
                \Large\item Need good logs \& tools
            \end{itemize}
        \end{itemize}
      }
    \end{columns}
\end{frame}
\note{A/B testing \& deployment requires sophisticated infrastructure and analytics to do well.}

\begin{frame}{Shadow Deployment \cite{deployment-strategies}}
    \vspace{1pt}
    \centering
    % \animategraphics output only works in Adobe Acrobat.
    % Replace with \includegraphics of the merged image for slides published via GitHub.
    \animategraphics[trim=10 115 10 100,clip,poster=13,controls,buttonsize=1em,width=1.05\linewidth]{6}{diagrams/shadow/fnum}{1}{15}
%    \includegraphics[height=\textheight]{diagrams/shadow/fnum14.png}
\end{frame}
\note[itemize]{
    \item Complex to setup.
    \item \textbf{V2} deployed alongside \textbf{V1}.
    \item All traffic is sent to \textbf{V1} \& \textbf{V2}.
    \item Tests \textbf{V2} ability to handle production load.
    \item Doesn't impact on production traffic or user experience.
    \item \textbf{V2} rolled out when it demonstrates it is stable.
    \item Need to manage interactions with external services (e.g. payment gateway).
    \item When customer checks out their shopping cart, you don't want to send two payment requests from \textbf{V1} \& \textbf{V2}.
    \item Mock external services.
    \item Persistent data from \textbf{V1} (production data) needs to be copied to \textbf{V2} when it's deployed as production, with any data migration.
}

\begin{frame}{Shadow Deployment}
    \vspace{1pt}
    \begin{columns}[t]
    \column{0.5\textwidth}
      \huge Pros
      {\LARGE
        \begin{itemize}
            \item {\setstretch{0.95} Performance testing with production traffic\\}
            \item No impact on users
        \end{itemize}
      }
    \column{0.5\textwidth}
      \huge Cons
      {\LARGE
        \vspace{3mm}
        \begin{itemize}
            \item Expensive
            \begin{itemize}
                \Large\item Double the infrastructure
            \end{itemize}
            \vspace{2mm}
            \item Complex to setup
            \begin{itemize}
                \Large\item Need mocks for external services
            \end{itemize}
        \end{itemize}
      }
    \end{columns}
\end{frame}
\note{Performance testing may give false confidence.\\ ~~~-- It's not user testing.}

\begin{frame}{Deployment Strategy Options}
  \vspace{1pt}
  {\huge
    \begin{itemize}
        \item Staging or beta testing
        \begin{itemize}
            \LARGE\item Recreate or Rolling
        \end{itemize}
        \vspace{1mm}
        \item Production (Live)
        \begin{itemize}
            \LARGE\item Rolling or Blue/Green
        \end{itemize}
        \vspace{1mm}
        \item Uncertain of system stability
        \begin{itemize}
            \LARGE\item Canary
        \end{itemize}
        \vspace{1mm}
        \item Evaluation
        \begin{itemize}
            \LARGE\item A/B or Shadow
        \end{itemize}
    \end{itemize}
  }
\end{frame}
\note{There isn't any one perfect deployment strategy.}

\begin{frame}{Deployment Considerations \cite{deployment-strategies}}
    \centering
    \includegraphics[height=0.93\textheight]{diagrams/deployment_strategies.png}
\end{frame}

% Manually created table of deployment_strategies.png content.
%\begin{frame}{Deployment Considerations}
%\vspace{-10mm}
%\begin{table}
%\begin{adjustwidth}{-12mm}{-12mm}
%\centering
%% Figure out how to fix first row's text having double-line spacing after line break.
%\def\arraystretch{1.5}
%\begin{tabular}{lccccccc}
%\rowcolor[rgb]{0.835,1,0.835} \multicolumn{1}{c}{Strategy} & \begin{tabular}[c]{@{}>{\cellcolor[rgb]{0.835,1,0.835}}c@{}}Zero\\Downtime\end{tabular} & \begin{tabular}[c]{@{}>{\cellcolor[rgb]{0.835,1,0.835}}c@{}}Real\\Traffic\end{tabular} & \begin{tabular}[c]{@{}>{\cellcolor[rgb]{0.835,1,0.835}}c@{}}Targeted\\Users\end{tabular} & Cost   & \begin{tabular}[c]{@{}>{\cellcolor[rgb]{0.835,1,0.835}}c@{}}Rollback\\Duration\end{tabular} & \begin{tabular}[c]{@{}>{\cellcolor[rgb]{0.835,1,0.835}}c@{}}Impact\\on Users\end{tabular} & \begin{tabular}[c]{@{}>{\cellcolor[rgb]{0.835,1,0.835}}c@{}}Setup\\Complexity\end{tabular}  \\
%\rowcolor[rgb]{0.941,1,1} Recreate                         & N                                                                                       & N                                                                                      & N                                                                                         & \$     & Long                                                                                        & High                                                                                      & Negligible                                                                                  \\
%\rowcolor[rgb]{1,0.937,1} Rolling                          & Y                                                                                       & N                                                                                      & N                                                                                         & \$     & Long                                                                                        & Low                                                                                       & Low                                                                                         \\
%\rowcolor[rgb]{0.941,1,1} Blue/Green                       & Y                                                                                       & N                                                                                      & N                                                                                         & \$\$\$ & Negligible                                                                                  & Medium                                                                                    & Medium                                                                                      \\
%\rowcolor[rgb]{1,0.937,1} Canary                           & Y                                                                                       & Y                                                                                      & N                                                                                         & \$     & Short                                                                                       & Low                                                                                       & Medium                                                                                      \\
%\rowcolor[rgb]{0.941,1,1} A/B Testing                      & Y                                                                                       & Y                                                                                      & Y                                                                                         & \$\$   & Short                                                                                       & Low                                                                                       & High                                                                                        \\
%\rowcolor[rgb]{1,0.937,1} Shadow                           & Y                                                                                       & Y                                                                                      & N                                                                                         & \$\$\$ & Negligible                                                                                  & Negligible                                                                                & High                                                                                       
%\end{tabular}
%\end{adjustwidth}
%\end{table}
%\end{frame}

\references{deployment}

\end{document}