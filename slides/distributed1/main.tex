\documentclass{slide}
\usepackage{tikz}

\usetikzlibrary{arrows}

\usepackage{tabto}

\usepackage{languages}

\title{Distributed Computing I}
\subtitle{CSSE6400}
\author{Brae Webb}
\date{\week{5}}

\titlegraphic {
    \tweet%
    {images/mathiasverraes}%
    {Mathias Verras}%
    {mathiasverraes}%
    {There are only two hard problems in distributed systems:  2. Exactly-once delivery 1. Guaranteed order of messages 2. Exactly-once delivery}%
    {https://twitter.com/mathiasverraes/status/632260618599403520}%
}

\begin{document}

\maketitle

\point[Previously in CSSE6400\dots]{
    Service-based Architecture
}

\note{Re-visiting service-based architectures from last lecture}

\begin{frame}{Previously in CSSE6400\dots}
    \vspace{1mm}
    {\LARGE
    \begin{description}
        \item[Simplicity] For a distributed system \tabto{15em}\includegraphics[width=8mm]{../../shared/images/thumbs-up.png}
        \item[Modularity] Services \tabto{15em}\includegraphics[width=8mm]{../../shared/images/thumbs-up.png}
        \item[Extensibility] New services \tabto{15em}\includegraphics[width=8mm]{../../shared/images/thumbs-up.png}
        \item[Deployability] Independent services \tabto{15em}\includegraphics[width=8mm]{../../shared/images/thumbs-up.png}
        \item[Testability] Independent services \tabto{15em}\includegraphics[width=8mm]{../../shared/images/thumbs-up.png}
        \item[Security] API layer \tabto{15em}\includegraphics[trim=57 145 70 85,clip,width=8mm]{../../shared/images/neutral.png}
        \item[Reliability] Independent services \tabto{15em}\includegraphics[trim=57 145 70 85,clip,width=8mm]{../../shared/images/neutral.png}
        \item[Interoperability] Service APIs \tabto{15em}\includegraphics[trim=57 145 70 85,clip,width=8mm]{../../shared/images/neutral.png}
        \item[Scalability] Coarse-grained services\tabto{15em}\includegraphics[trim=22 19 22 12,clip,width=8mm]{../../shared/images/thumbs-down.png}
    \end{description}
    }
\end{frame}

\note{Concluded on these attributes}

\begin{frame}{Previously in CSSE6400\dots}
    \vspace{1mm}
    {\LARGE
    \begin{description}
        \item[Simplicity] For a distributed system \tabto{15em}\includegraphics[width=8mm]{../../shared/images/thumbs-up.png}
        \item[Reliability] Independent services \tabto{15em}\includegraphics[trim=57 145 70 85,clip,width=8mm]{../../shared/images/neutral.png}
        \item[Scalability] Coarse-grained services\tabto{15em}\includegraphics[trim=22 19 22 12,clip,width=8mm]{../../shared/images/thumbs-down.png}
    \end{description}
    }
\end{frame}

\note{Let's revisit these attributes}

\begin{frame}{Previously in CSSE6400\dots}
    \vspace{1mm}
    {\LARGE
    \begin{description}
        \item[Simplicity] \highlight{For a distributed system} \tabto{15em}\includegraphics[width=8mm]{../../shared/images/thumbs-up.png}
    \end{description}
    }
\end{frame}

\note{This condition on simplicity is doing a lot of work}

\begin{frame}{Previously in CSSE6400\dots}
    \vspace{1mm}
    {\LARGE
    \begin{description}
        \item[Simplicity] \tabto{15em}\includegraphics[width=8mm]{../../shared/images/thumbs-down.png}
    \end{description}
    }
\end{frame}

\note{We'll look at a few reasons that distributed systems are \highlight{fundamentally} quite challenging}

\question{What is a \highlight{fallacy}?}

\definition{Fallacy}{
    Something that is believed or assumed to be true but is not.
}

\point[A \textsl{few} reasons for complexity]{
    The Fallacies of \highlight{Distributed Computing}
}

\note{Sun Microsystems in 1994, primarily accredited to Peter Deutsch (doy-ch)}

\point[Fallacy \#1]{The network is reliable}

\image{diagrams/Success}

\note{An expected interaction}

\image{diagrams/SendDrop}

\note{Packet gets lost on way to CardService}

\image{diagrams/SendDropResend}

\note{Solve it by resending it}

\image{diagrams/SendDropDos}

\note{If the service goes down and all clients are re-trying, the service is in for a shock when it comes back, we solve this with \highlight{exponential backoff}}

\begin{frame}[fragile]{Exponential backoff}
\begin{code}[language=python,morekeywords={do}]{}
retry = True
do:
    status = service.request()

    if status != SUCCESS:
        wait(2 ** retries)
    else:
        retry = False
while (retry and retries < MAX_RETIRES)
\end{code}
\end{frame}

\image{diagrams/ReceiveDrop}

\image{diagrams/ReceiveDropResend}

\note{Causes duplicate actions, problem for ordering/payments}

\image{diagrams/ReceiveDropToken}

\note{Use tokens to prevent duplicates.}

\point[Fallacy \#2]{Latency is zero}

\point[Network Statistics]{
    \begin{description}[leftmargin=!,labelwidth=\widthof{Data}]
        \item[Home to UQ] \only<2->{20.025ms}
        \item[Home to us-east-1] \only<3->{249.296ms}
        \item[EC2 to EC2] \only<4->{0.662ms}
    \end{description}
}

\note{Be mindful when designing distributed systems. Network call much slower then local call.}

\point[Fallacy \#3]{Bandwidth is infinite}

\note{Similar to previous fallacy, be mindful, distributed calls clog up network.}

\point[Fallacy \#4]{The network is secure}

\image[width=\textheight]{diagrams/Sahara}

\note{Authentication only occurs when entering Sahara data centre}

\image[width=0.75\textheight]{diagrams/NetworkInjection}

\note{Bad actor gets access via one insecure node, network is compromised. Practice defence in depth.}

\point[Fallacy \#5]{The topology never changes}

\note{Topology changes all the time, cloud has just made this easier. Don't rely on static IPs. Don't assume consistent latency.}

\point[Fallacy \#6]{There is only one administrator}

\point[Scenario]{
\begin{itemize}[<+->]
    \item Deployments are banned on the weekend.
    \item Sunday night users start complaining.
    \item There have been no deployments since Friday.
    \item You can still access the system.
    \item Who do you talk to?
\end{itemize}
}

\note{Things spontaniously break. Who can help you?}

\point[Fallacy \#7]{Transport cost is zero}

\point[Remember]{Distributed systems are \highlight{hard}.}

\point[Remember]{Distributed systems are often \highlight{not your friend}.}

\point[When you need to, prove it]{
    \centering \youtubevideo{images/isabelle-thumb}{https://youtube.com/watch?v=7w4KC6i9Yac}
}


\begin{frame}{Previously in CSSE6400\dots}
    \vspace{1mm}
    {\LARGE
    \begin{description}
        \item[Simplicity] For a distributed system \tabto{15em}\includegraphics[width=8mm]{../../shared/images/thumbs-up.png}
        \item[Reliability] Independent services \tabto{15em}\includegraphics[trim=57 145 70 85,clip,width=8mm]{../../shared/images/neutral.png}
        \item[Scalability] Coarse-grained services\tabto{15em}\includegraphics[trim=22 19 22 12,clip,width=8mm]{../../shared/images/thumbs-down.png}
    \end{description}
    }
\end{frame}

\begin{frame}{Previously in CSSE6400\dots}
    \vspace{1mm}
    {\LARGE
    \begin{description}
        \item[Reliability] Independent services \tabto{15em}\includegraphics[trim=57 145 70 85,clip,width=8mm]{../../shared/images/neutral.png}
     \end{description}
    }
\end{frame}

\question{What makes software \highlight{unreliable}?}

\point[`Working' software]{Satisfies the functional requirements}

\definition{Reliable Software}{Continues to work correctly, even when things go wrong.}

\definition{Fault}{Something goes wrong.}

\quote[Howard and LeBlanc]{Death, taxes, and computer system failure are all inevitable to some degree.\\\highlight{Plan for the event.}}

\point[Reliable software is]{Fault \highlight{tolerant}}

\note{John von Neumann built fault tolerant hardware in the 50s.}

\point[Problem]{Individual computers fail \highlight{all the time}}

\note{10-50 years hard-drive lifetime. 10,000 disks will fail daily.}

\point[Solution]{Spread the risk of faults over \highlight{multiple computers}}

\point[Spreading Risk]{
    \large
    If you have software that works with \highlight{just one} computer,
    spreading the software over \highlight{two} computers \highlight{halves} the risk that your software will fail.
    \\[2em]
    \only<2->{Adding \highlight{100} computers reduces the cuts the risk by \highlight{100}.}
    \extra[3-]{Of course, there are other reasons you might want run software on multiple computers.}
}

\image[width=\textheight]{diagrams/Sahara}

\begin{frame}{Previously in CSSE6400\dots}
    \vspace{1mm}
    {\LARGE
    \begin{description}
        \item[Simplicity] For a distributed system \tabto{15em}\includegraphics[width=8mm]{../../shared/images/thumbs-up.png}
        \item[Reliability] Independent services \tabto{15em}\includegraphics[trim=57 145 70 85,clip,width=8mm]{../../shared/images/neutral.png}
        \item[Scalability] Coarse-grained services\tabto{15em}\includegraphics[trim=22 19 22 12,clip,width=8mm]{../../shared/images/thumbs-down.png}
    \end{description}
    }
\end{frame}

\begin{frame}{Previously in CSSE6400\dots}
    \vspace{1mm}
    {\LARGE
    \begin{description}
        \item[Scalability] Coarse-grained services\tabto{15em}\includegraphics[trim=22 19 22 12,clip,width=8mm]{../../shared/images/thumbs-down.png}
     \end{description}
    }
\end{frame}

\question{Who has used \highlight{auto-scaling}?}

\point[Auto-scaling Terminology]{
\Large
\begin{description}[<+->]
    \item[Auto-scaling group] A collection of instances managed by auto-scaling.
    \item[Capacity] Amount of instances in an auto-scaling group.
    \item[Desired Capacity] Amount of instances we should have in an auto-scaling group.
    \item[Scaling Policy] How to determine the desired capacity.
    \item[Minimum/Maximum Capacity] Hard limits on the minimal and maximum amount of instances.
\end{description}
}

\point[What we really want]{
    \Large
    {\color{secondary}Desired Capacity} Amount of \highlight{healthy} instances we should have in an auto-scaling group.
}

\point[Health check]{
    User defined method to determine whether an instance is \highlight{healthy}.
}

\point[Auto-scaling]{An example}

\image[width=\textheight]{diagrams/Sahara}

\image{diagrams/ProductIsolate}

\image{diagrams/ProductEC2}

\image{diagrams/ProductScale}

\image{diagrams/ProductScaleReal}

\image[width=\textheight]{diagrams/ProductScaleLB}

\image[width=\textheight]{diagrams/SaharaScaled}

\begin{frame}{In Summary}
\Large
\begin{description}
    \item[Simplicity] \only<2->{\highlight{Minimal network communication} (compared to other distributed systems), less impacted by fallacies.}
    \item[Reliability] \only<3->{Traffic is spread to various services, still \highlight{partially operational} if one goes down. Auto-scaling allows for \highlight{basic replication}.}
    \item[Scalability] \only<4->{Auto-scaling and load balancing allows \highlight{individual services to scale}. However, the \highlight{database is a bottle-neck}.}  
\end{description}
\end{frame}

% \definition{Distributed Computing}{Multiple software components that are on multiple computers, but run as a single system}

% \point[The Problem]{
%     \begin{center}
%     \begin{tikzpicture}
%         \node[circle,draw, minimum size=1cm] (A) at  (0,0) {Distributed};
%         \node[circle,draw, minimum size=1cm] (B) at  (8,0)  {Simplicity};
%         \draw[<->] (A) -- (B);
%     \end{tikzpicture}
%     \end{center}
% }

% \point{A lot of modern software development focuses on dealing with the \highlight{complexity} of distributed systems.}

% \question{What makes distributed computing complex?}

% \point{
%     \begin{itemize}
%         \item Faults
%         \item Asynchronous communication
%         \item Monitoring
%         \item And much more\dots
%     \end{itemize}
% }


% \references{articles,books}

\end{document}