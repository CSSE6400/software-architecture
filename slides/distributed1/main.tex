\documentclass{slide}

% \usepackage{pgfpages}
\usepackage{tabto}
\usepackage{languages}
\usepackage{tikz}
\usetikzlibrary{arrows}
\usepackage{scrextend}  % For \begin{labeling

%\setbeameroption{show notes on second screen}

\title{Distributed Systems I}
\subtitle{Software Architecture}
\author{Brae Webb \& Richard Thomas}
\date{\week{5}}

\titlegraphic {
    \tweet%
    {images/mathiasverraes}%
    {Mathias Verras}%
    {mathiasverraes}%
    {There are only two hard problems in distributed systems:\\  2. Exactly-once delivery\\ 1. Guaranteed order of messages\\ 2. Exactly-once delivery}%
    {https://twitter.com/mathiasverraes/status/632260618599403520}%
}

\begin{document}

\maketitle
\note{Lecture Goal: Balance a healthy love-hate relationship with distributed systems}

\point[\Large Going forward]{
    Investigating architectures that are \highlight{distributed}.
}

\point[\Large Distributed Systems Series]{
    \begin{labeling}{Distributed III}
        \item[\color{secondary} Distributed I] \highlight{Reliability} and \highlight{scalability} of \highlight{stateless} systems.
        \item[\color{secondary} Distributed II] \highlight{Complexities} of \highlight{stateful} systems.
        \item[\color{secondary} Distributed III] \highlight{Hard problems} in distributed systems.
    \end{labeling}
}

\point[\Large What are the benefits?]{
    \begin{itemize}
        \item Improved \highlight{reliability}
        \item Improved \highlight{scalability}
        \item Improved \highlight{latency}
    \end{itemize}
}
\note{Some systems are inherently distributed.}

\point[\Large What are the drawbacks?]{
    \begin{itemize}
        \item Increased \highlight{complexity}
        \item Increased \highlight{attack surface}
        \item Increased \highlight{latency}
        \item Introduce \highlight{consistency} problems
    \end{itemize}
}
\note{We'll look at a few reasons that distributed systems are \highlight{fundamentally} quite challenging}


\section{Fallacies}

\point[\Large A \textit{few} reasons for complexity]{
    Fallacies of \highlight{Distributed Computing}
}
\note{Sun Microsystems in 1994, primarily accredited to Peter Deutsch (doy-ch)}

\point[\Large Fallacy \#1]{The network is reliable}

\image{diagrams/Success}
\note{Success: Send request to add item to cart.}

\image{diagrams/SendDrop}
\note{Failure 1: Request not received by server (CartService).}

\image{diagrams/SendDropResend}
\note{Failure 1: \textit{Solution} resend request?}

\image{diagrams/SendDropDos}
\note{
If the service goes down and all clients are re-trying,\\
the service is in for a shock when it comes back,\\
we solve this with \highlight{exponential backoff}.
}

\begin{frame}[fragile]{Exponential Backoff}
\begin{code}[language=python,morekeywords={do}]{}
retries = 0
do:
    status = service.request()

    if status != SUCCESS:
        retries += 1
        wait(2 ** retries)
while (status != SUCCESS and retries < MAX_RETRIES)
\end{code}
\end{frame}

\image{diagrams/ReceiveDrop}
\note{Failure 2: Response not received.}

\image{diagrams/ReceiveDropResend}
\note[itemize]{
\item Failure 1's \textit{solution} of resending request leads to:
\item Duplicate actions, problem for ordering/payments.
}

\image{diagrams/ReceiveDropToken}
\note{Solution: Use tokens to prevent duplicates.}

\point[\Large Fallacy \#2]{Latency is zero}

\point[\Large Network Statistics]{
    \begin{description}[leftmargin=!,labelwidth=\widthof{Data}]
        \item[Home to UQ] \only<2->{20.025ms}
        \item[Home to us-east-1] \only<3->{249.296ms}
        \item[EC2 to EC2] \only<4->{0.662ms}
    \end{description}
}
\note[itemize]{
\item Be mindful when designing distributed systems.
\item Network call \highlight{much} slower than local call.
}

\point[\Large Fallacy \#3]{Bandwidth is infinite}
\note{Similar to previous fallacy, be mindful,\\ distributed calls clog up network.}

\definition{Stamp Coupling}{
    Components which share a composite data structure.
}

\point[\Large Fallacy \#4]{The network is secure}

\image[height=1.07\textheight]{diagrams/Sahara}
\note{Authentication only occurs when entering Sahara data centre.}

\image[height=1.1\textheight]{diagrams/NetworkInjection}
\note[itemize]{
\item Bad actor gets access via one insecure node.
\item Network is \highlight{compromised}.
\item Practice \highlight{defence in depth}.
}

\point[\Large Fallacy \#5]{The topology never changes}
\note[itemize]{
\item Topology changes all the time, cloud makes this \highlight{easier}.
\item Don't rely on static IPs.
\item Don't assume consistent latency.
}

\point[\Large Fallacy \#6]{There is only one administrator}
\note[itemize]{
\item Things spontaneously break.
\item Who can help you?
}

\point[\Large Fallacy \#7]{Transport cost is zero}

\point[\Large Remember]{
\vspace{1em}
Distributed systems are \highlight{hard}.

\vspace{1em}
Choosing to use them should be \highlight{well considered}.
}
\note{Can often introduce more problems than they solve.}

\point[\Large When you need to, maybe prove it?]{
    \centering \youtubevideo{images/isabelle-thumb}{https://youtube.com/watch?v=7w4KC6i9Yac}
}
\note[itemize]{
    \item Presentation: Proving correctness of distributed systems.
    \item by Dr. Martin Kleppmann, University of Cambridge.
    \item Using Isabelle -- See CSS research sub-group.
}

\point[\Large Or, more realistically,]{
    Use \highlight{existing} algorithms and software
}

\point[\Large Distributed Systems Series]{
    \begin{labeling}{Distributed III}
        \item[\color{secondary} Distributed I] \highlight{Reliability} and \highlight{scalability} of \highlight{stateless} systems.
        \color{gray}{
        \item[Distributed II] Complexities of stateful systems.
        \item[Distributed III] Hard problems in distributed systems.
        }
    \end{labeling}
}

\point[\Large Stateless vs. Stateful Systems]{
\vspace{1em}
    \begin{labeling}{Stateless}
        \item[\color{secondary} Stateless] Does \highlight{not} utilise \highlight{persistent data}.
        \vspace{1em}
        \item[\color{secondary} Stateful] Does utilise \highlight{persistent data}.
    \end{labeling}
}

\question{\Large What makes software \highlight{reliable}?}

\definition{Reliable Software}{Continues to work, even when things go wrong.}

\definition{Fault}{Something goes wrong.}

\quote[Howard and LeBlanc]{Death, taxes, and computer system failure are all inevitable to some degree.\\
\vspace{1em}\highlight{Plan for the event.}}

\point[\Large Reliable software is]{Fault \highlight{tolerant}}
\note{John von Neumann built fault tolerant hardware in the 1950s.}

\point[\Large Problem]{Individual computers fail \highlight{all the time}}
\note[itemize]{
\item 10-50 years hard-drive lifetime.
\item 10,000 disks will fail daily.
\item Google last had 2.5 million servers.
\item ...
}

\point[\Large Solution]{Spread the risk of faults over \highlight{multiple computers} or \highlight{nodes}}

\point[\Large Spreading Risk]{
    \vspace{1em}
    \LARGE
    If you have software that works with \highlight{just one} computer,
    spreading the software over \highlight{two} computers \highlight{halves} the risk that your software will fail.
    \\[2em]
    \only<2->{Adding \highlight{100} computers reduces the risk by \highlight{100}.}
}

\image[height=1.085\textheight]{diagrams/Sahara}
\note[itemize]{
    \item Why is this software somewhat reliable?
    \item Any individual service can go down and the rest still work.
    \item Can we do better?
    \item Can a service go down but have that service still work?
}

\question{Who has used \highlight{auto-scaling}?}

\begin{frame}[fragile]{Auto-Scaling Terminology}
\Large
\begin{description}[<+->]
    \item[Auto-scaling group] A \highlight{collection of instances} managed by auto-scaling.
    \item[Capacity] Amount of instances \highlight{currently} in an auto-scaling group.
    \item[Desired Capacity] Amount of instances \highlight{we want to have} in an auto-scaling group.
    \item[Scaling Policy] How to determine the desired capacity.
    \item[Minimum/Maximum Capacity] \highlight{Hard limits} on the minimum and maximum number of instances.
\end{description}
\end{frame}

\point[\Large What we really want]{
    {\color{secondary}Desired Capacity} Amount of \highlight{healthy} instances we want to have in an auto-scaling group.
}

\point[\Large Health check]{
    Mechanism to determine whether an instance is \highlight{healthy}.
}

\point[\Large Auto-scaling]{An example}

\image[height=1.085\textheight]{diagrams/Sahara}
\note{Product Browsing service keeps going down}

\image{diagrams/ProductIsolate}
\note{We might expect product browsing to have a much higher load than other services}

\image{diagrams/ProductEC2}

\image{diagrams/ProductScale}
\note{Use an auto-scaling group to replicate the service}

\image{diagrams/ProductScaleReal}
\note{What's the problem?}

\image[height=1.1\textheight]{diagrams/ProductScaleLB}
\note[itemize]{
    \item Traffic was all sent through the \highlight{one} instance
    \item Load balancer routes to \highlight{all}
}

\image[width=\textheight]{diagrams/SaharaScaled}

\begin{frame}{In Summary}
\Large
\begin{labeling}{Scalability}
%\begin{description}
    \item[\color{secondary} Simplicity] \only<2->{\highlight{Minimal network communication} (compared to other distributed systems), less impacted by fallacies.}
    \vspace{2mm}
    \item[\color{secondary} Reliability] \only<3->{Traffic is spread to various services, still \highlight{partially operational} if one goes down. Auto-scaling allows for \highlight{basic replication}.}
    \vspace{2mm}
    \item[\color{secondary} Scalability] \only<4->{Auto-scaling and load balancing allows \highlight{individual services to scale}. However, the \highlight{database is a bottleneck}.}  
\end{labeling}
\end{frame}
\note{\highlight{Database is a bottleneck} is foreshadowing}

% \references{articles,books}

\end{document}
