\documentclass{csse4400}

\usepackage{tikz}
\usetikzlibrary{positioning}
\usetikzlibrary{arrows}

\usepackage{float}

\usepackage{enumitem}

\usepackage{languages}

\title{Containers}
\author{Brae Webb}

\date{\week{3}}
\begin{document}

\maketitle

\section{Before Class}
\begin{itemize}
    \item Ensure that you have \textbf{Docker installed} prior to class.
    \item Preferably read the Containers lecture note \cite{container-notes}.
\end{itemize}

\section{Brief}
This week we are getting our hands on using containers.
Specifically we will:
\begin{enumerate}
    \item Check our installation with the \texttt{hello-world} image.
    \item Manage our local docker images.
    \item Build a Docker image from a \texttt{Dockerfile}.
    \item Run the Docker image interactively.
    \item Publish the Docker image.
\end{enumerate}

\section{Hello World}
To ensure that our installation of Docker is working as expected,
we'll run the \texttt{hello-world} image.
The \texttt{hello-world} image is an 
\link{official Docker image}{https://hub.docker.com/search?q=&type=image&image_filter=official}
that simply prints out ``Hello from Docker!''.
It is similar to the \texttt{Hello FROM scratch} we looked at in the lecture notes.

\bash{docker run hello-world}

\section{Exploring Local Images}
We'll look at a few commands to explore our local images.

\paragraph{List all local Docker images}

\bash{docker images}

\paragraph{Remove local image}

\bash{docker rmi -f hello-world}

\paragraph{Clean up unused images}

\bash{docker image prune -af}

\section{Creating an image}

We can create images to run tools which run in an isolated environment.
For this exercise,
we will create a \texttt{Dockerfile} to build an image with
\link{Advanced Normalization Tools (ANTs)}{https://github.com/ANTsX/ANTs}
installed.

\codefile[language=docker]{Dockerfile}{code/Dockerfile}

Once the image has been specified in a \texttt{Dockerfile},
we can use the \texttt{docker build} command to run the commands and produce an image.
Run the following in the same directory as the \texttt{Dockerfile}.

\bash{docker build -t ants:latest -f Dockerfile .}

Note that if the tag \texttt{ants} is given \texttt{:latest} will be appended automatically.
Likewise, if the file is named \texttt{Dockerfile} then the \texttt{-f Dockerfile} flag can be dropped as Docker will automatically search for a \texttt{Dockerfile}.
Check that the image has been created:

\bash{docker images}

Now we are ready to run the image.
When running with the \texttt{-it} flag,
Docker will create an interactive terminal inside the container and connect it to the current shell.
You should notice that the shell prompt will change.

\bash{docker run -it ants:latest}

Now you can inspect the image,
note that the current directory has all the binaries of ANTs.
If you inspect other directories you'll find all your normal files are inaccessible.

You can exit the interactive container at any time by running the \texttt{exit} command to exit the shell within the container and return to the shell on your host machine.

\bash{exit}

\section{Publishing the container}
\dots

% \section{Going further}
% One of the trickier aspects of configuring containers is configuring an appropriate build environment for a tool.
% In the previous exercise we simply downloaded all the binaries for 

\bibliographystyle{ieeetr}
\bibliography{ours}

\end{document}