\documentclass{csse4400}

% \teachermodetrue

\usepackage{float}

\usepackage{languages}

\title{Application Programming Interfaces (APIs)}
\author{Brae Webb}

\date{\week{1}}
\begin{document}

\maketitle

\begin{figure}[h]
  \href{https://www.oreilly.com/library/view/designing-data-intensive-applications/9781491903063/ch04.html}{
    \includegraphics[width=\textwidth]{images/communication}
  }
\caption{A map of communication techniques from Designing Data-Intensive Applications \cite{data-intensive}.}
\end{figure}

\section{This Week}
This week our goal is to:
\begin{itemize}
  \item explore the various techniques developers use to communicate between components in distributed systems;
  \item build a skeleton of a todo app using the Flask framework;
\end{itemize}

\teacher{
    Spend approximately the first 30 minutes exploring concepts.
}

\section{Communication}
The world relies (heavily) on distributed systems,
there is no machine in the world powerful enough to process the requests Google receives every second%
\footnote{Current estimates are that Google requires over a million servers}.
Even systems of a much smaller scale can still need many hundreds or thousands of machines working seamlessly together.
This inter-machine teamwork requires machines to know how to talk to each other.
For the practical this week we will look at APIs as a mechanism for communication.

\section{Data Formats}
Communication requires an exchange of information.
On computer systems information is stored at runtime in memory as primitive data which your programming language can interpret;
bytes, integers, strings, etc.
In object-oriented languages,
primitive data is wrapped up into useful packages: objects.
If we want this information to escape the confines of our programming language runtime,
we need to package it up in a language-independent format.
For a format to be language-independent we just need many languages to implement an encoding and decoding mechanism for the format.
We have several language-independent formats available but a few defacto standards.

\subsection{XML}
Extensible Markup Language (XML) is one of the most widely used language-independent formats.
The use cases of XML are extensive,
it is the \link{foundation for many popular utilities}{https://en.wikipedia.org/wiki/List_of_XML_markup_languages},
such as SVG file formats, SAML authentication, RSS feeds, and ePub books.

\begin{code}[language=xml]{csse6400.xml}
<root>
  <item>
    <key>Course Code</key>
    <value>CSSE6400</value>
  </item>
  <item>
    <key>Course Title</key>
    <value>Software Architecture</value>
  </item>
</root>
\end{code}

XML is designed as a markup language, similar to HTML,
it is not designed as a data exchange format.
Developers have come to point out that the verbosity and complexity of XML,
compared to alternatives such as JSON, are deal breakers.
While XML can be used as a data exchange format it is not designed for it,
and as a result APIs built around XML as a data format are becoming less common.

\subsection{JSON}
JavaScript Object Notation (JSON) is quickly replacing XML as the data format used in APIs.
As you will note, it is more succinct and communicates the important points to a human reader better.
The popularity of JSON is largely due to its compatibility with JavaScript which has taken over web development.
JSON is the map-esque data type in JavaScript.
Detractors of JSON claim that its main disadvantage compared to XML is that it lacks a schema.
However, \link{schemas are possible in JSON}{https://json-schema.org/},
they are optional, just as in XML, but are used much less than in XML.

\begin{code}[language=json]{csse6400.json}
{
    "Course Code": "CSSE6400",
    "Course Title": "Software Architecture"
}
\end{code}

\subsection{MessagePack}
It should not be a surprise that the JSON and XML formats are not resource efficient.
Nowadays, we are less concerned with squeezing data into a tiny amount of data on the hard drive as our hard drives are massive.
However, we are often concerned with how much data is being transmitted via network communication.

In the example JSON snippet above, we use 78 bytes to encode the message.
\link{MessagePack}{https://msgpack.org/} is a standard for encoding and decoding JSON.
When encoding our original JSON snippet with MessagePack we shrink to just 57 bytes.
At our scale, a negligible difference,
but at the scale of terrabytes or petabytes,
a significant consideration.

\begin{code}[]{csse6400.msgpack}
82 ab 43 6f 75 72 73 65 20 43 6f 64 65 a8 43 53 53 45 36 34 30 30 ac 43 6f 75 72 73 65 20 54 69 74 6c 65 b5 53 6f 66 74 77 61 72 65 20 41 72 63 68 69 74 65 63 74 75 72 65
\end{code}

\info{
    For those interested,
    0x82 specifies a map type (0x80) with two fields (0x02).
    Followed by a string type (0xa0) of size eleven (0x0b).
    The rest is left as an exercise: \url{https://github.com/msgpack/msgpack/blob/master/spec.md}.
}

\subsection{Protobuf}
Protocol Buffers (protobuf) is another type of binary encoding.
However, unlike MessagePack, the format was designed from scratch,
allowing a more compact and better designed format.
Protobufs require all data to be defined by a schema.
For example:

\begin{code}[language=protobuf]{csse6400.proto}
message Course {
    required string code = 1;
    required string name = 2;
}
\end{code}

Protobufs differ from XML, JSON, and MessagePack via their method of integration.
In the previous examples,
your language would have a library to encode and decode the data format into and out of your language's type system.
With protobuf, an external tool, \textsl{protoc},
takes the schema and generates a model of the schema in your target language.
This gives every language a native method to interact with the data format,
it often means that developers do not need to be aware of the underlying encoding.

\begin{code}[language=java]{csse6400.java}
Course softarch = Course.newBuilder()
    .setCode("CSSE6400")
    .setName("Software Architecture")
    .build();
output = new FileOutputStream(args[0]);
softarch.writeTo(output);
\end{code}

\section{Application Programming Interfaces}
It is worth being aware of a few of the common API paradigms available.
We outline a few but note that in this course we will primarily focus on REST APIs.
This is only an indication that they are better understood by course staff,
not their superiority.

\subsection{XML-RPC}
XML-RPC was one of the first API standards.
It is a Remote Procedure Call (RPC) based API which communicates via XML.
It is not a commonly used standard anymore.

\subsection{SOAP}
After XML-RPC came SOAP.
SOAP was impactful for service-oriented (now microservices) architectures but has largely fallen out of favour in-place of REST APIs.
We do not cover SOAP in any meaningful depth due to its increasing irrelevance and complexity.

\subsection{REST}
REST is not a standard like SOAP once was,
instead REST is an architectural style guided by a set of architectural constraints that allows us to build flexible APIs.

In this course we do not dive too deep into the architectural style and instead opt for a more surface level understanding.
It is a common mistake for people to refer to REST as a HTTP based web service API,
they are different.
In this course we chose to embrace this mistake and often refer to a HTTP based web service API when saying REST.

\noindent An example of this type of API might be:
\begin{description}
  \item[GET /api/v1/todo] List all tasks todo
  \item[POST /api/v1/todo] Create a task todo
  \item[GET /api/v1/todo/{id}] List all details about a certain task
  \item[PUT /api/v1/todo/{id}] Update the fields of an existing task
  \item[DELETE /api/v1/todo/{id}] Delete a specific task
\end{description}

\subsection{JSON-RPC}
A JSON RPC is another RPC based API based around communication via JSON.
When deciding whether to use an RPC-based API or a REST/SOAP based API,
the common distinction to make is that RPC APIs should be action based, i.e. I want to do X.
Whereas REST/SOAP APIs are Create Read Update Delete (CRUD) based for providing a interface to mutable data.

\subsection{GraphQL}
GraphQL is a query language which allows more dynamic querying of data than REST based APIs.
You can create nested queries and specify the fields you want to receive in response.
GraphQL APIs are well-suited for building APIs designed for developers to consume but when dealing with rigid inter-service based requests,
REST APIs are generally preferable.

\subsection{gRPC}
Another RPC framework, gRPC has started gaining a lot of attention since its first release in 2016.
gRPC is based on the protobuf communication standard.
In addition to a more type-safe system,
gRPC based APIs provide authentication and streaming mechanisms.

\section{Creating our own Todo app}

\subsection{The API design}

\todo{Add description of all the API endpoints}

\subsection{Implementation with Flask}

\todo{Intro to getting python up and running}
\todo{Intro to flask}
\todo{returning json with flask}
\todo{Calling your api locally}

\bibliographystyle{ieeetr}
\bibliography{books,ours}

\end{document}