\documentclass{csse4400}

\usepackage{CJKutf8}
\usepackage{languages}

\title{Scalable Text-to-Speech}
\author{Brae Webb}
\date{Semester 1, 2022}

\begin{document}
\maketitle

\section*{Summary}
In this assignment, you will demonstrate your ability to \textsl{design},
\textsl{implement}, and \textsl{deploy} a web application that can process a high load,
i.e. a scalable application.
You will be asked to deploy a tool that accepts text input and generates synthesized speech output.
Specially you need to support:
\begin{itemize}
    \item Uploading and generating a large text input.
    \item Generating speech output while remaining responsive to the user.
    \item Access via a specified REST API for use by front-end interfaces.
\end{itemize}

Your service will be deployed to AWS and will undergo automated correctness and load-testing to ensure it meets the required scalability.

\section{Introduction}
Text-to-speech software supports accessibility,
enables smart-home devices,
and even breaks down language barriers.
Unfortunately, text-to-speech is computationally intensive.
While the technology has made great advances over the past few decades,
many open-source implementations are still inefficient.

\paragraph{Task}
For this assignment,
the University of Queensland is looking to convert all course content into speech.
This will support visually impaired students in their studies.
All course content from slack messages and blackboard announcements to text-books must be converted to speech.
You will be responsible for designing and implementing this service for use across the entire university.

\paragraph{Requirements}
As you might imagine,
blackboard announcements occur frequently and should be translated in almost real time.
While textbooks are set ahead of semester and may take several days to process.
The university will experience peaks of usage.
At the start of semester,
instructors many set textbooks which need to be processed.
The university will also experience usage lows over the summer holiday period when no translations will be required.
The university is not willing to pay for the resources required during start of semester all year round.
Your implementation must be able to scale dynamically based on the current amount of jobs to be processed.

% \section{Features}


\section{Interface}
Your service will be utilised by almost every system in the university.
It has been decided that every university services must support text-to-speech on the first Monday of semester two.
An interface specification has already been developed and distributed to existing service owners,
who are working hard to deliver support in their services.

You must implement this interface exactly as described,
the interface specification is available to all service owners online:
\url{https://csse6400.uqcloud.net/assessment/alexa}

\section{Submission}
Your final solution will be committed to a GitHub repository.
The repository \textbf{must} contain everything required to successfully deploy your application.
You must include all of the following in the repository:
\begin{itemize}
  \item Your implementation of the service API, including the source code and a mechanism to build the service via Docker.
  \item Terraform code that can provision your service from a fresh AWS learner lab.
\end{itemize}

\todo{
  We need to think about how we want to require deployment.
  If we assume that they will use containers to deploy we should give an example of using terraform to build, upload, and use a container as shown below.

  We have a few options:
  \begin{enumerate}
    \item Require that deployment occurs via \texttt{terraform apply}
    \item Require that deployment occurs via \texttt{./deploy.sh}
    \item Require that sufficient deployment documentation is provided in the \texttt{README.md}
  \end{enumerate}

  I'm inclined to pick both 2 and 3 so that when 2 fails we can try 3.

  We additionally need to specify the exact environment where we will run the deployment script.
  This should be a container with Terraform and Docker (via mounting the host) and the AWS credentials and project mounted in the workdir.
}

\begin{code}{}
resource "null_resource" "build_backend" {
  provisioner "local-exec" {
    command = "docker build -t backend ./backend"
  }
}


\end{code}

\section{Criteria}

\section{Academic Integrity}

% \bibliographystyle{ieeetr}
% \bibliography{references}

\end{document}