\documentclass{csse4400}

\usepackage{languages}
\usepackage{fancyhdr}

% RUBRIC
\usepackage{multirow}
\usepackage{array}
\usepackage{xltabular}
\usepackage{pdflscape}
\usepackage{enumitem}
\usepackage{xcolor}

\newcolumntype{P}[1]{>{\centering\arraybackslash}p{#1}}
% RUBRIC

\title{CoughOverflow}
\author{Richard Thomas}
\date{Semester 1, 2025}

\begin{document}

\input{copyright-footer}
\maketitle

%\warning{This document is a draft and is subject to change.}

\section*{Summary}
In this assignment, you will demonstrate your ability to \textit{design},
\textit{implement}, and \textit{deploy} a web API that can process a high load,
i.e. a scalable application.
You are to deploy an API to analyse images of saliva samples to identify if a patient has COVID-19
or avian influenza (H5N1), which is commonly called bird flu.
Specially your application needs to support:
\begin{itemize}
    \item Analysing an image received via an API request.
    \item Providing access to a specified REST API, e.g. for use by front-end interfaces and other applications.
    \item Remaining responsive while analysing images.
\end{itemize}

\noindent
Your service will be deployed to AWS and will undergo automated correctness and load-testing to ensure it meets the requirements.


\section{Introduction}

For this assignment, you are working for CoughOverflow, a new UQ start-up.
CoughOverflow uses machine learning techniques developed by
\link{QDHeC}{https://chsr.centre.uq.edu.au/research/queensland-digital-health-centre}
to analyse images of saliva samples.
The analysis is able to identify if an individual is infected with one of a few pathogens.
The initial service focuses on identification of COVID-19 or H5N1,
due to their current level of risk to the public.

\paragraph{Task}
CoughOverflow uses a microservices architecture to implement their analysis platform.
The CTO saw on your resume that you are taking Software Architecture
and has assigned you to design and implement the pathogen analysis service.
This service must scale to cope with the anticipated large number of tests.

\paragraph{Requirements}
Automated identification of pathogens is an important service.
Manual testing by lab technicians is labour intensive and time consuming.
Automated tests free lab technicians for more important work,
and provide faster responses to healthcare staff.
This is critical in an epidemic or pandemic scenario,
when tens or hundreds of thousands of tests need to be performed daily.

CoughOverflow's pathogen analysis service (PAS) needs to be designed to scale to match demand.
Pathology labs will obtain saliva samples from patients.
The labs will create images of the cells in the samples.
These images will be sent to the PAS for analysis.

The algorithms used to analyse the images are computationally intensive.
It is not possible to return a result immediately for a submitted image.
Labs, or other healthcare providers, will need to query the PAS to obtain results at a later time.
Results can be queried for a single test, or a batch of tests for a lab or patient.

As COVID-19 and H5N1 are potentially life threatening to some patients,
the service must be able to provide test results in a timely manner.
Early treatment and effective isolation practices can significantly reduce the impacts of these diseases,
as well as reducing strain on healthcare resources.

Persistence is an important characteristic of the platform.
Resubmitting analysis requests due to lost data would place unnecessary strain on pathology labs,
at times when they may be under extreme pressure to deliver results.
Upon receiving an analysis request, and after error checking,
the PAS must guarantee that the data has been saved to persistent storage before returning a success response.


\section{Interface}
As you are operating in a microservices context,
other service providers have been given an API specification for your service.
They have been developing their services based on this specification so you must match it \emph{exactly}.

The interface specification is available to all service owners online: 

\url{https://csse6400.uqcloud.net/assessment/coughoverflow}


\section{Implementation}
The following constraints apply to the implementation of your assignment solution.

\subsection{Analysis Engine}

You have been provided with a command line tool called \texttt{engine} that must be used to analyse sample images.
This tool was developed by AI and medical researchers at
\link{QDHeC}{https://chsr.centre.uq.edu.au/research/queensland-digital-health-centre}.
The tool has varying performance, due to how clear the pathogen markers are in the cell sample images.
You will have to work around this bottleneck in the design and development of the PAS.

Your service \textbf{must} utilise the \texttt{engine} command line tool provided for this assignment.
The compiled binaries are available in the tool's GitHub repository: \url{https://github.com/CSSE6400/pathogenanalysis}.

\warning{You are \textbf{not} allowed to reimplement or modify this tool.}

The analysis engine requires pre-processing of cell sample images to highlight pathogen markers.
This pre-processing is done by the pathology labs.
You \textbf{must} use the provided sample images for testing purposes.
If you try to generate your own images, they are likely to fail analysis or give false results.

%For the purposes of this assignment,
%in the API specification you will notice the metadata->spamhammer field in the POST body.
%This is a setting which decides whether the email is malicious and how long processing should take.
%Demonstrations are provided in the repository to show how to use the tool to generate your own examples.
%You \textbf{must} not use this field to simplify the task of scanning the email.

\subsection{AWS Services}
Please make note of the \link{AWS services}
{https://labs.vocareum.com/web/2460291/1564816.0/ASNLIB/public/docs/lang/en-us/README.html\#services}
that you can use in the AWS Learner Lab, and the limitations that are placed on the usage of these services.
To view this page you need to be logged in to your AWS Learner Lab environment and have a lab open.

\subsection{External Services}
You may \textbf{not} use services or products from outside of the AWS Learner Lab environment.
For example, you may not host instances of the \texttt{engine} command line tool on another cloud platform
(e.g. Google Cloud).

You may \textbf{not} use services or products that run on AWS infrastructure external to your Learner Lab environment.
For example, you may not deploy a third-party product like MongoDB Atlas on AWS and then use it from your service.

You may \textbf{not} deploy machine learning or GPU backed services.


\section{Submission}
This assignment has three submissions.

\begin{enumerate}[topsep=7pt,partopsep=2pt,itemsep=4pt,parsep=4pt]
  \item April 4$^{th}$ -- API Functionality
  \item April 17$^{th}$ -- Deployed to Cloud
  \item May 9$^{th}$ -- Scalable Application
\end{enumerate}
All submissions are due at 15:00 on the specified date.
Your solution for each submission must be committed and pushed to the GitHub repository specified in Section \ref{sec:github}.

%Each submission is to be tagged\footnote{Atlassian has a good tutorial about Git tag,
%if you are not familiar with tagging.
%See: \url{https://www.atlassian.com/git/tutorials/inspecting-a-repository/git-tag}.}
%to indicate which commit is to be marked.
%The tags that you \textbf{\underline{must}} use are:
Each submission is to be in its own branch.
You \textbf{\underline{must}} use the branch names \textbf{\underline{exactly}} as indicated below.
Failure to use these branch names may result in your submission not being marked,
and you obtaining a grade of 0 for the submission.
\begin{itemize}[topsep=7pt,partopsep=2pt,itemsep=4pt,parsep=4pt]
  \item \textbf{stage-1} for API Functionality, due on April 4$^{th}$
  \item \textbf{stage-2} for Deployed to Cloud, due on April 17$^{th}$
  \item \textbf{stage-3} for Scalable Application, due on May 9$^{th}$
\end{itemize}

When marking a stage, we will checkout the branch for that stage.
Any code in the main branch or \textit{any} other branch, will be ignored when marking.
We will checkout the latest commit in the branch being marked.
If the commit date and time is after the submission deadline, late penalties will be applied,
unless you have an extension.
Late penalties are described in the course profile.

\textbf{\underline{Note:}} Experience has shown that the large majority of students who make a late submission,
lose more marks from the late penalty than they gain from any improvements they make to their solution.
We \textit{strongly} encourage you to submit your work on-time.

%If you forget to tag your submission,
%we will checkout and mark the latest commit that you made to the main branch before the submission deadline.
You should commit and push your work to your repository regularly.
If a misconduct case is raised about your submission,
a history of regular progress on the assignment through a series of commits
could support your argument that the work was your own.

Extension requests \textbf{must} be made \emph{prior} to the submission deadline via \link{my.UQ}{https://my.uq.edu.au/}.

Your repository \textbf{must} contain everything required to successfully deploy your application.

%\begin{samepage}
\subsection{API Functionality Submission}
Your first submission \textbf{must} include all of the following in your repository:
\begin{itemize}
  \item Docker container (Dockerfile) of your implementation of the service API,
        including the source code and a mechanism to build and run the service.%
        \footnote{If you use external libraries,
                  ensure that you pin the versions to avoid external changes breaking your application.}
  \item A \texttt{local.sh} script that can be used to build and run your service locally.
        This script \textbf{must} be in the root directory of your repository.
        The \texttt{local.sh} script must launch your container with port 8080 being passed from the container
        to the testing environment, and your service \textbf{must} be available at \texttt{http://localhost:8080/}.
\end{itemize}
We will run a suite of tests against your API at this URL.
%\end{samepage}

\newpage
\begin{samepage}
\subsection{Deployed to Cloud \& Scalability Submissions}
The second and third submissions \textbf{must} include all of the following in your repository:
\begin{itemize}
  \item Your implementation of the service API, including the source code and a mechanism to build the service.%
        \footnote{If you use external libraries,
                  ensure that you pin the versions to avoid external changes breaking your application.}
  \item Terraform code that provisions your service in a fresh AWS environment.
  \item A \texttt{deploy.sh} script that uses your Terraform code to deploy your application.
        This script \textbf{must} be in the root directory of your repository.
        This script may perform other tasks as required by your implementation.
\end{itemize}
\end{samepage}

\noindent
When deploying your second and third submissions to mark, we will follow reproducible steps, outlined below.
You may re-create the process yourself.

\begin{enumerate}
  \item Your Git repository will be cloned locally.
        The submission branch will be checked out.
  \item AWS credentials will be copied into your repository in the root directory,
        in a file called \texttt{credentials}.
  \item The script \texttt{deploy.sh} in the \textbf{root directory} will be run.
  \item The \texttt{deploy.sh} script \textbf{must} create a file named \texttt{api.txt},
        which contains the URL at which your API is deployed,
        e.g. \texttt{http://my-api.com/} or \texttt{http://123.231.213.012/}.
  \item We will run automated functionality and load-testing on the URL provided in the \texttt{api.txt} file.
\end{enumerate}

\noindent
\textbf{Important Note:} Ensure your service does not exceed the resource limits of AWS Learner Labs.
For example, AWS will \textbf{deactivate} your account if more than fifteen EC2 instances are running.
If you use up your allocated budget in the Learner Lab, you will not be able to run any services.

\subsection{GitHub Repository}\label{sec:github}
You will be provisioned with a private repository on GitHub for this assignment, via GitHub Classroom.
You must click on the link below and associate your GitHub username with your UQ student ID in the Classroom.

\url{https://classroom.github.com/a/whdIS1AE}

\noindent
Associating your GitHub username with another student's ID,
or getting someone else to associate their GitHub username with your student ID, is \link{academic misconduct}
{https://my.uq.edu.au/information-and-services/manage-my-program/student-integrity-and-conduct/academic-integrity-and-student-conduct}.

If for some reason you have accidentally associated your GitHub username with the wrong student ID,
contact the course staff as soon as possible.

\subsection{Tips}

\paragraph{Terraform plan/apply hanging}
If your \texttt{terraform plan} or \texttt{terraform apply} command hangs without any output, check your AWS credentials. Using credentials of an expired Learner Lab session will cause Terraform to hang.

\paragraph{Fresh AWS Learner Lab}
Your AWS Learner Lab can be reset using the reset button in the toolbar.

\noindent
\includegraphics[width=\textwidth]{images/reset-button.png}

\noindent
To ensure that you are not accidentally depending on anything specific to your Learner Lab environment,
we recommend that you reset your lab prior to final submission.
Note that resetting the lab can take a \textit{considerable} amount of time, on the order of hours.
You should do this at \textit{least} 3 to 4 hours before the submission deadline.
Please do not wait to the last minute.

\paragraph{Deploying with Docker}
In this course, you have been shown how to use Docker containers to deploy on ECS. You may refer to the practical worksheets for a description of how to deploy with containers \cite{prac-week5}.

\subsection{Fine Print}
You can reproduce our process for deploying your application using our
\link{Docker image}{https://ghcr.io/CSSE6400/csse6400-cloud-testing}.
\codefile[language=docker]{Dockerfile}{deployment/Dockerfile}

Our steps for deploying your infrastructure using this container are as follows.
\texttt{\$REPO} is the name of your repository, and
\texttt{\$CREDENTIALS} is the path where we will store your AWS credentials.
\begin{code}[language=shell]{}
$ git clone git@github.com:CSSE6400/$REPO
$ cp $CREDENTIALS $REPO
$ docker run -v /var/run/docker.sock:/var/run/docker.sock -v $(pwd)/$REPO:/workspace csse6400-cloud-testing
$ cat $REPO/api.txt # This will be used for deployment and load testing.
\end{code}

Note that the Docker socket of the host has been mounted. This enables running \texttt{docker} in the container. This has been tested on Mac OSX and Linux but may require WSL2 on Windows.


\section{Criteria}
Your assignment submission will be assessed on its ability to support the specified use cases.
Testing is divided into functionality, deployment and scalability testing,
corresponding to the three submission stages of the assignment.
Functionality testing is to ensure that your backend software and API
meet the MVP requirements by satisfying the API specification without any excessive load.
Deployment is to ensure that this MVP can then be hosted in the target cloud provider.
Scalability testing is based upon several likely usage scenarios.
The scenarios create different scaling requirements.

\subsection{API Functionality} % Pesistence: Does not need to be "truly" persistent.
10\%, from the 35\% of the grade for the assignment, is for correctly implementing the API specification,
irrespective of whether it is able to cope with high loads.
A suite of automated API tests will assess the correctness of your implementation, via a sequence of API calls.

\subsection{Deployed to Cloud} % Persistence working correctly in the cloud.
10\%, from the 35\% of the grade for the assignment, is for deploying a correctly implemented service to AWS,
irrespective of whether it is able to cope with high loads.
The deployment will be assessed by running a script that deploys your service to AWS
and then runs a suite of automated API tests to assess the correctness of your implementation.

\subsection{Scalable Application}\label{sec:scenarios} % Can it scale!
15\%, from the 35\% of the grade for the assignment, will be derived from how well your service handles various scenarios.
These scenarios will require you to consider how your application performs under different load characteristics.
Examples of possible scenarios are described below.
These are not descriptions of specific tests that will be run,
rather they are examples of the types of tests that will be run.

\paragraph{Normal Circumstances}
Average number of analysis requests are submitted by a wide range of labs.
These requests are distributed fairly evenly over the working day.
Few of the analysis requests are urgent.
Queries for results are also distributed fairly evenly over the day.

\paragraph{Curiosity Killed the Server}
Queensland Health creates a web portal allowing patients to view their analysis results.
Tens of thousands of people all try to access their results in a short period of time.

\paragraph{Epidemic Early Stages}
There is a significant increase in the number of analysis request submitted by a few labs.
These labs also mark a much higher number of the requests as being urgent.
Other labs are operating as per normal circumstances.
There are many more queries for results, often being repeated for any pending analysis jobs.
Health authorities make many requests for batches of results from labs or for patients.

\paragraph{Epidemic Mid Stages}
There are periods of time where several labs submit a large volume of analysis requests.
There are other periods of time with reduced numbers of analysis requests.
Up to 15\% of the requests may be given an urgent status.
Queries for results follow the same fluctuating pattern as analysis requests.
Health authorities make a moderate number of requests for batches of results from labs or for patients.

\paragraph{Epidemic Peak}
Almost all labs submit a very large volume of analysis requests.
Up to 30\% of the requests may be given an urgent status.
The service must still analyse non-urgent requests in a timely manner.
If it does not, the risk is that labs will start to make all requests urgent.
Due to the high volume of analysis requests, there will be a high volume of queries for results.
Any results that are still pending analysis jobs, will be repeated.
Health authorities make a moderate number of requests for batches of results from almost all labs.
Labs start querying batches of results for individual patients.

\paragraph{Epidemic Tail}
Almost all labs submit a large volume of analysis requests.
Up to 10\% of the requests may be given an urgent status.
Due to the high volume of analysis requests, there will be a high volume of queries for results.
Most results that are still pending analysis jobs, will be repeated.
Health authorities make a moderate number of requests for batches of results from almost all labs.
Labs query batches of results for a moderate number of patients.

\paragraph{Epidemic Follow-up}
Some labs submit a large volume of analysis requests.
Up to 5\% of the requests may be given an urgent status.
Other labs are back to operating as per normal circumstances.
There will be a moderate volume of queries for results.
Most results that are still pending analysis jobs, will be repeated.
Health authorities make a large number of requests for batches of results from almost all labs.
Labs query batches of results for a large number of patients.

\paragraph{Research Project}
A researcher wants to analyse the progression of how a disease spreads in the community.
They will query all the results from all the labs to generate their database of infections in different areas.
The system must remain responsive to analysis requests while processing the researcher's queries.

\subsection{Marking}
Persistence is a core functional requirement of your service.
If your implementation does not save analysis requests, with their associated images,
to persistent storage, your grade for the assignment will be capped at 4.

Your persistence mechanism must be robust, so that it can cope with catastrophic failure of your service.
If all running instances of your services are terminated,
the system must be able to restart and guarantee that it has not lost \textit{any} data
about analysis requests for which it returned a success response to the client.
There will not be a test that explicitly kills all services and restarts the system.
This will be assessed based on the services you use and how your implementation invokes those services.
Not saving data to a persistent data store, or returning a success response before the data has been saved,
are the criteria that determine whether you have successfully implemented persistence.

Functionality of your service is worth \textbf{10\%} from the 35\% of the grade for the assignment.
This is based on successful implementation of the given API specification
and the ability to use the provided tool in your implementation.

Deploying your service is worth \textbf{10\%} from the 35\% of the grade for the assignment.
This is based on the successful deployment, using Terraform,
of your service to AWS and the ability to access the service via the API.
Your service must be fully functional while deployed,
so the functionality tests can be run which determines the marks for deployment.

Scaling your application to successfully handle the usage scenarios accounts for the remaining \textbf{15\%}
of the grade for the assignment.
The scenarios described in section \ref{sec:scenarios} provide guidance
as to the type of scalability issues your system is expected to handle.
They are not literal descriptions of the exact loads that will be used.
Tests related to scenarios that involve more complex behaviour will have higher weight than other tests.

The scenarios will evaluate whether your service is being wasteful in resource usage.
The amount of resources deployed in your AWS account will be monitored to ensure that
your service implements a scaling up \textbf{\emph{and}} scaling down procedure.

All stages of the assessment will be marked using automation and a subset of the tests will be released.
These tests may consume a \textbf{\emph{significant}} portion of your AWS credit.
You are advised to be prudent in how many times you execute these tests.
The amount of tests to be released is at the Course Coordinator's discretion.

Please refer to the marking criteria at the end of this document.

%\subsubsection{Grade Improvement}\label{sec:improve}
%Improving your application's functionality in a later submission will be used to improve the grade you received for an earlier submission.
%
%\paragraph{Stage 1 Improvement}
%If your stage 1 submission (API Functionality) performs poorly, you may improve your grade for this stage in a later stage.
%This will occur if your deployed stage 2 or 3 submission passes more of the functionality tests than your stage 1 submission.
%This new result will be used to recalculate your grade for stage 1.
%(We will \textbf{not} rerun the local functionality tests in stages 2 or 3. We will \textbf{only} test your deployed application.)
%
%\paragraph{Stage 2 Improvement}
%Similarly, if your deployed stage 3 submission passes more of the functionality tests than your stage 2 submission, this result will be used to recalculate your grade for stage 2.
%
%\paragraph{Late Penalties}
%If an earlier submission was late, the same late penalty will be applied to your improved grade.
%(e.g. If your stage 1 submission was 3 hours late,
%and in stage 2 your submission's functionality improved from a grade of 4 to a grade of 6,
%your grade for stage 1 would be increased to a 5 after the one grade point per 24 hours penalty was applied.)


\section{Academic Integrity}
As this is a higher-level course, you are expected to be familiar with the importance of academic integrity in general,
and the details of UQ's rules.
If you need a reminder, review the \link{Academic Integrity Modules}
{https://web.library.uq.edu.au/library-services/it/learnuq-blackboard-help/academic-integrity-modules}.
Submissions will be checked to ensure that the work submitted is not plagiarised or of no academic merit.

This is an \textit{individual} assignment.
You may \textbf{not} discuss details of approaches to solve the problem with other students in the course.
As some students may have an extension and then decide to make a late submission, 
you may \textbf{not} discuss details of a submission stage with other students
until \textbf{two weeks} \textit{after} the original submission date.
i.e. You may not discuss how to
\begin{itemize}
    \item implement \textbf{functionality} until \emph{after} \textbf{April 18$^{th}$};
    \item \textbf{deploy} the service to the cloud until \emph{after} \textbf{May 1$^{st}$}; and
    \item implement \textbf{scalability} until \emph{after} \textbf{May 23$^{rd}$}.
\end{itemize}
%The Grade Improvement rule (section \ref{sec:improve}) means that
%you may \textbf{not} discuss details of any earlier submission stage with other students
%until \textbf{two weeks} \textit{after} the final submission date of May 9$^{th}$ (i.e. \emph{after} \textbf{May 23$^{rd}$}).

All code that you submit \textbf{must} be your own work.
You may \textbf{not} directly copy code that you have found on-line to solve parts of the assignment.
If you find ideas from on-line sources (e.g. Stack Overflow),
you must \link{cite and reference}{https://web.library.uq.edu.au/node/4221/2} these sources.
Use the \link{IEEE referencing style}{https://libraryguides.vu.edu.au/ieeereferencing/gettingstarted}
for citations and references.
Citations should be included in a comment at the location where the idea is used in your code.
All references for citations \textbf{must} be included in a file called \texttt{refs.txt}.
This file must be in the root directory of your repository.

You may use generative AI tools (e.g. Copilot) to assist you in writing code to implement your solutions.
You may also use generative AI tools to help you test your implementations.
You \textbf{must} include, in the root directory of your repository,
a file called \texttt{AI.md} that indicates the generative AI tools that you used,
how you used them, and the extent of their use.
(e.g. All code was written by providing copilot with class descriptions and then revising the generated code.
Classes A and B were produced by the following prompts to ChatGPT
and were then adapted to work in the assignment's context.)

Uncited or unreferenced material, or unacknowledged use of generative AI tools,
will be treated as not being your own work.
Significant amounts of cited or acknowledged work from other sources
will be considered to be of no academic merit.


\bibliographystyle{ieeetr}
\bibliography{ours}


\clearpage
\begin{landscape}

\section*{Marking Criteria: Common Part~(20\%)}

All team members will be awarded the same result for the Title Slide, Introduction and Context
(by \textbf{Presenter \#1}), ASRs (by \textbf{Presenter \#2}), and Conclusion (by \textbf{Presenter \#6}).

\fontsize{9}{11}\selectfont

\begin{xltabular}{\linewidth}{| P{1.8cm} | X | X | X | X | X | X | X |}
\hline
\multicolumn{1}{|c}{\multirow{2}{*}{\textbf{Criteria}}} &
  \multicolumn{7}{c|}{\textbf{Standard}} \\ \cline{2-8}
\multicolumn{1}{|c}{} &
  \multicolumn{1}{c|}{\textbf{Exceptional ~ (7)}} &
  \multicolumn{1}{c|}{\textbf{Advanced ~ (6)}} &
  \multicolumn{1}{c|}{\textbf{Proficient ~ (5)}} &
  \multicolumn{1}{c|}{\textbf{Functional ~ (4)}} &
  \multicolumn{1}{c|}{\textbf{Developing ~ (3)}} &
  \multicolumn{1}{c|}{\textbf{Little Evidence ~ (2)}} &
  \multicolumn{1}{c|}{\textbf{No Evidence ~ (1)}} \\ \hline
\endhead
%
\textbf{~Context\newline 5\%} &
Project is introduced clearly and well situated within its context, providing an excellent starting point to understand the system. &
Project is introduced clearly with good~con\-textual information, providing a good starting point to understand the system. &
Project is introduced well with a good over\-view of its context, providing a clear but basic overview of the system. &
Project is introduced fairly well with some contextual informa\-tion, providing a com\-prehensible over\-view of the system. &
Project scope \& general context are fairly clear, providing a general overview of the system. &
Project scope \& context are not clear, providing a poor overview of the system. &
Project scope \& context are confusing, providing an inaccurate overview of the system. \\
\hline

\textbf{~ ~ASRs\newline 10\%} &
ASRs are clearly described, well justified, clearly of high importance, and all will influence architecture decisions. &
ASRs are clearly described, fairly well jus\-tified, seemingly of high importance, and all are likely to influ\-ence architecture decisions. &
Most ASRs are well described but a few justifications are a little weak. Most are important and likely to influence architecture decisions. &
Some ASRs are well described but a few justifications are weak. Most are important and likely to influence architecture decisions. &
Some ASRs are fairly well described but some justifications~are weak. Some are important and likely to influence architecture decisions. &
Most ASRs are poorly described or poorly justified. Few are im\-portant or likely to influence architecture decisions. &
Most ASRs are poorly described and poorly justified. Very few are important or likely to influence architecture decisions. \\
\hline

\textbf{~ ~Conclusion\newline 5\%} &
Conclusion provides a clear, well-structured summary of all key architectural points and offers insightful reflection on lessons learnt. &
Conclusion clearly summarises~most~key architectural points, and includes thoughtful reflection. &
Conclusion summarises main points clearly and includes some useful reflection. &
Conclusion presents a reasonable summary, though some points may be underdeveloped. &
Conclusion attempts to summarise key points but is vague or superficial. &
Conclusion is unclear or disorganised, with poor summarisation. &
Conclusion is confusing or missing. \\
\hline
\end{xltabular}

\clearpage


\section*{Marking Criteria: Individual Part~(80\%)}

\subsection*{Presentation \#1 Title Slide, Introduction and Context, Architecture}

\fontsize{9}{11}\selectfont

\begin{xltabular}{\linewidth}{| P{1.8cm} | X | X | X | X | X | X | X |}
\hline
\multicolumn{1}{|c}{\multirow{2}{*}{\textbf{Criteria}}} &
  \multicolumn{7}{c|}{\textbf{Standard}} \\ \cline{2-8}
\multicolumn{1}{|c}{} &
  \multicolumn{1}{c|}{\textbf{Exceptional ~ (7)}} &
  \multicolumn{1}{c|}{\textbf{Advanced ~ (6)}} &
  \multicolumn{1}{c|}{\textbf{Proficient ~ (5)}} &
  \multicolumn{1}{c|}{\textbf{Functional ~ (4)}} &
  \multicolumn{1}{c|}{\textbf{Developing ~ (3)}} &
  \multicolumn{1}{c|}{\textbf{Little Evidence ~ (2)}} &
  \multicolumn{1}{c|}{\textbf{No Evidence ~ (1)}} \\ \hline
\endhead
%

\textbf{Architecture\newline Completeness\newline25\%} &
Description is clear, complete, concise, and informative, resulting in an excellent and coherent understanding of the overall architecture and its major components. &
Description is clear, almost complete, and informative, resulting in a good and coherent understanding of the system's architecture and structure.	&
Description is mostly clear and informative, resulting in a good understanding of the system's architectural structure. &
Description is mostly clear and informative, though some architectural elements may be missing or underexplained. &
At times the description lacks clarity, leading to a vague or partial overview of the system's architecture. &
Description is unclear or incomplete, omitting important architectural elements or structure, leading to a poor understanding of the architecture. &
Description is confusing, severely incomplete, resulting in an incorrect or misleading understanding of the architecture. \\
\hline

\textbf{Architecture Clarity and\newline Consistency\newline20\%} &
Architectural structure is communicated with excellent clarity, logical flow, consistency and at an appropriate level of abstraction.  Relationships and res- ponsibilities between components~are~well explained and coherent. &
Structure is clearly presented and mostly consistent. Component responsibilities and relationships are explained well. &
Architecture is mostly clear and consistent, though some relationships or responsibilities may be weakly described.	Description is understandable but may lack cohesion, with minor inconsistencies or unclear relationships. &
Description is understandable~but~may lack cohesion, with minor inconsistencies or unclear relationships. &
Architectural explanation is somewhat disorganised or inconsistent, weakening the overall coherence. &
Explanation is unclear or inconsistent, making it difficult to follow architectural relationships. &
Explanation is highly inconsistent or incoherent, obscuring the system's architecture entirely. \\
\hline

\textbf{Design\newline ~~Diagrams\newline25\%} &
All diagrams are easy to comprehend, convey important information, and enhance the presentation. &
Most diagrams are easy to comprehend, convey important~in\-formation, and are used well in the presentation. &
Most diagrams are comprehensible, convey useful information, and are used well in the presentation. &
Most diagrams are comprehensible, convey useful information, and are connected to the presentation. &
Most diagrams are comprehensible, convey some useful information, and are mostly connected to the presentation. &
Some diagrams are incomprehensible, do not convey useful information, or are disconnected from the presentation. &
Most diagrams are incomprehensible, do not convey useful information, or are disconnected from the presentation. \\
\hline

\textbf{Presentation\newline 10\%} &
Presentation is well paced and delivered fluently. Information is logically sequenced, with clear objectives making it very easy to follow. &
Presentation is well paced and delivered clearly. Information is logically sequenced, with some clear objectives making it easy to follow. &
Presentation is mostly well paced and~de\-livered clearly. Information is logically sequenced, with signposting guiding audience through presentation. &
Presentation pace~is a little inconsistent or delivery is occasionally unclear. Information is logically sequenced allowing audience to follow presentation fairly well. &
Presentation pace~is inconsistent or delivery is sometimes unclear. Information is not always logically sequenced, distracting audience from presentation flow. &
Presentation pace~is inconsistent or delivery is unclear. Infor- mation is not logically sequenced, and planned progression was not clear to audience. &
Presentation pace~is inconsistent and~delivery is unclear. Infor- mation is poorly sequenced, confusing audience. \\
\hline

\end{xltabular}

\clearpage

\subsection*{Presentation \#2 Critique}

\fontsize{9}{11}\selectfont

\begin{xltabular}{\linewidth}{| P{1.8cm} | X | X | X | X | X | X | X |}
\hline
\multicolumn{1}{|c}{\multirow{2}{*}{\textbf{Criteria}}} &
  \multicolumn{7}{c|}{\textbf{Standard}} \\ \cline{2-8}
\multicolumn{1}{|c}{} &
  \multicolumn{1}{c|}{\textbf{Exceptional ~ (7)}} &
  \multicolumn{1}{c|}{\textbf{Advanced ~ (6)}} &
  \multicolumn{1}{c|}{\textbf{Proficient ~ (5)}} &
  \multicolumn{1}{c|}{\textbf{Functional ~ (4)}} &
  \multicolumn{1}{c|}{\textbf{Developing ~ (3)}} &
  \multicolumn{1}{c|}{\textbf{Little Evidence ~ (2)}} &
  \multicolumn{1}{c|}{\textbf{No Evidence ~ (1)}} \\ \hline
\endhead
%

\textbf{Depth\newline30\%} &
Provides a thorough, critical analysis of the architecture, addressing key strengths, weaknesses, and how well it meets the ASRs and quality attributes. The critique is insightful, balanced, and well-supported by evidence.&
Provides a comprehensive critique with clear analysis of the architecture's strengths, weaknesses, and how it addresses ASRs and quality attributes. Some evidence supports the critique. &
Critique is generally well-developed, covering major strengths and weaknesses, though it may lack some depth or specific evidence.&
Critique is adequate but lacks depth, with only superficial analysis of strengths, weaknesses, and ASRs. &
Critique is somewhat vague, with limited analysis of the architecture's strengths and weaknesses. &
Critique lacks meaningful analysis or focuses only on minor or irrelevant points. &
No meaningful critique is provided, or it fails to identify any strengths or weaknesses of the architecture. \\
\hline

\textbf{Relevance\newline25\%} &
Critique is closely aligned with the ASRs and quality attributes, offering a clear and detailed explanation of how well the architecture meets them. &
Critique is mostly aligned with ASRs and quality attributes, discussing their impact on the architecture effectively. &
Critique references ASRs and quality attributes, but the connection is not always clear or well-supported. &
Critique mentions ASRs and quality attributes, but the connection to the architecture is weak or unclear. &
Critique makes limited or superficial reference to ASRs or quality attributes. &
Critique mentions ASRs and quality attributes but fails to connect them to the architecture. &
Critique is entirely disconnected from the ASRs and quality attributes.\\
\hline

\textbf{Balanced Evaluation\newline15\%} &
Provides a well-balanced critique, discussing both strengths and weaknesses in a fair, objective, and constructive manner. &
Provides a fairly balanced critique, discussing both strengths and weaknesses, but may focus slightly more on one side.	&
Critique discusses strengths and weaknesses, but the evaluation may be unbalanced, focusing more on one aspect than the other. &
Critique covers strengths and weaknesses, but may not be sufficiently balanced or may favor one aspect too much. &
Critique lacks balance, focusing more on weaknesses or strengths, without giving adequate attention to the other side. &
Critique is unbalanced, only discussing strengths or weaknesses in detail with little consideration of the other side. &
Critique is entirely one-sided or overly negative without recognizing any positive aspects of the architecture. \\
\hline

\textbf{Presentation\newline 10\%} &
Presentation is well paced and delivered fluently. Information is logically sequenced, with clear objectives making it very easy to follow. &
Presentation is well paced and delivered clearly. Information is logically sequenced, with some clear objectives making it easy to follow. &
Presentation is mostly well paced and~de\-livered clearly. Information is logically sequenced, with signposting guiding audience through presentation. &
Presentation pace~is a little inconsistent or delivery is occasionally unclear. Information is logically sequenced allowing audience to follow presentation fairly well. &
Presentation pace~is inconsistent or delivery is sometimes unclear. Information is not always logically sequenced, distracting audience from presentation flow. &
Presentation pace~is inconsistent or delivery is unclear. Infor- mation is not logically sequenced, and planned progression was not clear to audience. &
Presentation pace~is inconsistent and~delivery is unclear. Infor- mation is poorly sequenced, confusing audience. \hline

\end{xltabular}

\clearpage

\subsection*{Presentation \#3 Detailed Design}

\fontsize{9}{11}\selectfont

\begin{xltabular}{\linewidth}{| P{1.8cm} | X | X | X | X | X | X | X |}
\hline
\multicolumn{1}{|c}{\multirow{2}{*}{\textbf{Criteria}}} &
  \multicolumn{7}{c|}{\textbf{Standard}} \\ \cline{2-8}
\multicolumn{1}{|c}{} &
  \multicolumn{1}{c|}{\textbf{Exceptional ~ (7)}} &
  \multicolumn{1}{c|}{\textbf{Advanced ~ (6)}} &
  \multicolumn{1}{c|}{\textbf{Proficient ~ (5)}} &
  \multicolumn{1}{c|}{\textbf{Functional ~ (4)}} &
  \multicolumn{1}{c|}{\textbf{Developing ~ (3)}} &
  \multicolumn{1}{c|}{\textbf{Little Evidence ~ (2)}} &
  \multicolumn{1}{c|}{\textbf{No Evidence ~ (1)}} \\ \hline
\endhead
%

\textbf{Selection of Design Focus\newline15\%} &
An important and significant part of the system was selected, showing excellent judgement. &
A relevant and fairly significant part of the system was selected, which reflects key design complexity or importance. &
A reasonable part of the system was selected to present in detail.	&
Design focus is acceptable, but may not show the most relevant or complex aspect of the system. &
Design focus is only partially appropriate.	&
Focus is weak or only marginally relevant to understanding the detailed design.	&
Design focus is inappropriate, trivial, or disconnected from the system. \\
\hline

\textbf{Design Clarity and\newline Completeness\newline30\%} &
Detailed design is presented clearly and comprehensively, with excellent coverage of component responsibilities and interactions. &
Detailed design is mostly clear and complete, effectively showing how components interact and function.	&
Design is generally clear, with most responsibilities and flows explained; minor gaps may exist. &
Design is presented in an understandable way, though some areas are underdeveloped or unclear.	&
Design presentation lacks detail or clarity in key parts, limiting understanding.	&
Design is hard to follow or significantly incomplete. &
Design is confusing, vague, or missing critical information. \\
\hline

\textbf{Design\newline ~~Diagrams\newline25\%} &
All diagrams are easy to comprehend, convey important information, and enhance the presentation. &
Most diagrams are easy to comprehend, convey important~in\-formation, and are used well in the presentation. &
Most diagrams are comprehensible, convey useful information, and are used well in the presentation. &
Most diagrams are comprehensible, convey useful information, and are connected to the presentation. &
Most diagrams are comprehensible, convey some useful information, and are mostly connected to the presentation. &
Some diagrams are incomprehensible, do not convey useful information, or are disconnected from the presentation. &
Most diagrams are incomprehensible, do not convey useful information, or are disconnected from the presentation. \\
\hline

\textbf{Presentation\newline 10\%} &
Presentation is well paced and delivered fluently. Information is logically sequenced, with clear objectives making it very easy to follow. &
Presentation is well paced and delivered clearly. Information is logically sequenced, with some clear objectives making it easy to follow. &
Presentation is mostly well paced and~de\-livered clearly. Information is logically sequenced, with signposting guiding audience through presentation. &
Presentation pace~is a little inconsistent or delivery is occasionally unclear. Information is logically sequenced allowing audience to follow presentation fairly well. &
Presentation pace~is inconsistent or delivery is sometimes unclear. Information is not always logically sequenced, distracting audience from presentation flow. &
Presentation pace~is inconsistent or delivery is unclear. Infor- mation is not logically sequenced, and planned progression was not clear to audience. &
Presentation pace~is inconsistent and~delivery is unclear. Infor- mation is poorly sequenced, confusing audience. \hline

\end{xltabular}


\clearpage

\subsection*{Presentations \#4 and \#5 Comparison}

\fontsize{9}{11}\selectfont

\begin{xltabular}{\linewidth}{| P{1.8cm} | X | X | X | X | X | X | X |}
\hline
\multicolumn{1}{|c}{\multirow{2}{*}{\textbf{Criteria}}} &
  \multicolumn{7}{c|}{\textbf{Standard}} \\ \cline{2-8}
\multicolumn{1}{|c}{} &
  \multicolumn{1}{c|}{\textbf{Exceptional ~ (7)}} &
  \multicolumn{1}{c|}{\textbf{Advanced ~ (6)}} &
  \multicolumn{1}{c|}{\textbf{Proficient ~ (5)}} &
  \multicolumn{1}{c|}{\textbf{Functional ~ (4)}} &
  \multicolumn{1}{c|}{\textbf{Developing ~ (3)}} &
  \multicolumn{1}{c|}{\textbf{Little Evidence ~ (2)}} &
  \multicolumn{1}{c|}{\textbf{No Evidence ~ (1)}} \\ \hline
\endhead
%

\textbf{Alternative Selection\newline15\%} &
Clearly identifies a relevant and credible alternative architecture, with strong justification for its suitability for the project. &
Identifies a relevant alternative architecture with a good, but not thorough, justification. &
Identifies a plausible alternative architecture but justification of its suitability is a little weak.	&
Identifies a plausible alternative architecture but with minimal justification or clarity of why it's viable. &
Identifies an alternative, but the choice may be weak or poorly explained. &
Identifies an alternative that is irrelevant or unclear. &
Does not identify any meaningful alternative architecture or design philosophy. \\
\hline

\textbf{Comparison\newline30\%} &
Provides a highly detailed and insightful comparison of the chosen architecture and alternative, covering key dimensions. Clearly explains which architecture is more suitable and why. &
Provides an informative and clear comparison, covering key dimensions, with good reasoning behind the preference for the chosen architecture. &
Provides a good comparison, touching on the main aspects, though the explanation may lack depth or full clarity in some areas. &
Comparison addresses key aspects, but it lacks depth in areas such as complexity or the impact on ASRs.	&
Comparison is basic and lacks clarity. &
Provides only superficial comparisons, missing key aspects of the architectures or failing to explain their impact on the system. &
Comparison is poorly developed or nonexistent, providing minimal or no insights into how the two architectures compare. \\
\hline

\textbf{Trade-off Analysis\newline25\%} &
Provides a thorough analysis of the trade-offs involved in choosing the alternative, detailing both its strengths and weaknesses, and how these trade-offs might impact the overall system.	&
Provides a strong analysis of trade-offs, with a clear explanation of how the alternative would affect the system's quality attributes.	&
Identifies major trade- offs but lacks a detailed explanation of how they would impact the project's quality attributes.	&
Provides a basic analysis of trade-offs, but lacks depth in understanding their potential impact on the project. &
Mentions trade-offs but provides limited insight into their impact on the overall system, or the trade-offs are unclear. &
Provides minimal analysis of trade-offs, with little connection to system goals or project needs. &
No trade-off analysis is provided, or it is wholly inadequate or unsubstantiated. \\
\hline

\textbf{Presentation\newline 10\%} &
Presentation is well paced and delivered fluently. Information is logically sequenced, with clear objectives making it very easy to follow. &
Presentation is well paced and delivered clearly. Information is logically sequenced, with some clear objectives making it easy to follow. &
Presentation is mostly well paced and~de\-livered clearly. Information is logically sequenced, with signposting guiding audience through presentation. &
Presentation pace~is a little inconsistent or delivery is occasionally unclear. Information is logically sequenced allowing audience to follow presentation fairly well. &
Presentation pace~is inconsistent or delivery is sometimes unclear. Information is not always logically sequenced, distracting audience from presentation flow. &
Presentation pace~is inconsistent or delivery is unclear. Infor- mation is not logically sequenced, and planned progression was not clear to audience. &
Presentation pace~is inconsistent and~delivery is unclear. Infor- mation is poorly sequenced, confusing audience. \hline

\end{xltabular}

\clearpage

\subsection*{Presentation \#6 Security}

\fontsize{9}{11}\selectfont

\begin{xltabular}{\linewidth}{| P{1.8cm} | X | X | X | X | X | X | X |}
\hline
\multicolumn{1}{|c}{\multirow{2}{*}{\textbf{Criteria}}} &
  \multicolumn{7}{c|}{\textbf{Standard}} \\ \cline{2-8}
\multicolumn{1}{|c}{} &
  \multicolumn{1}{c|}{\textbf{Exceptional ~ (7)}} &
  \multicolumn{1}{c|}{\textbf{Advanced ~ (6)}} &
  \multicolumn{1}{c|}{\textbf{Proficient ~ (5)}} &
  \multicolumn{1}{c|}{\textbf{Functional ~ (4)}} &
  \multicolumn{1}{c|}{\textbf{Developing ~ (3)}} &
  \multicolumn{1}{c|}{\textbf{Little Evidence ~ (2)}} &
  \multicolumn{1}{c|}{\textbf{No Evidence ~ (1)}} \\ \hline
\endhead
%

\textbf{Security\newline Threats\newline30\%} &
Clearly and comprehensively identifies all security threats specific to the system, with a deep understanding of their potential impact. &
Identifies key security threats and risks, with a good understanding of their potential impact. &
Identifies several security threats, though some may be less relevant or insufficiently detailed. &
Identifies a few key security threats but misses some major ones or provides insufficient detail. &
Identifies only a limited range of security threats, missing major threats that could impact the system. &
Provides an incomplete or unclear iden- tification of security threats, omitting cri- tical issues.	&
Fails to identify or improperly identifies the security threats. \\
\hline

\textbf{Mitigations\newline30\%} &
Thoroughly explains the security mechanisms used to mitigate the identified threats, and how they are integrated into the architecture. Mechanisms are clearly linked to specific threats.	&
Explains the security mechanisms effectively, and links them to the identified threats and risks, with minor gaps in explanation. &
Provides a good explanation of security mechanisms, but some parts lack clarity or details of how they address specific threats. &
Provides an explanation of security mechanisms, but with vague or incomplete descriptions of how they mitigate the risks.	&
Provides a minimal explanation of security mechanisms, leaving out key details or failing to fully connect them to identified threats.	&
The explanation of security mechanisms is unclear or disconnected from the identified threats, with many important aspects missing.	&
Does not explain the security mechanisms or fails to show how they address the identified threats. \\
\hline

\textbf{Remaining Challenges\newline10\%} &
Thoroughly identifies any remaining security challenges or risks in the architecture and suggests thoughtful, feasible improvements. &
Identifies remaining security challenges with a good explanation of potential future improvements and strategies to address them. &
Acknowledges some remaining security challenges but does not offer concrete or comprehensive strategies for improvement. &
Identifies some challenges but does not offer specific or actionable recommendations for future improvements. &
Mentions remaining security issues, but provides no or very weak suggestions for improvement. &
Superficial identification of remaining security challenges or improvement oppor- tunities. &
Fails to identify remaining security chal- lenges or improvement opportunities, or completely overlooks the topic. \\
\hline

\textbf{Presentation\newline 10\%} &
Presentation is well paced and delivered fluently. Information is logically sequenced, with clear objectives making it very easy to follow. &
Presentation is well paced and delivered clearly. Information is logically sequenced, with some clear objectives making it easy to follow. &
Presentation is mostly well paced and~de\-livered clearly. Information is logically sequenced, with signposting guiding audience through presentation. &
Presentation pace~is a little inconsistent or delivery is occasionally unclear. Information is logically sequenced allowing audience to follow presentation fairly well. &
Presentation pace~is inconsistent or delivery is sometimes unclear. Information is not always logically sequenced, distracting audience from presentation flow. &
Presentation pace~is inconsistent or delivery is unclear. Infor- mation is not logically sequenced, and planned progression was not clear to audience. &
Presentation pace~is inconsistent and~delivery is unclear. Infor- mation is poorly sequenced, confusing audience. \hline

\end{xltabular}

\end{landscape}


\end{document}


