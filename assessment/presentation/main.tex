\documentclass{csse4400}

\usepackage{CJKutf8}  % Support Chinese characters.
\usepackage{fancyhdr}

% RUBRIC
\usepackage{multirow}
\usepackage{array}
\usepackage{xltabular}
\usepackage{pdflscape}
\usepackage{enumitem}
\usepackage{hyperref}

\newcolumntype{P}[1]{>{\centering\arraybackslash}p{#1}}
% RUBRIC

\title{Architecture Presentation}
\author{Richard Thomas \& Guangdong Bai}
\date{Semester 1, 2025}

\begin{document}

\input{copyright-footer}
\maketitle

\section*{Summary}

In this assessment, you will deliver a presentation around the software architecture of your group's capstone project.
Your task is to explain the design decisions made, critically evaluate your architecture, compare it with viable alternatives, and demonstrate awareness of architectural trade-offs.

This assessment item is designed to showcase your skills to solve problems as a software architect. You should demonstrate:
\begin{enumerate}
    \item a clear understanding of software architecture principles,
    \item the ability to evaluate and defend architectural choices, and
    \item effective communication of complex architectural concepts.
\end{enumerate}


\section{Introduction}

Software architecture forms the foundation of every software system. Whether you are designing a cloud-based service, a data processing platform, or a mobile application, software architecture defines how well the software system meets its goals such as scalability, maintainability, and security.

As part of your capstone project, your team has collaboratively developed the architecture of your software system.
In this assessment item, you will take a deep analysis into the architecture.
You will provide not only a description of your architecture, but a critical evaluation of how well your architecture meets its intended goals.

Each team is allocated a time slot for your presentation.
During the presentation of your team, each member will focus on a specific architectural concern, such as architectural design, detailed design, design rationale and trade-offs, alternative patterns, and security, and then will be \textbf{assessed individually}.


\section{Presentation}

\subsection{Presentation Scheduling}

\begin{itemize}
  \item Each team will be allocated \textbf{35 minutes} during \textbf{Week 13} to deliver their presentations. This time includes setup, individual presentations, and a brief Q\&A/discussion.
  \item The teaching team will provide a list of available time slots, which will primarily take place during the scheduled lecture hours, case study sessions and practical sessions of Week 13.
  \item Your team must \textbf{collaboratively choose a single time slot} that all team members are available for. \textbf{All team members are required to be present during the allocated session}. This includes being available to present your assigned topic individually, being present for your teammates' presentations, and being available for answering questions regarding your project.
  \item Please inform the course coordinators of any constraints you may have regarding presentation time \textbf{\textit{before}} Easter.
\end{itemize}






\subsection{Presentation Content}

Every team member is free to structure your presentation however you wish, though the team together needs to cover the following content.

\begin{description}
    \item[Title Slide] Name of the software project, and names and student numbers of team members in the order of presentation.
    \item[Introduction and Context] Describe the software project, explaining its key functionality and target users, and provide an overview of the software system's context and its external dependencies.
    \item[Architecture] Describe the software's architecture, and the Architecturally Significant Requirements (ASR) of most importance to the project.
    \item[Detailed Design] Describe the internals of key components or subsystems.
    \item[Critique] Analyse the software's architecture, describing how well it delivers its ASRs.
    \item[Comparison] Compare your architecture with chosen viable alternatives.
    \item[Security] Describe the security concerns and your mitigation mechanisms in your architecture.
    \item[Conclusion] Highlight the key points of your team's presentation.
\end{description}



\subsection{Presentation Guidelines}


\subsubsection{Presenter \#1 (6.5 mins): Title Slide, Introduction and Context, Architecture}

Your presentation should start with the introduction of your team's capstone project \textbf{within 1.5 mins}.
Give an elevator pitch style summary of what problem the project solves and its key features.


Describe the project's software architecture using appropriate views \cite{view-notes}.
You must use the C4 modelling notation \cite{view-notes} \cite{brown2022c4} to describe the software architecture.
You may supplement the C4 diagrams with other diagrams to help describe the architecture.
For example, you may use UML use case, class, or sequence diagrams \cite{view-notes} \cite{uml}
to describe system requirements or details of how the architectural design works.
Other diagrams may also be used, if they clarify aspects of your C4 model.
Any diagrams obtained from other sources  must be cited.

Your description of the software architecture should cover all of its important aspects.
You are not expected to get down to the level of describing the detailed design of the software, which will be done by \textbf{Presenter \#2}, nor the design trade-offs, which will be done by \textbf{Presenter \#3}.
You should not need to provide class or dynamic diagrams for the entire system.

Your audience is other students in this course. You may assume the audience has knowledge of the course content,
though you should not assume they are familiar with the project you are describing.


\subsubsection{Presenter \#2 (5 mins): Critique}


Your presentation should deliver a \textbf{critical evaluation of your software architecture}, focusing on how well it addresses the ASRs and supports the overall project goals.

Begin by describing which ASRs and, in particular, the quality attributes you think are most important for the project, and why.
Then, assess how well the architecture you've designed satisfies those attributes.
Your presentation is advised to be specific to particular design choices or structures, and to discuss how they contribute to or potentially hinder your quality goals.

You should also highlight any limitations, trade-offs, or compromises that were made.


\subsubsection{Presenter \#3 (5 mins): Detailed Design}

Your presentation should focus on the \textbf{detailed design} of \textbf{one or two key parts} of your capstone project. The goal is to show how architectural decisions are designed at a lower level, and how key components work together to fulfill the system's ASRs and quality attributes.

Select only \textbf{one or two} significant components, services, or subsystems. Your selected one(s) should central to the system's functionality. Describe their internal structure, key interfaces, and important interactions with other parts of the system.

You should use appropriate UML diagrams to support your explanation. Depending on what you are describing, these may include:

\begin{itemize}
    \item \emph{Class diagrams} to show internal structure and relationships
    \item \emph{Sequence diagrams} to show interactions between components
    \item \emph{State diagrams} if the component involves complex state transitions
    \item \emph{Activity diagrams} for workflow or process modeling
\end{itemize}

Explain any design patterns, principles, or trade-offs applied in your detailed design. For example, if you've used an adapter, explain why, and how it contributes to quality attributes.

You are not expected to describe the full system in detail. Focus on the parts that are most important or interesting from an architectural perspective.

Assume your audience is not familiar with the inner workings of your project. You are expected to clearly communicate how your detailed design brings the architecture to life.


\subsubsection{Presenters \#4 and \#5 (each 5 mins): Comparison}

For each presenter, your task is to compare your team's chosen software architecture with \textbf{one viable alternative}.

Your presentation is advised to begin by clearly identifying the alternative architecture. This could be an alternative style (e.g., microservices v.s. monolithic), or even a fundamentally different design philosophy.

Explain what makes this alternative a \textbf{credible} option for your project. What trade-offs would it involve? What problems might it solve better? What new challenges would it introduce?
Then, compare your chosen architecture and the alternative along key dimensions such as: support for ASRs, complexity, and team expertise.

You can use diagrams, tables, or summaries where appropriate to make your comparison clear and visual.


\subsubsection{Presenter \#6 (5.5 mins): Security, Conclusion}

Your presentation should focus on the security aspects of your software architecture. Discuss the key security concerns specific to your project and how the architecture is designed to address them.

Start by identifying the primary security threats or risks your system faces, such as unauthorized access, data breaches, and Denial of Service (DoS).

Explain the security mechanisms built into your architecture to mitigate these threats, such as authentication and authorization strategies, and data protection measures.
You are advised to use security design patterns (e.g., secure-by-design, defense-in-depth).

Highlight any remaining security challenges or areas where future improvements could be made.



\subsection{Citations \& References}
You may use references in your presentation to support points you are making.
These must be cited and referenced using the \link{IEEE referencing style}
{https://libraryguides.vu.edu.au/ieeereferencing/gettingstarted}.
The final slide(s) of your presentation should include the references to any cited material.
You should display the reference slide(s) for about 3 seconds at the end of your presentation.
You are not required to speak to the reference slides,
aside from possibly thanking your audience for listening and stating these are your references.


\subsection{Presentation Hints}

If your presentation exceeds designed minutes, the marker will ask you to stop your presentation.
No content of your presentation past that will be marked.

As a presenter, you should not read a script.
You may wish to write a script to prepare for the presentation but should not read it during the presentation.
You may make use of notes during the presentation but you should only quickly glance at your notes to keep yourself on track.
You should not be constantly referring to notes.
You should try to maintain eye contact with your audience, rather than focussing on your notes or slides.

\section{Identity Verification}
The presentation is an identity verified assignment.
You must make your presentation in-person.
At the start of your presentation you must show your UQ student card to one of the markers at your session.
Like in an exam situation, if you have lost your student card
you must obtain a temporary identity verification document from the UQ student centre \emph{before} your presentation.

%The marked result of your presentation will be used to determine any caps applied to your grade.
%(That means failing the presentation because you did not submit the required number of peer evaluations
%will \textbf{\textit{not}} affect the mark used to determine a final grade cap.)
%The first slide of your presentation \textbf{must} contain your full name, as recorded in UQ's student enrolment system,
%and full 8-digit student number.

%\subsection{On-line Identity Verification}
%If you are are an external student it does not matter if your UQ student card has expired.
%If you do not have a UQ student card, you may use an official government photo id that shows your full name.
%Your id must be clearly visible for at least 3 seconds.
%If a marker cannot view your card clearly enough, they will ask you to move it so it is clearly readable.
%
%\begin{CJK*}{UTF8}{gbsn}
%If your government id does not show your name in Roman characters, as recorded in UQ's student enrolment system,
%you need to include a clear image of your government id on your first slide and a textual
%representation of your name that can be selected and copied from your slide so that it may be pasted into a translator.
%(e.g. If you use your China Resident Identity Card, you must provide clear images of the front and back
%of the card. You also need to provide a textual representation of your name in Chinese characters, e.g. 蒙晶.)
%\end{CJK*}
%
%Your face must be visible throughout the presentation to show that you are the one speaking during the presentation.
%This may be through Zoom's participants window.
%If you cannot arrange for your face to be visible throughout the presentation,
%you \textbf{must} contact the course coordinator before 28 April 2023 to discuss your constraints.

%\subsection{On-Campus Identity Verification}
%If you are presenting on-campus, you \textbf{must} show the marker your current and valid UQ student card.
%Like in an exam situation, if you have lost your student card
%you must obtain a temporary identity verification document from the UQ student centre \emph{before} the presentation.


\section{Submission}


There are two components that make up your assessable content for the presentation, i.e., the slides you use for your presentation, and the presentation itself.

%\subsection{Draft Model}
%You must show a tutor a draft architectural model of your selected system in your tutorial in week 6 (March 30).
%The model must include appropriate views that give an overview of the key aspects of the system's software architecture.
%You may need to give the tutor a one minute overview of the project you have selected and its key goals.
%If your provided model is not an appropriate overview of the system
%(e.g. too superficial, missing key parts, or too detailed)
%the grade you achieve for the presentation will be reduced by one grade level.

\subsection{Slides}
The slides for your presentation are to be submitted as a PDF file to a link provided on BlackBoard.
Your slides are due at 11:00am on May 26 2025.
Late submission of your slides will result in a penalty of 1 grade per 24 hour period that they are late.
Regardless of any penalty applied to the presentation, \emph{even} if the penalty is a failing grade,
you \textbf{\textit{must}} still make your presentation in your allocated timeslot.

\subsection{Presentation}
Your presentation is to use the slides you submit to BlackBoard.
If you do not deliver your presentation, your final grade will be capped at a failing grade.
If you are unable to attend your session to give your presentation due to exceptional circumstances,
you may apply to defer your presentation to another date.
You are not able to defer a deferred presentation.
Please find more information in the \link{course profile}{https://course-profiles.uq.edu.au/course-profiles/CSSE6400-21553-7520}.

%\subsection{Peer Evaluation}
%You are expected to attend all presentations.
%You are required to submit an evaluation of each presentation you observe.
%Submission of \emph{meaningful} feedback for at least \textbf{75\%} of the presentations in your class sessions
%is required to obtain a passing grade or higher for the presentation assessment.
%
%An online form will be provided for you to submit your evaluation for each presentation.
%You must submit your evaluation of each presentation separately in order for the system to record all of your evaluations.
%
%If you are unable to attend a practical session due to exceptional circumstances,
%and miss viewing several presentations,
%you may apply for a modified limit on the number of presentations you must evaluate.


\section{Academic Integrity}
As this is a higher-level course, you are expected to be familiar with the importance of academic integrity in general,
and the details of UQ's rules.
If you need a reminder, review the \link{Academic Integrity Modules}
{https://web.library.uq.edu.au/library-services/it/learnuq-blackboard-help/academic-integrity-modules}.
Submissions will be checked to ensure that the work submitted is not plagiarised.
If you have quoted or paraphrased any material from another source, it must be correctly \link{cited and referenced}
{https://guides.library.uq.edu.au/referencing}.
Use the \link{IEEE referencing style}{https://libraryguides.vu.edu.au/ieeereferencing/gettingstarted} for citations and your bibliography.

Note that text generated by an AI tool, such as ChatGPT, is based on text from the Internet.
Consequently all text, whether written on slides or spoken during a presentation,
that was generated by an AI tool must be cited.

Uncited or unreferenced material will be treated as not being your own work.
Extensive quotation or minor rephrasing of material from cited sources should be avoided.
Significant amounts of cited material from other sources, even if paraphrased, will be considered to be of no academic merit.
In all cases, any material that you cite must support the arguments and points that you are making in your presentation.


\bibliographystyle{ieeetr}
\bibliography{ours,books}


%\section*{Draft Model Criteria}
%
%\begin{table}[h]
%\centering
%\footnotesize
%\begin{tabular}{|p{2cm}|p{7.5cm}p{7.5cm}|}
%\hline  & \multicolumn{2}{c|}{\textbf{Standard}}  \\ \hline
%\multicolumn{1}{|c|}{\textbf{Criteria}} & \multicolumn{1}{c|}{\textbf{Acceptable}}  & \multicolumn{1}{c|}{\textbf{Not Sufficient}}  \\ \hline
%\textbf{Context}               & \multicolumn{1}{p{7.5cm}|}{Provides a generally clear overview of the system.}    & System's scope and usage context are not clear.  \\ \hline
%\textbf{ASRs}                  & \multicolumn{1}{p{7.5cm}|}{Identifies seemingly important goals and constraints.} & Important goals and constraints are not clear or not identified.                                                        \\ \hline
%\textbf{Architecture Diagrams} & \multicolumn{1}{p{7.5cm}|}{Provide an overview of the system’s architectural structure. They also demonstrate an initial understanding of parts of the system’s internal design.} & Provides a superficial overview of the architecture structure, or architectural design is lost in system design detail. \\ \hline
%\end{tabular}
%\end{table}


\clearpage
\begin{landscape}

\section*{Marking Criteria: Common Part~(20\%)}

All team members will be awarded the same result for the Title Slide, Introduction and Context
(by \textbf{Presenter \#1}), ASRs (by \textbf{Presenter \#2}), and Conclusion (by \textbf{Presenter \#6}).

\fontsize{9}{11}\selectfont

\begin{xltabular}{\linewidth}{| P{1.8cm} | X | X | X | X | X | X | X |}
\hline
\multicolumn{1}{|c}{\multirow{2}{*}{\textbf{Criteria}}} &
  \multicolumn{7}{c|}{\textbf{Standard}} \\ \cline{2-8}
\multicolumn{1}{|c}{} &
  \multicolumn{1}{c|}{\textbf{Exceptional ~ (7)}} &
  \multicolumn{1}{c|}{\textbf{Advanced ~ (6)}} &
  \multicolumn{1}{c|}{\textbf{Proficient ~ (5)}} &
  \multicolumn{1}{c|}{\textbf{Functional ~ (4)}} &
  \multicolumn{1}{c|}{\textbf{Developing ~ (3)}} &
  \multicolumn{1}{c|}{\textbf{Little Evidence ~ (2)}} &
  \multicolumn{1}{c|}{\textbf{No Evidence ~ (1)}} \\ \hline
\endhead
%
\textbf{~Context\newline 5\%} &
Project is introduced clearly and well situated within its context, providing an excellent starting point to understand the system. &
Project is introduced clearly with good~con\-textual information, providing a good starting point to understand the system. &
Project is introduced well with a good over\-view of its context, providing a clear but basic overview of the system. &
Project is introduced fairly well with some contextual informa\-tion, providing a com\-prehensible over\-view of the system. &
Project scope \& general context are fairly clear, providing a general overview of the system. &
Project scope \& context are not clear, providing a poor overview of the system. &
Project scope \& context are confusing, providing an inaccurate overview of the system. \\
\hline

\textbf{~ ~ASRs\newline 10\%} &
ASRs are clearly described, well justified, clearly of high importance, and all will influence architecture decisions. &
ASRs are clearly described, fairly well jus\-tified, seemingly of high importance, and all are likely to influ\-ence architecture decisions. &
Most ASRs are well described but a few justifications are a little weak. Most are important and likely to influence architecture decisions. &
Some ASRs are well described but a few justifications are weak. Most are important and likely to influence architecture decisions. &
Some ASRs are fairly well described but some justifications~are weak. Some are important and likely to influence architecture decisions. &
Most ASRs are poorly described or poorly justified. Few are im\-portant or likely to influence architecture decisions. &
Most ASRs are poorly described and poorly justified. Very few are important or likely to influence architecture decisions. \\
\hline

\textbf{~ ~Conclusion\newline 5\%} &
Conclusion provides a clear, well-structured summary of all key architectural points and offers insightful reflection on lessons learnt. &
Conclusion clearly summarises~most~key architectural points, and includes thoughtful reflection. &
Conclusion summarises main points clearly and includes some useful reflection. &
Conclusion presents a reasonable summary, though some points may be underdeveloped. &
Conclusion attempts to summarise key points but is vague or superficial. &
Conclusion is unclear or disorganised, with poor summarisation. &
Conclusion is confusing or missing. \\
\hline
\end{xltabular}

\clearpage


\section*{Marking Criteria: Individual Part~(80\%)}

\subsection*{Presentation \#1 Title Slide, Introduction and Context, Architecture}

\fontsize{9}{11}\selectfont

\begin{xltabular}{\linewidth}{| P{1.8cm} | X | X | X | X | X | X | X |}
\hline
\multicolumn{1}{|c}{\multirow{2}{*}{\textbf{Criteria}}} &
  \multicolumn{7}{c|}{\textbf{Standard}} \\ \cline{2-8}
\multicolumn{1}{|c}{} &
  \multicolumn{1}{c|}{\textbf{Exceptional ~ (7)}} &
  \multicolumn{1}{c|}{\textbf{Advanced ~ (6)}} &
  \multicolumn{1}{c|}{\textbf{Proficient ~ (5)}} &
  \multicolumn{1}{c|}{\textbf{Functional ~ (4)}} &
  \multicolumn{1}{c|}{\textbf{Developing ~ (3)}} &
  \multicolumn{1}{c|}{\textbf{Little Evidence ~ (2)}} &
  \multicolumn{1}{c|}{\textbf{No Evidence ~ (1)}} \\ \hline
\endhead
%

\textbf{Architecture\newline Completeness\newline25\%} &
Description is clear, complete, concise, and informative, resulting in an excellent and coherent understanding of the overall architecture and its major components. &
Description is clear, almost complete, and informative, resulting in a good and coherent understanding of the system's architecture and structure.	&
Description is mostly clear and informative, resulting in a good understanding of the system's architectural structure. &
Description is mostly clear and informative, though some architectural elements may be missing or underexplained. &
At times the description lacks clarity, leading to a vague or partial overview of the system's architecture. &
Description is unclear or incomplete, omitting important architectural elements or structure, leading to a poor understanding of the architecture. &
Description is confusing, severely incomplete, resulting in an incorrect or misleading understanding of the architecture. \\
\hline

\textbf{Architecture Clarity and\newline Consistency\newline20\%} &
Architectural structure is communicated with excellent clarity, logical flow, consistency and at an appropriate level of abstraction.  Relationships and res- ponsibilities between components~are~well explained and coherent. &
Structure is clearly presented and mostly consistent. Component responsibilities and relationships are explained well. &
Architecture is mostly clear and consistent, though some relationships or responsibilities may be weakly described.	Description is understandable but may lack cohesion, with minor inconsistencies or unclear relationships. &
Description is understandable~but~may lack cohesion, with minor inconsistencies or unclear relationships. &
Architectural explanation is somewhat disorganised or inconsistent, weakening the overall coherence. &
Explanation is unclear or inconsistent, making it difficult to follow architectural relationships. &
Explanation is highly inconsistent or incoherent, obscuring the system's architecture entirely. \\
\hline

\textbf{Design\newline ~~Diagrams\newline25\%} &
All diagrams are easy to comprehend, convey important information, and enhance the presentation. &
Most diagrams are easy to comprehend, convey important~in\-formation, and are used well in the presentation. &
Most diagrams are comprehensible, convey useful information, and are used well in the presentation. &
Most diagrams are comprehensible, convey useful information, and are connected to the presentation. &
Most diagrams are comprehensible, convey some useful information, and are mostly connected to the presentation. &
Some diagrams are incomprehensible, do not convey useful information, or are disconnected from the presentation. &
Most diagrams are incomprehensible, do not convey useful information, or are disconnected from the presentation. \\
\hline

\textbf{Presentation\newline 10\%} &
Presentation is well paced and delivered fluently. Information is logically sequenced, with clear objectives making it very easy to follow. &
Presentation is well paced and delivered clearly. Information is logically sequenced, with some clear objectives making it easy to follow. &
Presentation is mostly well paced and~de\-livered clearly. Information is logically sequenced, with signposting guiding audience through presentation. &
Presentation pace~is a little inconsistent or delivery is occasionally unclear. Information is logically sequenced allowing audience to follow presentation fairly well. &
Presentation pace~is inconsistent or delivery is sometimes unclear. Information is not always logically sequenced, distracting audience from presentation flow. &
Presentation pace~is inconsistent or delivery is unclear. Infor- mation is not logically sequenced, and planned progression was not clear to audience. &
Presentation pace~is inconsistent and~delivery is unclear. Infor- mation is poorly sequenced, confusing audience. \\
\hline

\end{xltabular}

\clearpage

\subsection*{Presentation \#2 Critique}

\fontsize{9}{11}\selectfont

\begin{xltabular}{\linewidth}{| P{1.8cm} | X | X | X | X | X | X | X |}
\hline
\multicolumn{1}{|c}{\multirow{2}{*}{\textbf{Criteria}}} &
  \multicolumn{7}{c|}{\textbf{Standard}} \\ \cline{2-8}
\multicolumn{1}{|c}{} &
  \multicolumn{1}{c|}{\textbf{Exceptional ~ (7)}} &
  \multicolumn{1}{c|}{\textbf{Advanced ~ (6)}} &
  \multicolumn{1}{c|}{\textbf{Proficient ~ (5)}} &
  \multicolumn{1}{c|}{\textbf{Functional ~ (4)}} &
  \multicolumn{1}{c|}{\textbf{Developing ~ (3)}} &
  \multicolumn{1}{c|}{\textbf{Little Evidence ~ (2)}} &
  \multicolumn{1}{c|}{\textbf{No Evidence ~ (1)}} \\ \hline
\endhead
%

\textbf{Depth\newline30\%} &
Provides a thorough, critical analysis of the architecture, addressing key strengths, weaknesses, and how well it meets the ASRs and quality attributes. The critique is insightful, balanced, and well-supported by evidence.&
Provides a comprehensive critique with clear analysis of the architecture's strengths, weaknesses, and how it addresses ASRs and quality attributes. Some evidence supports the critique. &
Critique is generally well-developed, covering major strengths and weaknesses, though it may lack some depth or specific evidence.&
Critique is adequate but lacks depth, with only superficial analysis of strengths, weaknesses, and ASRs. &
Critique is somewhat vague, with limited analysis of the architecture's strengths and weaknesses. &
Critique lacks meaningful analysis or focuses only on minor or irrelevant points. &
No meaningful critique is provided, or it fails to identify any strengths or weaknesses of the architecture. \\
\hline

\textbf{Relevance\newline25\%} &
Critique is closely aligned with the ASRs and quality attributes, offering a clear and detailed explanation of how well the architecture meets them. &
Critique is mostly aligned with ASRs and quality attributes, discussing their impact on the architecture effectively. &
Critique references ASRs and quality attributes, but the connection is not always clear or well-supported. &
Critique mentions ASRs and quality attributes, but the connection to the architecture is weak or unclear. &
Critique makes limited or superficial reference to ASRs or quality attributes. &
Critique mentions ASRs and quality attributes but fails to connect them to the architecture. &
Critique is entirely disconnected from the ASRs and quality attributes.\\
\hline

\textbf{Balanced Evaluation\newline15\%} &
Provides a well-balanced critique, discussing both strengths and weaknesses in a fair, objective, and constructive manner. &
Provides a fairly balanced critique, discussing both strengths and weaknesses, but may focus slightly more on one side.	&
Critique discusses strengths and weaknesses, but the evaluation may be unbalanced, focusing more on one aspect than the other. &
Critique covers strengths and weaknesses, but may not be sufficiently balanced or may favor one aspect too much. &
Critique lacks balance, focusing more on weaknesses or strengths, without giving adequate attention to the other side. &
Critique is unbalanced, only discussing strengths or weaknesses in detail with little consideration of the other side. &
Critique is entirely one-sided or overly negative without recognizing any positive aspects of the architecture. \\
\hline

\textbf{Presentation\newline 10\%} &
Presentation is well paced and delivered fluently. Information is logically sequenced, with clear objectives making it very easy to follow. &
Presentation is well paced and delivered clearly. Information is logically sequenced, with some clear objectives making it easy to follow. &
Presentation is mostly well paced and~de\-livered clearly. Information is logically sequenced, with signposting guiding audience through presentation. &
Presentation pace~is a little inconsistent or delivery is occasionally unclear. Information is logically sequenced allowing audience to follow presentation fairly well. &
Presentation pace~is inconsistent or delivery is sometimes unclear. Information is not always logically sequenced, distracting audience from presentation flow. &
Presentation pace~is inconsistent or delivery is unclear. Infor- mation is not logically sequenced, and planned progression was not clear to audience. &
Presentation pace~is inconsistent and~delivery is unclear. Infor- mation is poorly sequenced, confusing audience. \hline

\end{xltabular}

\clearpage

\subsection*{Presentation \#3 Detailed Design}

\fontsize{9}{11}\selectfont

\begin{xltabular}{\linewidth}{| P{1.8cm} | X | X | X | X | X | X | X |}
\hline
\multicolumn{1}{|c}{\multirow{2}{*}{\textbf{Criteria}}} &
  \multicolumn{7}{c|}{\textbf{Standard}} \\ \cline{2-8}
\multicolumn{1}{|c}{} &
  \multicolumn{1}{c|}{\textbf{Exceptional ~ (7)}} &
  \multicolumn{1}{c|}{\textbf{Advanced ~ (6)}} &
  \multicolumn{1}{c|}{\textbf{Proficient ~ (5)}} &
  \multicolumn{1}{c|}{\textbf{Functional ~ (4)}} &
  \multicolumn{1}{c|}{\textbf{Developing ~ (3)}} &
  \multicolumn{1}{c|}{\textbf{Little Evidence ~ (2)}} &
  \multicolumn{1}{c|}{\textbf{No Evidence ~ (1)}} \\ \hline
\endhead
%

\textbf{Selection of Design Focus\newline15\%} &
An important and significant part of the system was selected, showing excellent judgement. &
A relevant and fairly significant part of the system was selected, which reflects key design complexity or importance. &
A reasonable part of the system was selected to present in detail.	&
Design focus is acceptable, but may not show the most relevant or complex aspect of the system. &
Design focus is only partially appropriate.	&
Focus is weak or only marginally relevant to understanding the detailed design.	&
Design focus is inappropriate, trivial, or disconnected from the system. \\
\hline

\textbf{Design Clarity and\newline Completeness\newline30\%} &
Detailed design is presented clearly and comprehensively, with excellent coverage of component responsibilities and interactions. &
Detailed design is mostly clear and complete, effectively showing how components interact and function.	&
Design is generally clear, with most responsibilities and flows explained; minor gaps may exist. &
Design is presented in an understandable way, though some areas are underdeveloped or unclear.	&
Design presentation lacks detail or clarity in key parts, limiting understanding.	&
Design is hard to follow or significantly incomplete. &
Design is confusing, vague, or missing critical information. \\
\hline

\textbf{Design\newline ~~Diagrams\newline25\%} &
All diagrams are easy to comprehend, convey important information, and enhance the presentation. &
Most diagrams are easy to comprehend, convey important~in\-formation, and are used well in the presentation. &
Most diagrams are comprehensible, convey useful information, and are used well in the presentation. &
Most diagrams are comprehensible, convey useful information, and are connected to the presentation. &
Most diagrams are comprehensible, convey some useful information, and are mostly connected to the presentation. &
Some diagrams are incomprehensible, do not convey useful information, or are disconnected from the presentation. &
Most diagrams are incomprehensible, do not convey useful information, or are disconnected from the presentation. \\
\hline

\textbf{Presentation\newline 10\%} &
Presentation is well paced and delivered fluently. Information is logically sequenced, with clear objectives making it very easy to follow. &
Presentation is well paced and delivered clearly. Information is logically sequenced, with some clear objectives making it easy to follow. &
Presentation is mostly well paced and~de\-livered clearly. Information is logically sequenced, with signposting guiding audience through presentation. &
Presentation pace~is a little inconsistent or delivery is occasionally unclear. Information is logically sequenced allowing audience to follow presentation fairly well. &
Presentation pace~is inconsistent or delivery is sometimes unclear. Information is not always logically sequenced, distracting audience from presentation flow. &
Presentation pace~is inconsistent or delivery is unclear. Infor- mation is not logically sequenced, and planned progression was not clear to audience. &
Presentation pace~is inconsistent and~delivery is unclear. Infor- mation is poorly sequenced, confusing audience. \hline

\end{xltabular}


\clearpage

\subsection*{Presentations \#4 and \#5 Comparison}

\fontsize{9}{11}\selectfont

\begin{xltabular}{\linewidth}{| P{1.8cm} | X | X | X | X | X | X | X |}
\hline
\multicolumn{1}{|c}{\multirow{2}{*}{\textbf{Criteria}}} &
  \multicolumn{7}{c|}{\textbf{Standard}} \\ \cline{2-8}
\multicolumn{1}{|c}{} &
  \multicolumn{1}{c|}{\textbf{Exceptional ~ (7)}} &
  \multicolumn{1}{c|}{\textbf{Advanced ~ (6)}} &
  \multicolumn{1}{c|}{\textbf{Proficient ~ (5)}} &
  \multicolumn{1}{c|}{\textbf{Functional ~ (4)}} &
  \multicolumn{1}{c|}{\textbf{Developing ~ (3)}} &
  \multicolumn{1}{c|}{\textbf{Little Evidence ~ (2)}} &
  \multicolumn{1}{c|}{\textbf{No Evidence ~ (1)}} \\ \hline
\endhead
%

\textbf{Alternative Selection\newline15\%} &
Clearly identifies a relevant and credible alternative architecture, with strong justification for its suitability for the project. &
Identifies a relevant alternative architecture with a good, but not thorough, justification. &
Identifies a plausible alternative architecture but justification of its suitability is a little weak.	&
Identifies a plausible alternative architecture but with minimal justification or clarity of why it's viable. &
Identifies an alternative, but the choice may be weak or poorly explained. &
Identifies an alternative that is irrelevant or unclear. &
Does not identify any meaningful alternative architecture or design philosophy. \\
\hline

\textbf{Comparison\newline30\%} &
Provides a highly detailed and insightful comparison of the chosen architecture and alternative, covering key dimensions. Clearly explains which architecture is more suitable and why. &
Provides an informative and clear comparison, covering key dimensions, with good reasoning behind the preference for the chosen architecture. &
Provides a good comparison, touching on the main aspects, though the explanation may lack depth or full clarity in some areas. &
Comparison addresses key aspects, but it lacks depth in areas such as complexity or the impact on ASRs.	&
Comparison is basic and lacks clarity. &
Provides only superficial comparisons, missing key aspects of the architectures or failing to explain their impact on the system. &
Comparison is poorly developed or nonexistent, providing minimal or no insights into how the two architectures compare. \\
\hline

\textbf{Trade-off Analysis\newline25\%} &
Provides a thorough analysis of the trade-offs involved in choosing the alternative, detailing both its strengths and weaknesses, and how these trade-offs might impact the overall system.	&
Provides a strong analysis of trade-offs, with a clear explanation of how the alternative would affect the system's quality attributes.	&
Identifies major trade- offs but lacks a detailed explanation of how they would impact the project's quality attributes.	&
Provides a basic analysis of trade-offs, but lacks depth in understanding their potential impact on the project. &
Mentions trade-offs but provides limited insight into their impact on the overall system, or the trade-offs are unclear. &
Provides minimal analysis of trade-offs, with little connection to system goals or project needs. &
No trade-off analysis is provided, or it is wholly inadequate or unsubstantiated. \\
\hline

\textbf{Presentation\newline 10\%} &
Presentation is well paced and delivered fluently. Information is logically sequenced, with clear objectives making it very easy to follow. &
Presentation is well paced and delivered clearly. Information is logically sequenced, with some clear objectives making it easy to follow. &
Presentation is mostly well paced and~de\-livered clearly. Information is logically sequenced, with signposting guiding audience through presentation. &
Presentation pace~is a little inconsistent or delivery is occasionally unclear. Information is logically sequenced allowing audience to follow presentation fairly well. &
Presentation pace~is inconsistent or delivery is sometimes unclear. Information is not always logically sequenced, distracting audience from presentation flow. &
Presentation pace~is inconsistent or delivery is unclear. Infor- mation is not logically sequenced, and planned progression was not clear to audience. &
Presentation pace~is inconsistent and~delivery is unclear. Infor- mation is poorly sequenced, confusing audience. \hline

\end{xltabular}

\clearpage

\subsection*{Presentation \#6 Security}

\fontsize{9}{11}\selectfont

\begin{xltabular}{\linewidth}{| P{1.8cm} | X | X | X | X | X | X | X |}
\hline
\multicolumn{1}{|c}{\multirow{2}{*}{\textbf{Criteria}}} &
  \multicolumn{7}{c|}{\textbf{Standard}} \\ \cline{2-8}
\multicolumn{1}{|c}{} &
  \multicolumn{1}{c|}{\textbf{Exceptional ~ (7)}} &
  \multicolumn{1}{c|}{\textbf{Advanced ~ (6)}} &
  \multicolumn{1}{c|}{\textbf{Proficient ~ (5)}} &
  \multicolumn{1}{c|}{\textbf{Functional ~ (4)}} &
  \multicolumn{1}{c|}{\textbf{Developing ~ (3)}} &
  \multicolumn{1}{c|}{\textbf{Little Evidence ~ (2)}} &
  \multicolumn{1}{c|}{\textbf{No Evidence ~ (1)}} \\ \hline
\endhead
%

\textbf{Security\newline Threats\newline30\%} &
Clearly and comprehensively identifies all security threats specific to the system, with a deep understanding of their potential impact. &
Identifies key security threats and risks, with a good understanding of their potential impact. &
Identifies several security threats, though some may be less relevant or insufficiently detailed. &
Identifies a few key security threats but misses some major ones or provides insufficient detail. &
Identifies only a limited range of security threats, missing major threats that could impact the system. &
Provides an incomplete or unclear iden- tification of security threats, omitting cri- tical issues.	&
Fails to identify or improperly identifies the security threats. \\
\hline

\textbf{Mitigations\newline30\%} &
Thoroughly explains the security mechanisms used to mitigate the identified threats, and how they are integrated into the architecture. Mechanisms are clearly linked to specific threats.	&
Explains the security mechanisms effectively, and links them to the identified threats and risks, with minor gaps in explanation. &
Provides a good explanation of security mechanisms, but some parts lack clarity or details of how they address specific threats. &
Provides an explanation of security mechanisms, but with vague or incomplete descriptions of how they mitigate the risks.	&
Provides a minimal explanation of security mechanisms, leaving out key details or failing to fully connect them to identified threats.	&
The explanation of security mechanisms is unclear or disconnected from the identified threats, with many important aspects missing.	&
Does not explain the security mechanisms or fails to show how they address the identified threats. \\
\hline

\textbf{Remaining Challenges\newline10\%} &
Thoroughly identifies any remaining security challenges or risks in the architecture and suggests thoughtful, feasible improvements. &
Identifies remaining security challenges with a good explanation of potential future improvements and strategies to address them. &
Acknowledges some remaining security challenges but does not offer concrete or comprehensive strategies for improvement. &
Identifies some challenges but does not offer specific or actionable recommendations for future improvements. &
Mentions remaining security issues, but provides no or very weak suggestions for improvement. &
Superficial identification of remaining security challenges or improvement oppor- tunities. &
Fails to identify remaining security chal- lenges or improvement opportunities, or completely overlooks the topic. \\
\hline

\textbf{Presentation\newline 10\%} &
Presentation is well paced and delivered fluently. Information is logically sequenced, with clear objectives making it very easy to follow. &
Presentation is well paced and delivered clearly. Information is logically sequenced, with some clear objectives making it easy to follow. &
Presentation is mostly well paced and~de\-livered clearly. Information is logically sequenced, with signposting guiding audience through presentation. &
Presentation pace~is a little inconsistent or delivery is occasionally unclear. Information is logically sequenced allowing audience to follow presentation fairly well. &
Presentation pace~is inconsistent or delivery is sometimes unclear. Information is not always logically sequenced, distracting audience from presentation flow. &
Presentation pace~is inconsistent or delivery is unclear. Infor- mation is not logically sequenced, and planned progression was not clear to audience. &
Presentation pace~is inconsistent and~delivery is unclear. Infor- mation is poorly sequenced, confusing audience. \hline

\end{xltabular}

\end{landscape}


\end{document}
