\documentclass{csse4400}

\usepackage{CJKutf8}  % Support Chinese characters.
\usepackage{fancyhdr}

% RUBRIC
\usepackage{multirow}
\usepackage{array}
\usepackage{xltabular}
\usepackage{pdflscape}
\usepackage{enumitem}
\usepackage{hyperref}

\newcolumntype{P}[1]{>{\centering\arraybackslash}p{#1}}
% RUBRIC

\title{Architecture Presentation}
\author{Richard Thomas \& Guangdong Bai}
\date{Semester 1, 2025}

\begin{document}

% Custom footer with UQ copyright notice to facilitate takedown requests at academic file sharing sites (e.g. Course Hero or Chegg).
% Requires document to \usepackage{fancyhdr}.

\pagestyle{fancy}

% Remove all default header content.
\fancyhead{} % Clear default header fields.
\renewcommand{\headrulewidth}{0pt} % Remove horizontal rule from header.

% Set footer details.
\setlength{\footskip}{15mm}
\fancyfoot{}  % Clear default footer fields.
\fancyfoot[L]{\small \copyright \ The University of Queensland \the\year\ }
\fancyfoot[R]{\small Page \thepage}

\maketitle

\section*{Summary}

In this assessment, you will deliver a presentation around the software architecture of your team's capstone project.
Your task is to explain the design decisions made, critically evaluate your architecture, compare it with viable alternatives, and demonstrate awareness of architectural trade-offs.

This assessment item is designed to showcase your skills to solve problems as a software architect. You should demonstrate:
\begin{enumerate}
    \item a clear understanding of software architecture principles,
    \item the ability to evaluate and defend architectural choices, and
    \item effective communication of complex architectural concepts.
\end{enumerate}


\section{Introduction}

Software architecture forms the foundation of every software system. Whether you are designing a cloud-based service, a data processing platform, or a mobile application, software architecture defines how well the software system meets its goals such as scalability, maintainability, and security.

As part of your capstone project, your team has collaboratively developed the architecture of your software system.
In this assessment item, you will conduct a deep analysis of the architecture.
You will provide not only a description of your architecture, but a critical evaluation of how well your architecture meets its intended goals.

Each team is allocated a time slot for your presentation.
During your team's presentation, each member will focus on a specific architectural concern, such as architectural design, detailed design, design rationale and trade-offs, alternative patterns, and security.
You will be \textbf{assessed individually}.


\section{Presentation}

\subsection{Presentation Scheduling}

\begin{itemize}
  \item Each team will be allocated \textbf{35 minutes} during \textbf{Week 13} to deliver their presentations. This time includes setup, individual presentations, and a brief Q\&A/discussion.
  \item The teaching team will provide a list of available time slots, which will primarily take place during the scheduled lecture hours, case study sessions and practical sessions of Week 13.
  \item Your team must \textbf{collaboratively choose a single time slot} for which all team members are available. \textbf{All team members are required to be present during the allocated session}. This includes being available to present your assigned topic individually, being present for your teammates' presentations, and being available for answering questions regarding your project.
  \item Please inform the course coordinators of any constraints you may have regarding presentation time \textbf{\textit{before}} 9 May 2025.
\end{itemize}






\subsection{Presentation Content}

Every team member is free to structure your presentation however you wish, though the team together needs to cover the following content.

\begin{description}
    \item[Title Slide] Name of the software project, and names and student numbers of team members in the order of presentation.
    \item[Introduction and Context] Describe the software project, explaining its key functionality and target users, and provide an overview of the software system's context and its external dependencies.
    \item[Architecture] Describe the software's architecture, and the Architecturally Significant Requirements (ASR) of most importance to the project.
    \item[Detailed Design] Describe the internals of key components or subsystems.
    \item[Critique] Analyse the software's architecture, describing how well it delivers its ASRs.
    \item[Comparison] Compare your architecture with chosen viable alternatives.
    \item[Security] Describe the security concerns and your mitigation mechanisms in your architecture.
    \item[Conclusion] Highlight the key points of your team's presentation.
\end{description}



\subsection{Presentation Guidelines}


\subsubsection{Presenter \#1 (6.5 mins): Title Slide, Introduction and Context, Architecture}

Your presentation should start with the introduction of your team's capstone project \textbf{within 1.5 mins}.
Give an elevator pitch style summary of what problem the project solves and its key features.


Describe the project's software architecture using appropriate views \cite{view-notes}.
You must use the C4 modelling notation \cite{view-notes} \cite{brown2022c4} to describe the software architecture.
You may supplement the C4 diagrams with other diagrams to help describe the architecture.
For example, you may use UML use case, class, or sequence diagrams \cite{view-notes} \cite{uml}
to describe system requirements or details of how the architectural design works.
Other diagrams may also be used, if they clarify aspects of your C4 model.
Any diagrams obtained from other sources  must be cited.

Your description of the software architecture should cover all of its important aspects.
You are not expected to get down to the level of describing the detailed design of the software, which will be done by \textbf{Presenter \#2}, nor the design trade-offs, which will be done by \textbf{Presenter \#3}.
You should not need to provide class or dynamic diagrams for the entire system.

Your audience is other students in this course. You may assume the audience has knowledge of the course content,
though you should not assume they are familiar with the project you are describing.


\subsubsection{Presenter \#2 (5 mins): Critique}


Your presentation should deliver a \textbf{critical evaluation of your software architecture}, focusing on how well it addresses the ASRs and supports the overall project goals.

Begin by describing which ASRs and, in particular, the quality attributes you think are most important for the project, and why.
Then, assess how well the architecture you've designed satisfies those attributes.
Your presentation is advised to be specific to particular design choices or structures, and to discuss how they contribute to or potentially hinder your quality goals.

You should also highlight any limitations, trade-offs, or compromises that were made.


\subsubsection{Presenter \#3 (5 mins): Detailed Design}

Your presentation should focus on the \textbf{detailed design} of \textbf{one or two key parts} of your capstone project. The goal is to show how architectural decisions are designed at a lower level, and how key components work together to fulfill the system's ASRs and quality attributes.

Select only \textbf{one or two} significant components, services, or subsystems. Your selected one(s) should be central to the system's functionality. Describe their internal structure, key interfaces, and important interactions with other parts of the system.

You should use appropriate UML diagrams to support your explanation. Depending on what you are describing, these may include:

\begin{itemize}
    \item \emph{Class diagrams} to show internal structure and relationships
    \item \emph{Sequence diagrams} to show interactions between components
    \item \emph{State diagrams} if the component involves complex state transitions
    \item \emph{Activity diagrams} for workflow or process modeling
\end{itemize}

Explain any design patterns, principles, or trade-offs applied in your detailed design. For example, if you've used an adapter, explain why, and how it contributes to quality attributes.

You are not expected to describe the full system in detail. Focus on the parts that are most important or interesting from an architectural perspective.

Assume your audience is not familiar with the inner workings of your project. You are expected to clearly communicate how your detailed design brings the architecture to life.


\subsubsection{Presenters \#4 and \#5 (each 5 mins): Comparison}

For each presenter, your task is to compare your team's chosen software architecture with \textbf{one viable alternative}.

Your presentation is advised to begin by clearly identifying the alternative architecture. This could be an alternative style (e.g., microservices v.s. monolithic), or even a fundamentally different design philosophy.

Explain what makes this alternative a \textbf{credible} option for your project. What trade-offs would it involve? What problems might it solve better? What new challenges would it introduce?
Then, compare your chosen architecture and the alternative along key dimensions such as: support for ASRs, complexity, and team expertise.

You can use diagrams, tables, or summaries where appropriate to make your comparison clear and visual.


\subsubsection{Presenter \#6 (5.5 mins): Security, Conclusion}

Your presentation should focus on the security aspects of your software architecture. Discuss the key security concerns specific to your project and how the architecture is designed to address them.

Start by identifying the primary security threats or risks your system faces, such as unauthorized access, data breaches, and Denial of Service (DoS).

Explain the security mechanisms built into your architecture to mitigate these threats, such as authentication and authorization strategies, and data protection measures.
You are advised to use security design patterns (e.g., secure-by-design, defense-in-depth).

Highlight any remaining security challenges or areas where future improvements could be made.



\subsection{Citations \& References}
You may use references in your presentation to support points you are making.
These must be cited and referenced using the \link{IEEE referencing style}
{https://libraryguides.vu.edu.au/ieeereferencing/gettingstarted}.
The final slide(s) of your presentation should include the references to any cited material.
You should display the reference slide(s) for about 3 seconds at the end of your presentation.
You are not required to speak to the reference slides,
aside from possibly thanking your audience for listening and stating these are your references.


\subsection{Presentation Hints}

If your presentation exceeds the designated minutes, the marker will ask you to stop your presentation.
It there is a presenter to follow you, they will be asked to step forward and start their section of the presentation.
No content of your presentation past that point will be marked.

As a presenter, you should not read a script.
You may wish to write a script to prepare for the presentation but should not read it during the presentation.
You may make use of notes during the presentation but you should only quickly glance at your notes to keep yourself on track.
You should not be constantly referring to notes.
You should try to maintain eye contact with your audience, rather than focussing on your notes or slides.

\section{Identity Verification}
The presentation is an identity verified assignment.
You must make your presentation in-person.
At the start of your presentation you must show your UQ student card to one of the markers at your session.
Like in an exam situation, if you have lost your student card
you must obtain a temporary identity verification document from the UQ student centre \emph{before} your presentation.

%The marked result of your presentation will be used to determine any caps applied to your grade.
%(That means failing the presentation because you did not submit the required number of peer evaluations
%will \textbf{\textit{not}} affect the mark used to determine a final grade cap.)
%The first slide of your presentation \textbf{must} contain your full name, as recorded in UQ's student enrolment system,
%and full 8-digit student number.

%\subsection{On-line Identity Verification}
%If you are are an external student it does not matter if your UQ student card has expired.
%If you do not have a UQ student card, you may use an official government photo id that shows your full name.
%Your id must be clearly visible for at least 3 seconds.
%If a marker cannot view your card clearly enough, they will ask you to move it so it is clearly readable.
%
%\begin{CJK*}{UTF8}{gbsn}
%If your government id does not show your name in Roman characters, as recorded in UQ's student enrolment system,
%you need to include a clear image of your government id on your first slide and a textual
%representation of your name that can be selected and copied from your slide so that it may be pasted into a translator.
%(e.g. If you use your China Resident Identity Card, you must provide clear images of the front and back
%of the card. You also need to provide a textual representation of your name in Chinese characters, e.g. 蒙晶.)
%\end{CJK*}
%
%Your face must be visible throughout the presentation to show that you are the one speaking during the presentation.
%This may be through Zoom's participants window.
%If you cannot arrange for your face to be visible throughout the presentation,
%you \textbf{must} contact the course coordinator before 28 April 2023 to discuss your constraints.

%\subsection{On-Campus Identity Verification}
%If you are presenting on-campus, you \textbf{must} show the marker your current and valid UQ student card.
%Like in an exam situation, if you have lost your student card
%you must obtain a temporary identity verification document from the UQ student centre \emph{before} the presentation.


\section{Submission}


There are two components that make up your assessable content for the presentation, i.e., the slides you use for your presentation, and the presentation itself.

%\subsection{Draft Model}
%You must show a tutor a draft architectural model of your selected system in your tutorial in week 6 (March 30).
%The model must include appropriate views that give an overview of the key aspects of the system's software architecture.
%You may need to give the tutor a one minute overview of the project you have selected and its key goals.
%If your provided model is not an appropriate overview of the system
%(e.g. too superficial, missing key parts, or too detailed)
%the grade you achieve for the presentation will be reduced by one grade level.

\subsection{Slides}
The slides for your presentation are to be submitted as a PDF file to a link provided on BlackBoard.
Your slides are due at 11:00am on May 26 2025.
Late submission of your slides will result in a penalty of 1 grade per 24 hour period that they are late.
Regardless of any penalty applied to the presentation, \emph{even} if the penalty is a failing grade,
you \textbf{\textit{must}} still make your presentation in your allocated timeslot.

\subsection{Presentation}
Your presentation is to use the slides you submit to BlackBoard.
If you do not deliver your presentation, your final grade will be capped at a failing grade.
If you are unable to attend your session to give your presentation due to exceptional circumstances,
you may apply to defer your presentation to another date.
You are not able to defer a deferred presentation.
Please find more information in the \link{course profile}{https://course-profiles.uq.edu.au/course-profiles/CSSE6400-21553-7520}.

%\subsection{Peer Evaluation}
%You are expected to attend all presentations.
%You are required to submit an evaluation of each presentation you observe.
%Submission of \emph{meaningful} feedback for at least \textbf{75\%} of the presentations in your class sessions
%is required to obtain a passing grade or higher for the presentation assessment.
%
%An online form will be provided for you to submit your evaluation for each presentation.
%You must submit your evaluation of each presentation separately in order for the system to record all of your evaluations.
%
%If you are unable to attend a practical session due to exceptional circumstances,
%and miss viewing several presentations,
%you may apply for a modified limit on the number of presentations you must evaluate.


\section{Academic Integrity}
As this is a higher-level course, you are expected to be familiar with the importance of academic integrity in general,
and the details of UQ's rules.
If you need a reminder, review the \link{Academic Integrity Modules}
{https://web.library.uq.edu.au/library-services/it/learnuq-blackboard-help/academic-integrity-modules}.
Submissions will be checked to ensure that the work submitted is not plagiarised.
If you have quoted or paraphrased any material from another source, it must be correctly \link{cited and referenced}
{https://guides.library.uq.edu.au/referencing}.
Use the \link{IEEE referencing style}{https://libraryguides.vu.edu.au/ieeereferencing/gettingstarted} for citations and your bibliography.

Note that text generated by an AI tool, such as ChatGPT, is based on text from the Internet.
Consequently all text, whether written on slides or spoken during a presentation,
that was generated by an AI tool must be cited.

Uncited or unreferenced material will be treated as not being your own work.
Extensive quotation or minor rephrasing of material from cited sources should be avoided.
Significant amounts of cited material from other sources, even if paraphrased, will be considered to be of no academic merit.
In all cases, any material that you cite must support the arguments and points that you are making in your presentation.


\bibliographystyle{ieeetr}
\bibliography{ours,books}


%\section*{Draft Model Criteria}
%
%\begin{table}[h]
%\centering
%\footnotesize
%\begin{tabular}{|p{2cm}|p{7.5cm}p{7.5cm}|}
%\hline  & \multicolumn{2}{c|}{\textbf{Standard}}  \\ \hline
%\multicolumn{1}{|c|}{\textbf{Criteria}} & \multicolumn{1}{c|}{\textbf{Acceptable}}  & \multicolumn{1}{c|}{\textbf{Not Sufficient}}  \\ \hline
%\textbf{Context}               & \multicolumn{1}{p{7.5cm}|}{Provides a generally clear overview of the system.}    & System's scope and usage context are not clear.  \\ \hline
%\textbf{ASRs}                  & \multicolumn{1}{p{7.5cm}|}{Identifies seemingly important goals and constraints.} & Important goals and constraints are not clear or not identified.                                                        \\ \hline
%\textbf{Architecture Diagrams} & \multicolumn{1}{p{7.5cm}|}{Provide an overview of the system’s architectural structure. They also demonstrate an initial understanding of parts of the system’s internal design.} & Provides a superficial overview of the architecture structure, or architectural design is lost in system design detail. \\ \hline
%\end{tabular}
%\end{table}


\clearpage

\newgeometry{left=12mm,right=7mm,top=5mm,bottom=12mm}

\begin{landscape}

\fontsize{9}{11}\selectfont

\begin{xltabular}{\linewidth}{| P{1.55cm} | X | X | X | X | X | X | X |}
\hline
\multicolumn{1}{|c}{\multirow{2}{*}{\textbf{Criteria}}} &
  \multicolumn{7}{c|}{\textbf{Standard}} \\ \cline{2-8} 
\multicolumn{1}{|c}{} &
  \multicolumn{1}{c|}{\textbf{Exceptional ~ (7)}} &
  \multicolumn{1}{c|}{\textbf{Advanced ~ (6)}} &
  \multicolumn{1}{c|}{\textbf{Proficient ~ (5)}} &
  \multicolumn{1}{c|}{\textbf{Functional ~ (4)}} &
  \multicolumn{1}{c|}{\textbf{Developing ~ (3)}} &
  \multicolumn{1}{c|}{\textbf{Little Evidence ~ (2)}} &
  \multicolumn{1}{c|}{\textbf{No Evidence ~ (1)}} \\ \hline
\endhead
%
\textbf{System\newline Scope\newline20\%} &
MVP's originally proposed functional \& non-functional requirements, or those agreed \& documented early in the project, are fully delivered. &
MVP's originally proposed functional \& non-functional~require\-ments, or those agreed \& documented early in the project, are delivered with small variances. &
MVP's functional \& non-functional requirements were revised \& documented later in the project, and are almost fully delivered. &
All important functional \& non-functional requirements are delivered but some other requirements are not, whether or not original plan was revised. &
Most important functional \& non-functional requirements are delivered, whether or not original plan was revised. &
Some important functional \& non-functional requirements are delivered, whether or not original plan was revised. &
Few important functional \& non-functional requirements are delivered, whether or not original plan was revised. \\
\hline

\textbf{Architecture\newline Suitability\newline 15\%} &
Delivered architecture, supplemented by the design reflection, is very well suited to delivering all specified functional \& non-functional require\-ments, including an appropriate level of security. &
Delivered architecture, supplemented by the design reflection, is~well suited to delivering~al\-most all specified functional \& non-functional requirements, including an appropriate level of security. &
Delivered architecture, supplemented by the design reflection, is fairly well suited to delivering the key functional \& non-functional requirements, including a mostly appropriate level of security. &
Delivered architecture, supplemented by the design reflection, is capable of delivering most key functional \& non-functional requirements, including a mostly appropriate level of security. &
Delivered architecture, supplemented by the design reflection, requires workarounds in a few cases to deliver key functional \& non-functional requirements. Design has one or two obvious security issues. &
Delivered architecture, supplemented by the design reflection, requires workarounds in several cases to deliver key functional \& non-functional requirements. Design has a few obvious security issues. &
Delivered architecture, supplemented by the design reflection, makes it difficult to deliver many functional \& non-functional requirements. Design does not appear to consider security issues. \\
\hline

\textbf{Testing\newline Quality\newline 20\%} &
All functional \& non-functional requirements, \& architectural components are well tested (or are described well in a test plan) and, where feasible, are automated. &
Most key functional \& non-functional require\-ments, \& key architec\-tural components are well tested (or are described adequately in a test plan) and, where feasible, are mostly automated. &
Most key functional \& non-functional require\-ments, \& key architec\-tural components are fairly well tested (or are described fairly adequately in a test plan) and, where feasible, many are automated. &
Most key functional \& non-functional require\-ments, \& key architectural components are fairly well tested (or are described fairly adequately in a test plan) and, with some attempt at automation. &
Main test cases for most key functional \& non-functional requirements, \& key architectural components are fairly well tested (or have some informative description in a test plan). &
Main test cases for a few key functional \& non-functional requirements, \& key architectural components are moderately well tested (or have a general description in a test plan). &
Testing is poor, superficial or extremely limited. Or, extent of testing cannot be determined from submitted artefacts. \\
\hline

\textbf{Architecture\newline Description\newline 25\%} &
Clear, accurate, concise \& complete description of all aspects of the architecture. Diagrams \& narrative text complement each other. Views enhance understanding all aspects of the architecture. Choice of architecture, \& decisions about design trade-offs, are well described. &
Clear, accurate \& mostly complete description of the architecture. Diagrams \& narrative text complement each other. Views support description of the architecture. Choice of architecture, \& decisions about important design trade-offs, are well described. &
Mostly clear, accurate \& complete description of the architecture. Diagrams \& narrative text support each other. Views support some description of the architecture. Choice of architecture, \& decisions about most important design trade-offs, are adequately described. &
Fairly clear, \& mostly accurate \& complete,~des- cription of the architecture. Diagrams \&~narra- tive text are consistent. Views provide little~sup- port describing the architecture.  Choice of~archi- tecture \& decisions about some important design trade-offs, are fairly adequately described. &
Some parts of the description are unclear, in- accurate or incomplete. Most diagrams are relevant to the narrative text or a necessary diagram is missing. Justification of choice of architecture is unclear. Decisions about a few important design trade-offs are fairly adequately described. &
Some parts of the description are inaccurate or incomplete, or many parts are unclear. Some diagrams are relevant to the narrative text or a few necessary diagrams are missing. Poor justification of choice of architecture. Few design trade-offs are adequately described. &
Many parts of the description are unclear, inaccurate or incomplete. Few diagrams are relevant to the narrative text or many necessary diagrams are missing. No, or very poor, justification of choice of architecture. Trade-offs are poorly described. \\
\hline

\textbf{Architecture\newline Evaluation\newline 20\%} &
Critique \& evaluation clearly demonstrate that the delivered architecture, varied a little by the reflection comments, can deliver all functional \& non-functional requirements of the full system. &
Critique \& evaluation clearly demonstrate~that the delivered architecture, varied by the reflection comments, can deliver all functional \& non-functional requirements of the full system. &
Critique \& evaluation demonstrate that the delivered architecture, varied by the reflection comments, can deliver all important functional \& non-functional requirements of the full system. &
Critique \& evaluation demonstrate that the delivered architecture, varied by the reflection comments, can deliver all important functional \& non-functional requirements of the MVP \& part of the full system. &
Critique \& evaluation demonstrate that the delivered architecture, varied by the reflection comments, can deliver all important functional \& non-functional requirements of the MVP but little of the full system. &
Critique \& evaluation demonstrate that the delivered architecture, varied by the reflection comments, can deliver some important functional \& non-functional requirements of the MVP. &
Critique \& evaluation demonstrate that the delivered architecture, varied by the reflection comments, is unlikely to deliver most functional or non-functional requirements of the MVP. Or, they are too unclear to determine. \\
\hline

\end{xltabular}

\end{landscape}

\restoregeometry

\end{document}
