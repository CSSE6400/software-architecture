\documentclass{csse4400}

\usepackage{CJKutf8}

\title{Case Study Presentation}
\author{Brae Webb \& Richard Thomas}
\date{Semester 1, 2022}

\begin{document}
\maketitle

\section*{Summary}
In this assignment, you will be asked to demonstrate your ability to
\textsl{understand}, \textsl{communicate}, and \textsl{critique} an architecture of an existing software project.
\begin{enumerate}
    \item You will present the key information about the architecture of the project you documented in your case study assignment to your practical class.
    \item This will include an updated critique of the architecture and any other relevant updates to the information original provided in your report.
\end{enumerate}


\section{Introduction}
...


\section{Presentation Structure}

\begin{description}
    \item[Title] Name of the software project.
    \item[Abstract] Summarise the key points of your document.
    \item[Introduction] Describe the software project, explaining the its key functionality and target users.
    \item[Context] Provide an overview of the software system's context and its external dependencies.
    \item[Architecture] Describe the software's architecture.
    \item[Critique] Analyse the software's architecture, describing its advantages and disadvantages.
    \item[Conclusion] Highlight the key points or lessons learnt about the software's architecture.
\end{description}


\section{Presentation Content}
How you present the information in your report, is up to you. You will need to select an appropriate notation 
to provide a visual representation of the software's architecture. You need to select appropriate ways of describing
the architecture.

An important aspect of the software system's context is the quality attributes that are important to the success of the project.
You need to identify what you think are the important quality attributes for the project. 
You need to justify why these attributes would be important.
This may require considering how these attributes would be prioritised to make decisions when trade offs need to be made in the design.
Your critique should describe how well the software's architecture supports delivering the quality attributes that you identified as being important.

You may have noticed that a number of the documents produced by the students at TU Delft use \textsl{views} 
as a way of structuring their description of different aspects of the architecture. The idea of architectural views
is common to different approaches to documenting software architecture. The idea was popularised by Phillipe 
Krutchen in his 4+1 View Model of Software Architecture \cite{4+1-model}.
You should read this article to understand why different views are often a useful way to describe different
aspects of a software architecture. You do not need to use these views, but you may if you find them useful.

\section{Presentation}
Presentations will take place in your practical class sessions during weeks 9 to 13.
You will have six minutes for your presentation, plus two minutes for questions.
Your presentation should introduce the software project.
Give an elevator pitch style summary of what problem the project solves and its key features.
Describe which quality attributes you think are most important for the project, and why.
Describe the project's software architecture. Use appropriate views and notation to convey the important aspects of its architecture.
Summarise your critique of the software architecture, highlighting how well it supports delivering the project's key quality attributes.
Your audience is other students in this course. You may assume the audience has knowledge of the course content.

\begin{CJK*}{UTF8}{gbsn}
You are free to structure your presentation however you wish, though you should use some form of slides to support the delivery of information.
One constraint is that your first slide \textbf{must} contain your full name, as recorded in UQ's student enrolment system, and student number.
If you are presenting online, you \textbf{must} show your UQ student card (it does not matter if your UQ student card has expired),
or official government photo id that shows your full name. If your government id does not show your name in Roman characters,
as recorded in UQ's student enrolment system, you need to include a clear image of your government id on your first slide and a textual
representation of your name that can be selected and copied from your slide so that it may be pasted into a translator.
(e.g. If you use your China Resident Identity Card, you must provide clear images of the front and back
of the card. You also need to provide a textual representation of your name in Chinese characters, e.g. 蒙晶.)
\end{CJK*}

The presentation is an identity verified assignment.
You must make your presentation in-person.
The marked result of your presentation will be used to determine any caps applied to your grade.
(That means a late penalty on the submission of your slides will not affect the mark used to determine a grade cap.)


\section{Criteria}
...


\section{Submission}
The slides for your presentation are to be submitted to a link provided on BlackBoard.
Your slides are due at \textbf{13:00 (AEST) on 26 April 2022}.
Late submission of your slides will result in a penalty of 2\% of the maximum possible marks for the presentation, per minute that they are late.
Regardless of any penalty applied to the presentation, \emph{even} if the penalty is 100\%,
you \textbf{must} still make your presentation in your allocated timeslot.


\section{Academic Integrity}
As this is a higher-level course, you are expected to be familiar with the importance of academic integrity in general, and the details of UQ's rules.
If you need a reminder, review the \link{Academic Integrity Modules}{https://web.library.uq.edu.au/library-services/it/learnuq-blackboard-help/academic-integrity-modules}.
Submissions will be checked to ensure that the work submitted is not plagiarised.
If you have quoted or paraphrased any material from another source, it must be correctly \link{cited and referenced}{https://web.library.uq.edu.au/node/4221/2}.
Use the \link{IEEE referencing style}{https://libraryguides.vu.edu.au/ieeereferencing/gettingstarted} for citations and your bibliography.

Uncited or unreferenced material will be treated as not being your own work.
Extensive quotation or minor rephrasing of material from external sources should be avoided.
Large blocks of cited material from other sources, even if paraphrased, will be considered to be of no academic merit.
In all cases, any material that you cite must support the arguments and points that you are making in your presentation.


%\bibliographystyle{ieeetr}
%\bibliography{articles}

\end{document}