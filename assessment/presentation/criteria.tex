\clearpage
\begin{landscape}

\section*{Marking Criteria: Common Part~(20\%)}

All team members will be awarded the same result for the Title Slide, Introduction and Context
(by \textbf{Presenter \#1}), ASRs (by \textbf{Presenter \#2}), and Conclusion (by \textbf{Presenter \#6}).

\fontsize{9}{11}\selectfont

\begin{xltabular}{\linewidth}{| P{1.8cm} | X | X | X | X | X | X | X |}
\hline
\multicolumn{1}{|c}{\multirow{2}{*}{\textbf{Criteria}}} &
  \multicolumn{7}{c|}{\textbf{Standard}} \\ \cline{2-8}
\multicolumn{1}{|c}{} &
  \multicolumn{1}{c|}{\textbf{Exceptional ~ (7)}} &
  \multicolumn{1}{c|}{\textbf{Advanced ~ (6)}} &
  \multicolumn{1}{c|}{\textbf{Proficient ~ (5)}} &
  \multicolumn{1}{c|}{\textbf{Functional ~ (4)}} &
  \multicolumn{1}{c|}{\textbf{Developing ~ (3)}} &
  \multicolumn{1}{c|}{\textbf{Little Evidence ~ (2)}} &
  \multicolumn{1}{c|}{\textbf{No Evidence ~ (1)}} \\ \hline
\endhead
%
\textbf{~Context\newline 5\%} &
Project is introduced clearly and well situated within its context, providing an excellent starting point to understand the system. &
Project is introduced clearly with good~con\-textual information, providing a good starting point to understand the system. &
Project is introduced well with a good over\-view of its context, providing a clear but basic overview of the system. &
Project is introduced fairly well with some contextual informa\-tion, providing a com\-prehensible over\-view of the system. &
Project scope \& general context are fairly clear, providing a general overview of the system. &
Project scope \& context are not clear, providing a poor overview of the system. &
Project scope \& context are confusing, providing an inaccurate overview of the system. \\
\hline

\textbf{~ ~ASRs\newline 10\%} &
ASRs are clearly described, well justified, clearly of high importance, and all will influence architecture decisions. &
ASRs are clearly described, fairly well jus\-tified, seemingly of high importance, and all are likely to influ\-ence architecture decisions. &
Most ASRs are well described but a few justifications are a little weak. Most are important and likely to influence architecture decisions. &
Some ASRs are well described but a few justifications are weak. Most are important and likely to influence architecture decisions. &
Some ASRs are fairly well described but some justifications~are weak. Some are important and likely to influence architecture decisions. &
Most ASRs are poorly described or poorly justified. Few are im\-portant or likely to influence architecture decisions. &
Most ASRs are poorly described and poorly justified. Very few are important or likely to influence architecture decisions. \\
\hline

\textbf{~ ~Conclusion\newline 5\%} &
Conclusion provides a clear, well-structured summary of all key architectural points and offers insightful reflection on lessons learnt. &
Conclusion clearly summarises~most~key architectural points, and includes thoughtful reflection. &
Conclusion summarises main points clearly and includes some useful reflection. &
Conclusion presents a reasonable summary, though some points may be underdeveloped. &
Conclusion attempts to summarise key points but is vague or superficial. &
Conclusion is unclear or disorganised, with poor summarisation. &
Conclusion is confusing or missing. \\
\hline
\end{xltabular}

\clearpage


\section*{Marking Criteria: Individual Part~(80\%)}

\subsection*{Presentation \#1 Title Slide, Introduction and Context, Architecture}

\fontsize{9}{11}\selectfont

\begin{xltabular}{\linewidth}{| P{1.8cm} | X | X | X | X | X | X | X |}
\hline
\multicolumn{1}{|c}{\multirow{2}{*}{\textbf{Criteria}}} &
  \multicolumn{7}{c|}{\textbf{Standard}} \\ \cline{2-8}
\multicolumn{1}{|c}{} &
  \multicolumn{1}{c|}{\textbf{Exceptional ~ (7)}} &
  \multicolumn{1}{c|}{\textbf{Advanced ~ (6)}} &
  \multicolumn{1}{c|}{\textbf{Proficient ~ (5)}} &
  \multicolumn{1}{c|}{\textbf{Functional ~ (4)}} &
  \multicolumn{1}{c|}{\textbf{Developing ~ (3)}} &
  \multicolumn{1}{c|}{\textbf{Little Evidence ~ (2)}} &
  \multicolumn{1}{c|}{\textbf{No Evidence ~ (1)}} \\ \hline
\endhead
%

\textbf{Architecture\newline Completeness\newline25\%} &
Description is clear, complete, concise, and informative, resulting in an excellent and coherent understanding of the overall architecture and its major components. &
Description is clear, almost complete, and informative, resulting in a good and coherent understanding of the system's architecture and structure.	&
Description is mostly clear and informative, resulting in a good understanding of the system's architectural structure. &
Description is mostly clear and informative, though some architectural elements may be missing or underexplained. &
At times the description lacks clarity, leading to a vague or partial overview of the system's architecture. &
Description is unclear or incomplete, omitting important architectural elements or structure, leading to a poor understanding of the architecture. &
Description is confusing, severely incomplete, resulting in an incorrect or misleading understanding of the architecture. \\
\hline

\textbf{Architecture Clarity and\newline Consistency\newline20\%} &
Architectural structure is communicated with excellent clarity, logical flow, consistency and at an appropriate level of abstraction.  Relationships and res- ponsibilities between components~are~well explained and coherent. &
Structure is clearly presented and mostly consistent. Component responsibilities and relationships are explained well. &
Architecture is mostly clear and consistent, though some relationships or responsibilities may be weakly described.	Description is understandable but may lack cohesion, with minor inconsistencies or unclear relationships. &
Description is understandable~but~may lack cohesion, with minor inconsistencies or unclear relationships. &
Architectural explanation is somewhat disorganised or inconsistent, weakening the overall coherence. &
Explanation is unclear or inconsistent, making it difficult to follow architectural relationships. &
Explanation is highly inconsistent or incoherent, obscuring the system's architecture entirely. \\
\hline

\textbf{Design\newline ~~Diagrams\newline25\%} &
All diagrams are easy to comprehend, convey important information, and enhance the presentation. &
Most diagrams are easy to comprehend, convey important~in\-formation, and are used well in the presentation. &
Most diagrams are comprehensible, convey useful information, and are used well in the presentation. &
Most diagrams are comprehensible, convey useful information, and are connected to the presentation. &
Most diagrams are comprehensible, convey some useful information, and are mostly connected to the presentation. &
Some diagrams are incomprehensible, do not convey useful information, or are disconnected from the presentation. &
Most diagrams are incomprehensible, do not convey useful information, or are disconnected from the presentation. \\
\hline

\textbf{Presentation\newline 10\%} &
Presentation is well paced and delivered fluently. Information is logically sequenced, with clear objectives making it very easy to follow. &
Presentation is well paced and delivered clearly. Information is logically sequenced, with some clear objectives making it easy to follow. &
Presentation is mostly well paced and~de\-livered clearly. Information is logically sequenced, with signposting guiding audience through presentation. &
Presentation pace~is a little inconsistent or delivery is occasionally unclear. Information is logically sequenced allowing audience to follow presentation fairly well. &
Presentation pace~is inconsistent or delivery is sometimes unclear. Information is not always logically sequenced, distracting audience from presentation flow. &
Presentation pace~is inconsistent or delivery is unclear. Infor- mation is not logically sequenced, and planned progression was not clear to audience. &
Presentation pace~is inconsistent and~delivery is unclear. Infor- mation is poorly sequenced, confusing audience. \\
\hline

\end{xltabular}

\clearpage

\subsection*{Presentation \#2 Critique}

\fontsize{9}{11}\selectfont

\begin{xltabular}{\linewidth}{| P{1.8cm} | X | X | X | X | X | X | X |}
\hline
\multicolumn{1}{|c}{\multirow{2}{*}{\textbf{Criteria}}} &
  \multicolumn{7}{c|}{\textbf{Standard}} \\ \cline{2-8}
\multicolumn{1}{|c}{} &
  \multicolumn{1}{c|}{\textbf{Exceptional ~ (7)}} &
  \multicolumn{1}{c|}{\textbf{Advanced ~ (6)}} &
  \multicolumn{1}{c|}{\textbf{Proficient ~ (5)}} &
  \multicolumn{1}{c|}{\textbf{Functional ~ (4)}} &
  \multicolumn{1}{c|}{\textbf{Developing ~ (3)}} &
  \multicolumn{1}{c|}{\textbf{Little Evidence ~ (2)}} &
  \multicolumn{1}{c|}{\textbf{No Evidence ~ (1)}} \\ \hline
\endhead
%

\textbf{Depth\newline30\%} &
Provides a thorough, critical analysis of the architecture, addressing key strengths, weaknesses, and how well it meets the ASRs and quality attributes. The critique is insightful, balanced, and well-supported by evidence.&
Provides a comprehensive critique with clear analysis of the architecture's strengths, weaknesses, and how it addresses ASRs and quality attributes. Some evidence supports the critique. &
Critique is generally well-developed, covering major strengths and weaknesses, though it may lack some depth or specific evidence.&
Critique is adequate but lacks depth, with only superficial analysis of strengths, weaknesses, and ASRs. &
Critique is somewhat vague, with limited analysis of the architecture's strengths and weaknesses. &
Critique lacks meaningful analysis or focuses only on minor or irrelevant points. &
No meaningful critique is provided, or it fails to identify any strengths or weaknesses of the architecture. \\
\hline

\textbf{Relevance\newline25\%} &
Critique is closely aligned with the ASRs and quality attributes, offering a clear and detailed explanation of how well the architecture meets them. &
Critique is mostly aligned with ASRs and quality attributes, discussing their impact on the architecture effectively. &
Critique references ASRs and quality attributes, but the connection is not always clear or well-supported. &
Critique mentions ASRs and quality attributes, but the connection to the architecture is weak or unclear. &
Critique makes limited or superficial reference to ASRs or quality attributes. &
Critique mentions ASRs and quality attributes but fails to connect them to the architecture. &
Critique is entirely disconnected from the ASRs and quality attributes.\\
\hline

\textbf{Balanced Evaluation\newline15\%} &
Provides a well-balanced critique, discussing both strengths and weaknesses in a fair, objective, and constructive manner. &
Provides a fairly balanced critique, discussing both strengths and weaknesses, but may focus slightly more on one side.	&
Critique discusses strengths and weaknesses, but the evaluation may be unbalanced, focusing more on one aspect than the other. &
Critique covers strengths and weaknesses, but may not be sufficiently balanced or may favor one aspect too much. &
Critique lacks balance, focusing more on weaknesses or strengths, without giving adequate attention to the other side. &
Critique is unbalanced, only discussing strengths or weaknesses in detail with little consideration of the other side. &
Critique is entirely one-sided or overly negative without recognizing any positive aspects of the architecture. \\
\hline

\textbf{Presentation\newline 10\%} &
Presentation is well paced and delivered fluently. Information is logically sequenced, with clear objectives making it very easy to follow. &
Presentation is well paced and delivered clearly. Information is logically sequenced, with some clear objectives making it easy to follow. &
Presentation is mostly well paced and~de\-livered clearly. Information is logically sequenced, with signposting guiding audience through presentation. &
Presentation pace~is a little inconsistent or delivery is occasionally unclear. Information is logically sequenced allowing audience to follow presentation fairly well. &
Presentation pace~is inconsistent or delivery is sometimes unclear. Information is not always logically sequenced, distracting audience from presentation flow. &
Presentation pace~is inconsistent or delivery is unclear. Infor- mation is not logically sequenced, and planned progression was not clear to audience. &
Presentation pace~is inconsistent and~delivery is unclear. Infor- mation is poorly sequenced, confusing audience. \hline

\end{xltabular}

\clearpage

\subsection*{Presentation \#3 Detailed Design}

\fontsize{9}{11}\selectfont

\begin{xltabular}{\linewidth}{| P{1.8cm} | X | X | X | X | X | X | X |}
\hline
\multicolumn{1}{|c}{\multirow{2}{*}{\textbf{Criteria}}} &
  \multicolumn{7}{c|}{\textbf{Standard}} \\ \cline{2-8}
\multicolumn{1}{|c}{} &
  \multicolumn{1}{c|}{\textbf{Exceptional ~ (7)}} &
  \multicolumn{1}{c|}{\textbf{Advanced ~ (6)}} &
  \multicolumn{1}{c|}{\textbf{Proficient ~ (5)}} &
  \multicolumn{1}{c|}{\textbf{Functional ~ (4)}} &
  \multicolumn{1}{c|}{\textbf{Developing ~ (3)}} &
  \multicolumn{1}{c|}{\textbf{Little Evidence ~ (2)}} &
  \multicolumn{1}{c|}{\textbf{No Evidence ~ (1)}} \\ \hline
\endhead
%

\textbf{Selection of Design Focus\newline15\%} &
An important and significant part of the system was selected, showing excellent judgement. &
A relevant and fairly significant part of the system was selected, which reflects key design complexity or importance. &
A reasonable part of the system was selected to present in detail.	&
Design focus is acceptable, but may not show the most relevant or complex aspect of the system. &
Design focus is only partially appropriate.	&
Focus is weak or only marginally relevant to understanding the detailed design.	&
Design focus is inappropriate, trivial, or disconnected from the system. \\
\hline

\textbf{Design Clarity and\newline Completeness\newline30\%} &
Detailed design is presented clearly and comprehensively, with excellent coverage of component responsibilities and interactions. &
Detailed design is mostly clear and complete, effectively showing how components interact and function.	&
Design is generally clear, with most responsibilities and flows explained; minor gaps may exist. &
Design is presented in an understandable way, though some areas are underdeveloped or unclear.	&
Design presentation lacks detail or clarity in key parts, limiting understanding.	&
Design is hard to follow or significantly incomplete. &
Design is confusing, vague, or missing critical information. \\
\hline

\textbf{Design\newline ~~Diagrams\newline25\%} &
All diagrams are easy to comprehend, convey important information, and enhance the presentation. &
Most diagrams are easy to comprehend, convey important~in\-formation, and are used well in the presentation. &
Most diagrams are comprehensible, convey useful information, and are used well in the presentation. &
Most diagrams are comprehensible, convey useful information, and are connected to the presentation. &
Most diagrams are comprehensible, convey some useful information, and are mostly connected to the presentation. &
Some diagrams are incomprehensible, do not convey useful information, or are disconnected from the presentation. &
Most diagrams are incomprehensible, do not convey useful information, or are disconnected from the presentation. \\
\hline

\textbf{Presentation\newline 10\%} &
Presentation is well paced and delivered fluently. Information is logically sequenced, with clear objectives making it very easy to follow. &
Presentation is well paced and delivered clearly. Information is logically sequenced, with some clear objectives making it easy to follow. &
Presentation is mostly well paced and~de\-livered clearly. Information is logically sequenced, with signposting guiding audience through presentation. &
Presentation pace~is a little inconsistent or delivery is occasionally unclear. Information is logically sequenced allowing audience to follow presentation fairly well. &
Presentation pace~is inconsistent or delivery is sometimes unclear. Information is not always logically sequenced, distracting audience from presentation flow. &
Presentation pace~is inconsistent or delivery is unclear. Infor- mation is not logically sequenced, and planned progression was not clear to audience. &
Presentation pace~is inconsistent and~delivery is unclear. Infor- mation is poorly sequenced, confusing audience. \hline

\end{xltabular}


\clearpage

\subsection*{Presentations \#4 and \#5 Comparison}

\fontsize{9}{11}\selectfont

\begin{xltabular}{\linewidth}{| P{1.8cm} | X | X | X | X | X | X | X |}
\hline
\multicolumn{1}{|c}{\multirow{2}{*}{\textbf{Criteria}}} &
  \multicolumn{7}{c|}{\textbf{Standard}} \\ \cline{2-8}
\multicolumn{1}{|c}{} &
  \multicolumn{1}{c|}{\textbf{Exceptional ~ (7)}} &
  \multicolumn{1}{c|}{\textbf{Advanced ~ (6)}} &
  \multicolumn{1}{c|}{\textbf{Proficient ~ (5)}} &
  \multicolumn{1}{c|}{\textbf{Functional ~ (4)}} &
  \multicolumn{1}{c|}{\textbf{Developing ~ (3)}} &
  \multicolumn{1}{c|}{\textbf{Little Evidence ~ (2)}} &
  \multicolumn{1}{c|}{\textbf{No Evidence ~ (1)}} \\ \hline
\endhead
%

\textbf{Alternative Selection\newline15\%} &
Clearly identifies a relevant and credible alternative architecture, with strong justification for its suitability for the project. &
Identifies a relevant alternative architecture with a good, but not thorough, justification. &
Identifies a plausible alternative architecture but justification of its suitability is a little weak.	&
Identifies a plausible alternative architecture but with minimal justification or clarity of why it's viable. &
Identifies an alternative, but the choice may be weak or poorly explained. &
Identifies an alternative that is irrelevant or unclear. &
Does not identify any meaningful alternative architecture or design philosophy. \\
\hline

\textbf{Comparison\newline30\%} &
Provides a highly detailed and insightful comparison of the chosen architecture and alternative, covering key dimensions. Clearly explains which architecture is more suitable and why. &
Provides an informative and clear comparison, covering key dimensions, with good reasoning behind the preference for the chosen architecture. &
Provides a good comparison, touching on the main aspects, though the explanation may lack depth or full clarity in some areas. &
Comparison addresses key aspects, but it lacks depth in areas such as complexity or the impact on ASRs.	&
Comparison is basic and lacks clarity. &
Provides only superficial comparisons, missing key aspects of the architectures or failing to explain their impact on the system. &
Comparison is poorly developed or nonexistent, providing minimal or no insights into how the two architectures compare. \\
\hline

\textbf{Trade-off Analysis\newline25\%} &
Provides a thorough analysis of the trade-offs involved in choosing the alternative, detailing both its strengths and weaknesses, and how these trade-offs might impact the overall system.	&
Provides a strong analysis of trade-offs, with a clear explanation of how the alternative would affect the system's quality attributes.	&
Identifies major trade- offs but lacks a detailed explanation of how they would impact the project's quality attributes.	&
Provides a basic analysis of trade-offs, but lacks depth in understanding their potential impact on the project. &
Mentions trade-offs but provides limited insight into their impact on the overall system, or the trade-offs are unclear. &
Provides minimal analysis of trade-offs, with little connection to system goals or project needs. &
No trade-off analysis is provided, or it is wholly inadequate or unsubstantiated. \\
\hline

\textbf{Presentation\newline 10\%} &
Presentation is well paced and delivered fluently. Information is logically sequenced, with clear objectives making it very easy to follow. &
Presentation is well paced and delivered clearly. Information is logically sequenced, with some clear objectives making it easy to follow. &
Presentation is mostly well paced and~de\-livered clearly. Information is logically sequenced, with signposting guiding audience through presentation. &
Presentation pace~is a little inconsistent or delivery is occasionally unclear. Information is logically sequenced allowing audience to follow presentation fairly well. &
Presentation pace~is inconsistent or delivery is sometimes unclear. Information is not always logically sequenced, distracting audience from presentation flow. &
Presentation pace~is inconsistent or delivery is unclear. Infor- mation is not logically sequenced, and planned progression was not clear to audience. &
Presentation pace~is inconsistent and~delivery is unclear. Infor- mation is poorly sequenced, confusing audience. \hline

\end{xltabular}

\clearpage

\subsection*{Presentation \#6 Security}

\fontsize{9}{11}\selectfont

\begin{xltabular}{\linewidth}{| P{1.8cm} | X | X | X | X | X | X | X |}
\hline
\multicolumn{1}{|c}{\multirow{2}{*}{\textbf{Criteria}}} &
  \multicolumn{7}{c|}{\textbf{Standard}} \\ \cline{2-8}
\multicolumn{1}{|c}{} &
  \multicolumn{1}{c|}{\textbf{Exceptional ~ (7)}} &
  \multicolumn{1}{c|}{\textbf{Advanced ~ (6)}} &
  \multicolumn{1}{c|}{\textbf{Proficient ~ (5)}} &
  \multicolumn{1}{c|}{\textbf{Functional ~ (4)}} &
  \multicolumn{1}{c|}{\textbf{Developing ~ (3)}} &
  \multicolumn{1}{c|}{\textbf{Little Evidence ~ (2)}} &
  \multicolumn{1}{c|}{\textbf{No Evidence ~ (1)}} \\ \hline
\endhead
%

\textbf{Security\newline Threats\newline30\%} &
Clearly and comprehensively identifies all security threats specific to the system, with a deep understanding of their potential impact. &
Identifies key security threats and risks, with a good understanding of their potential impact. &
Identifies several security threats, though some may be less relevant or insufficiently detailed. &
Identifies a few key security threats but misses some major ones or provides insufficient detail. &
Identifies only a limited range of security threats, missing major threats that could impact the system. &
Provides an incomplete or unclear iden- tification of security threats, omitting cri- tical issues.	&
Fails to identify or improperly identifies the security threats. \\
\hline

\textbf{Mitigations\newline30\%} &
Thoroughly explains the security mechanisms used to mitigate the identified threats, and how they are integrated into the architecture. Mechanisms are clearly linked to specific threats.	&
Explains the security mechanisms effectively, and links them to the identified threats and risks, with minor gaps in explanation. &
Provides a good explanation of security mechanisms, but some parts lack clarity or details of how they address specific threats. &
Provides an explanation of security mechanisms, but with vague or incomplete descriptions of how they mitigate the risks.	&
Provides a minimal explanation of security mechanisms, leaving out key details or failing to fully connect them to identified threats.	&
The explanation of security mechanisms is unclear or disconnected from the identified threats, with many important aspects missing.	&
Does not explain the security mechanisms or fails to show how they address the identified threats. \\
\hline

\textbf{Remaining Challenges\newline10\%} &
Thoroughly identifies any remaining security challenges or risks in the architecture and suggests thoughtful, feasible improvements. &
Identifies remaining security challenges with a good explanation of potential future improvements and strategies to address them. &
Acknowledges some remaining security challenges but does not offer concrete or comprehensive strategies for improvement. &
Identifies some challenges but does not offer specific or actionable recommendations for future improvements. &
Mentions remaining security issues, but provides no or very weak suggestions for improvement. &
Superficial identification of remaining security challenges or improvement oppor- tunities. &
Fails to identify remaining security chal- lenges or improvement opportunities, or completely overlooks the topic. \\
\hline

\textbf{Presentation\newline 10\%} &
Presentation is well paced and delivered fluently. Information is logically sequenced, with clear objectives making it very easy to follow. &
Presentation is well paced and delivered clearly. Information is logically sequenced, with some clear objectives making it easy to follow. &
Presentation is mostly well paced and~de\-livered clearly. Information is logically sequenced, with signposting guiding audience through presentation. &
Presentation pace~is a little inconsistent or delivery is occasionally unclear. Information is logically sequenced allowing audience to follow presentation fairly well. &
Presentation pace~is inconsistent or delivery is sometimes unclear. Information is not always logically sequenced, distracting audience from presentation flow. &
Presentation pace~is inconsistent or delivery is unclear. Infor- mation is not logically sequenced, and planned progression was not clear to audience. &
Presentation pace~is inconsistent and~delivery is unclear. Infor- mation is poorly sequenced, confusing audience. \hline

\end{xltabular}

\end{landscape}
