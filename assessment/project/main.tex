\documentclass{csse4400}

\usepackage{enumitem}
\usepackage{fancyhdr}

%Change space before start of \paragraph to 1.5ex.
\makeatletter
\renewcommand{\paragraph}{%
  \@startsection{paragraph}{4}%
  {\z@}{1.5ex \@plus 1ex \@minus .2ex}{-1em}%
  {\normalfont\normalsize\bfseries}%
}
\makeatother

% RUBRIC
\usepackage{multirow}
\usepackage{array}
\usepackage{xltabular}
\usepackage{pdflscape}
\usepackage{enumitem}

\newcolumntype{P}[1]{>{\centering\arraybackslash}p{#1}}
% RUBRIC

\title{Capstone Project}
\author{Richard Thomas \& Brae Webb}
\date{Semester 1, 2025}

\begin{document}

\input{copyright-footer}
\maketitle

\section*{Summary}
Throughout the software architecture course,
you have learnt about a subset of quality attributes of concern to software architects.
You have also been exposed to a number of techniques to satisfy these attributes.
Now, as the capstone project, you are required to
\begin{itemize}
    \item propose a non-trivial software project,
    \item identify the primary quality attributes which would enable success of the project,
    \item design an architecture suitable for the aims of the project,
    \item deploy the architecture, utilising any techniques you have learnt in or out of the course, and
    \item evaluate and report on the success of the software project.
\end{itemize}

\noindent
The successful completion of the project will result in three deliverables, namely,
\begin{enumerate}[label=\roman*]
    \item a proposal of a software project, the proposal must clearly indicate and prioritise two or three quality attributes most important to the project's success,
    \item the developed software, as both source code, and a deployed artifact, and
    \item a report which evaluates the success of the developed software relative to the chosen quality attributes.
\end{enumerate}

\noindent
Your software deliverable must include all supporting software (e.g. test suites or utilities) that are developed to support the delivered software.


\section{Introduction}
We have looked at several core quality attributes in this course, and will continue to look at more over the remainder of the semester.
These attributes were selected because they are key concerns of many real-world software projects.
In this project, you will have an opportunity to explore some of the fun of industry.
You will take the role of an entrepreneur, software architecture, developer, and operations team.

As entrepreneur, you have proposed a project idea.
These proposals have been evaluated by your peers and the teaching team.
You have been allocated to a project based on interest you have indicated by voting on proposals and on your previous coursework experience.

As a team you now need to perform the roles of software architect, developer, and operations team.
You should design the basic structure of the initial software architecture,
based on the scope, functionality, quality attributes, and evaluation plan from the proposal.
The details of the architecture are expected to evolve as you start implementing parts of the system.

Part of the assessment will be how the architecture evolves in response to what you learn during development.
You need to write Architectural Decision Records (ADRs) for each decision you make about the design of the architecture \cite{adr-notes}.
These are to be recorded in your GitHub repository so that the marker can see how your architecture evolved and the reasons for the decisions you made.


\section{Software}
You need to implement a software system that delivers a \link{Minimum Viable Product (MVP)}{https://www.agilealliance.org/glossary/mvp/}.
The MVP needs to implement a usable core of the system's functionality,
which demonstrates that the architecture could deliver the full system functionality.
The MVP also needs to allow the software architecture to be tested to determine if it can deliver the project's important quality attributes.

You may renegotiate the scope of the system during the project,
if you determine that certain aspects of the original scope are not feasible within the project time constraints.
The earlier you do this, the less it will impact on your final result.
You will not explicitly lose marks for renegotiating scope, unless the revised scope limits your ability to adequately test important quality attributes.
But, late changes to scope are likely to have a flow-on effect that could reduce the quality of your final deliverables.
This means that you should attempt to implement some of the riskier parts of the project early.


\section{Evaluation}
You need to test the software system that you implement to demonstrate
how well its architecture supports delivering system functionality and its quality attributes.
This evaluation should be based on the proposal's evaluation plan, but should \textbf{\emph{not}} be limited to only what is in that plan.
You will be assessed on how well you test your system in terms of both functionality and quality attributes.
Discovering issues with the system or its architecture during testing will not adversely affect your marks for the evaluation component of the assessment.

A section of your project report needs to summarise the test results and provide access to the full suite of tests.
You should automate as much of the testing as possible.
Any manual tests need to be documented so that they can be duplicated.
The results of all manual tests need to be recorded in a test report.
This may be a section of the project report, or it may be a separate document with a link to it from the project report.
You need to include test code and test infrastructure in your project's repository.


\section{Report}
The report should include the following content.

\begin{description}
    \item[Title] Name of your software project.
    \item[Abstract] Summarise the key points of your document.
    \item[Changes] Describe and justify any changes made to the project from what was outlined in the proposal.
    \item[Architecture Options] What architectural design patterns were considered and their pros and cons.
    \item[Architecture] Describe the MVP's software architecture in detail.
    \item[Trade-Offs] Describe and justify the trade-offs made in designing the architecture.
    \item[Critique] Describe how well the architecture supports delivering the complete system.
    \item[Evaluation] Summarise testing results and justify how well the software achieves its quality attributes.
    \item[Reflection] Lessons learnt and what you would do differently.
\end{description}

No page limit is specified.
You are expected to be able to make rationale choices of what to include in your report.
A key phrase to keep in mind is ``less is more''.

You do not need to have sections for each topic above, though your report needs to contain the content summarised above.
For example, the description of the architecture could include discussion of trade-offs.
Similarly, the critique and evaluation could be combined so that both are discussed in relation to an architecturally significant requirement (ASR) \cite{view-notes}.

When writing your report, you may assume that the reader is familiar with the project proposal.
You will need to describe any changes your team has made to the original proposal.
A rationale should be provided for each change.
Small changes only need a brief summary of the reason for the change.
Significant changes to functionality of the MVP, or changes to important quality attributes,
need a more detailed justification for the change.
You should provide a reference and link to the original proposal.

Compare and contrast different architectural design patterns that could be used to deliver the system.
Explain the pros and cons of each architectural design pattern in the context of the system's functionality and ASRs.
Justify your choice of the architectural design pattern you used in your design.

Describe the full architecture of your MVP in enough detail
to give the reader a complete understanding of the architecture's design.
Use appropriate views, diagrams and commentary to describe the software architecture.
You should describe parts of the detailed design that demonstrate how the architecture supports delivering key quality attributes \cite{view-notes}.
(e.g. If interoperability was a key quality attribute, you would need to describe the parts of the detailed design that support this.
For example, how you use the adapter design pattern to communicate with external services.)

Describe any trade-offs made during the design of the architecture.
Explain what were the competing issues%
\footnote{``\href{http://www.cs.unc.edu/~stotts/COMP723-s15/patterns/forces.html}{Forces}'' in design patterns terminology.}
and explain why you made the decisions that resulted in your submitted design.

When describing the architecture and trade-offs,
you should summarise and/or reference ADRs that relate to important decisions that affected your architecture.

Your critique should discuss how well the architecture is suited to delivering the full system functionality and quality attributes.
Use test results to support your claims, where this can be shown through testing.
For quality attributes that cannot be easily tested (e.g. extensibility, interoperability, ...),
you will need to provide an argument, based on your architectural design, about how the design supports or enables the attribute.
Some quality attributes (e.g. scalability) may require both test results
and argumentation to demonstrate how well the attributed is delivered.

Summarise test plans and test results in the report.
Provide links to any test plans, scripts or code in your repository.
Where feasible, tests should be automated.
Describe how to run the tests.
%Ideally, you should use \link{GitHub Actions}{https://docs.github.com/en/actions}
%to run tests and potentially deploy artefacts.

Your report should end with a reflection that summarises what you have learnt from designing and implementing this project.
It should include descriptions of what you would do differently, after the experience of implementing the project.
Describe potential benefits or improvements that may be delivered by applying the lessons you have learnt during the project.
You will not lose marks for identifying problems with your architecture or software design.


\section{Repository}
Your team will be provisioned with a repository on GitHub.
All source code, documentation and support artefacts are to be committed to the repository.

\begin{itemize}[itemsep=3pt,parsep=3pt]
    \item Model artefacts (e.g. Structurizr DSL or PUML files) should be stored in the \textbf{\texttt{/model}} directory.
    \item ADRs are to be stored in the \textbf{\texttt{/model/adrs}} directory.
    \item The report must be stored in the \textbf{\texttt{/report}} directory.
    \item The link to your demonstration video must be in a file called \textbf{\texttt{demo.md}}, in the root directory of your repository.
\end{itemize}

Do \textbf{\emph{not}} commit large binary files to the repository.
(i.e. Do not commit Word documents or frequently changing PDF files to the repository.
Do not store your demonstration video in your repository.)
It is recommended that you use LaTeX, or possibly markdown, to write your report.
If you use LaTeX, you should use GitHub actions to produce a PDF of the report.

Your final submission will be what is in the main branch of your repository at the due date of 15:00 on 9 June 2025.


\section{Academic Integrity}
As this is a higher-level course, you are expected to be familiar with the importance of academic integrity in general, and the details of UQ's rules.
If you need a reminder, review the \link{Academic Integrity Modules}
{https://web.library.uq.edu.au/library-services/it/learnuq-blackboard-help/academic-integrity-modules}.
Submissions will be checked to ensure that the work submitted is not plagiarised.

All code that you submit must be your own work or must be appropriately cited.
If you find ideas, code fragments, or libraries from external sources (e.g. Stack Overflow), you must \link{cite and reference}{https://guides.library.uq.edu.au/referencing} these sources.
Use the \link{IEEE referencing style}{https://libraryguides.vu.edu.au/ieeereferencing/gettingstarted} for citations and references.
Citations should be included in a comment at the location where the idea is used in your code.
All references for citations must be included in a file called \texttt{refs.md}.
This file should be in the root directory of your repository.

You are encouraged to use a generative AI tool (e.g. Copilot) to help you write the source code for this project.
The expectation is that the software architecture and detailed design are your team's own work.
Create a file in the root directory of your repository called \texttt{ai.md}.
In this file, describe how you used generative AI to develop your project.
(e.g. We prompted ChatGPT with a description of our design
and refined the provided code via further prompting and manually revising the code.)
Provide examples of prompts provided to any generative AI tool.

In \texttt{ai.md} indicate which files contain code produced with the assistance of an AI tool.
Estimate how much of the code was produced by the tool and how much was your own work.\\
(e.g. \texttt{logic.py ~~40\% generated})

You may use libraries to help implement your project.
The library's license must allow you to use it in the context of your project.
All libraries used in your project must be listed in a file called \texttt{libs.md}.
You must include a link to each library's homepage.
This \texttt{libs.md} file must be in your repository's root directory.

Uncited, unreferenced, or unacknowledged material will be treated as not being your own work.
Significant amounts of cited material from other sources will be considered to be of no academic merit.
Having an AI tool produce significant amounts of source code is acceptable,
\textbf{\textit{if}} the design is your own and you have conducted detailed verification of the code.


\section{Demonstration}
Your team needs to demonstrate your project's functionality and how well it achieves its goals.
This should include demonstrating how quality attributes are achieved,
or briefly summarising how the architecture facilitates delivering a quality attribute.

Your project demonstration will be a video.
This is \textbf{\textit{completely separate}} from your presentation assessment.
It is a demonstration of your completed project.
Provide a link to the video in a file called \textbf{\texttt{demo.md}}, stored in the root directory of your repository.
Do \textbf{\textit{not}} store the video in your GitHub repository.
The link may be to the video on a platform like YouTube, or a file sharing site from where the video may be downloaded.
If you upload the video to a platform like YouTube, you may make it private.
If it is a private video, you must share it with \texttt{richard.thomas@uq.edu.au},
\texttt{g.bai@uq.edu.au}, \texttt{thanhthuy.dao@uq.edu.au}, \\\texttt{zaidul.alam@uq.edu.au},
\texttt{vukhanhvy.ho@uq.edu.au}, \texttt{n.garg@uq.edu.au}, and all of your team members.
The video must remain available until \textbf{\textit{at least}} 31 July 2025.
Viewers must be able to easily see what is being demonstrated and read any text or images.
Audio must be clear and comprehensible.

\begin{samepage}
The video  should be approximately as follows.

\begin{description}
    \item[3 min] Introduction to the project.
    \item[3 min] Demonstration of the functional requirements.
    \item[3 min] Demonstration of the non-functional requirements.
    \item[3 min] Overview of the software architecture.
    \item[3 min] Summary of your reflection on lessons learnt from implementing the software.
\end{description}
\end{samepage}

\noindent
The total duration of your video must be \textbf{\emph{less}} than 15 minutes.

\paragraph{Introduction} Briefly introduce the project context and summarise the delivered functional and non-functional requirements.
Mention any differences between what was originally proposed, what was renegotiated, and what was delivered.
Briefly explain why changes in scope were made.
The person who may have approved a change in scope may not be the person marking your demonstration.
If you did not deliver everything in the revised scope of the project, the marker needs to know why that occurred.

\paragraph{Functional Requirements} Demonstrate the key features of the software.
You do not have time to demonstrate every feature of the software.
Plan your time wisely to highlight the completeness and quality of your delivered system.

\paragraph{Non-Functional Requirements} Show how well the software delivers its important quality attributes.
This may take some thought and planning to demonstrate within a short time frame.
Delivery of some non-functional requirements can be shown by test results.
Delivery of other non-functional requirements may be shown through a combination of tests, metrics, and commentary.

For example, you cannot show ten minutes of k6 testing to demonstrate scalability.
You could provide screenshots of different stages of the testing, or an edited video of parts of the testing.
You would provide commentary summarising how the testing was done and explaining how well the system scaled under different loads.

For security, you could show results of simple fuzz testing of APIs.
You could then show examples of parts of your design, explaining how it demonstrates following key security design principles.

For extensibility or interoperability, you could calculate one or more complexity metrics for parts of the design.
You could then use the data from these metrics to support an argument as to why the design was extensible or had a simple interface.
For example, if many interfaces could be shown to have high cohesion and there was low coupling between different modules
or services in the design, you could argue how this shows that the design is likely to be extensible.
You could measure documentation for interfaces or APIs,
and use that to argue that mechanisms used to extend the design, or that the APIs, were comprehensible.

These are examples to help you to start thinking about demonstrating how your design delivers non-functional requirements.
They are not a definitive list of the only or best approaches.
For the demonstration, focus on the most important non-functional requirements for your project.
You should discuss your ideas with course staff if you are unsure of the effectiveness of an approach.

\paragraph{Software Architecture} Provide an overview the system's architecture.
Briefly explain how well it supports delivery of the MVP's, \textit{and} the full system's, functional and non-functional requirements.

\paragraph{Reflection} Summarise the lessons you learnt from implementing the software.
What would you do differently and why?
Explain how you would apply those lessons to design a different architecture or take a different approach to implementing the project.
Or, explain how the lessons learnt demonstrate that you made good choices at each stage of development.

\paragraph{Presentation} There are no constraints on who in your team presents in the video.
One person could present all parts of the video, or you could have different people presenting each part.
Assume that the viewer has read the project proposal but may not yet have read the project report.

%You may use this \link{booking page}{https://calendly.com/richard-thomas-uq/csse6400-demo}
%to schedule a time to demonstrate your project between June 10 and 17.
%The demonstration is to take a \textbf{maximum} of 20 minutes.
%Your system must be deployed and set up so you can start your demonstration \textbf{immediately}.
%You should not take demonstration time to do any deployment or initialisation.
%
%When you book the demonstration time, you will be sent a confirmation email with a Zoom link for the demonstration.
%Only \textbf{one person} from your team should make a demonstration booking.
%Ensure you discuss with your other team members to find suitable times that are available for booking,
%and for which at least all of your key team members are available.


\section{Grading Criteria}

\begin{description}[itemsep=3pt]
    \item[20\%] Extent to which project's scope was delivered.
    \item[15\%] Suitability of architecture to deliver system goals.
    \item[20\%] Quality and thoroughness of testing.
    \item[25\%] Clarity, accuracy and completeness of architecture's description.
    \item[20\%] Insightfulness of architecture's evaluation.
\end{description}


\bibliographystyle{ieeetr}
\bibliography{ours}


\clearpage
\begin{landscape}

\section*{Marking Criteria: Common Part~(20\%)}

All team members will be awarded the same result for the Title Slide, Introduction and Context
(by \textbf{Presenter \#1}), ASRs (by \textbf{Presenter \#2}), and Conclusion (by \textbf{Presenter \#6}).

\fontsize{9}{11}\selectfont

\begin{xltabular}{\linewidth}{| P{1.8cm} | X | X | X | X | X | X | X |}
\hline
\multicolumn{1}{|c}{\multirow{2}{*}{\textbf{Criteria}}} &
  \multicolumn{7}{c|}{\textbf{Standard}} \\ \cline{2-8}
\multicolumn{1}{|c}{} &
  \multicolumn{1}{c|}{\textbf{Exceptional ~ (7)}} &
  \multicolumn{1}{c|}{\textbf{Advanced ~ (6)}} &
  \multicolumn{1}{c|}{\textbf{Proficient ~ (5)}} &
  \multicolumn{1}{c|}{\textbf{Functional ~ (4)}} &
  \multicolumn{1}{c|}{\textbf{Developing ~ (3)}} &
  \multicolumn{1}{c|}{\textbf{Little Evidence ~ (2)}} &
  \multicolumn{1}{c|}{\textbf{No Evidence ~ (1)}} \\ \hline
\endhead
%
\textbf{~Context\newline 5\%} &
Project is introduced clearly and well situated within its context, providing an excellent starting point to understand the system. &
Project is introduced clearly with good~con\-textual information, providing a good starting point to understand the system. &
Project is introduced well with a good over\-view of its context, providing a clear but basic overview of the system. &
Project is introduced fairly well with some contextual informa\-tion, providing a com\-prehensible over\-view of the system. &
Project scope \& general context are fairly clear, providing a general overview of the system. &
Project scope \& context are not clear, providing a poor overview of the system. &
Project scope \& context are confusing, providing an inaccurate overview of the system. \\
\hline

\textbf{~ ~ASRs\newline 10\%} &
ASRs are clearly described, well justified, clearly of high importance, and all will influence architecture decisions. &
ASRs are clearly described, fairly well jus\-tified, seemingly of high importance, and all are likely to influ\-ence architecture decisions. &
Most ASRs are well described but a few justifications are a little weak. Most are important and likely to influence architecture decisions. &
Some ASRs are well described but a few justifications are weak. Most are important and likely to influence architecture decisions. &
Some ASRs are fairly well described but some justifications~are weak. Some are important and likely to influence architecture decisions. &
Most ASRs are poorly described or poorly justified. Few are im\-portant or likely to influence architecture decisions. &
Most ASRs are poorly described and poorly justified. Very few are important or likely to influence architecture decisions. \\
\hline

\textbf{~ ~Conclusion\newline 5\%} &
Conclusion provides a clear, well-structured summary of all key architectural points and offers insightful reflection on lessons learnt. &
Conclusion clearly summarises~most~key architectural points, and includes thoughtful reflection. &
Conclusion summarises main points clearly and includes some useful reflection. &
Conclusion presents a reasonable summary, though some points may be underdeveloped. &
Conclusion attempts to summarise key points but is vague or superficial. &
Conclusion is unclear or disorganised, with poor summarisation. &
Conclusion is confusing or missing. \\
\hline
\end{xltabular}

\clearpage


\section*{Marking Criteria: Individual Part~(80\%)}

\subsection*{Presentation \#1 Title Slide, Introduction and Context, Architecture}

\fontsize{9}{11}\selectfont

\begin{xltabular}{\linewidth}{| P{1.8cm} | X | X | X | X | X | X | X |}
\hline
\multicolumn{1}{|c}{\multirow{2}{*}{\textbf{Criteria}}} &
  \multicolumn{7}{c|}{\textbf{Standard}} \\ \cline{2-8}
\multicolumn{1}{|c}{} &
  \multicolumn{1}{c|}{\textbf{Exceptional ~ (7)}} &
  \multicolumn{1}{c|}{\textbf{Advanced ~ (6)}} &
  \multicolumn{1}{c|}{\textbf{Proficient ~ (5)}} &
  \multicolumn{1}{c|}{\textbf{Functional ~ (4)}} &
  \multicolumn{1}{c|}{\textbf{Developing ~ (3)}} &
  \multicolumn{1}{c|}{\textbf{Little Evidence ~ (2)}} &
  \multicolumn{1}{c|}{\textbf{No Evidence ~ (1)}} \\ \hline
\endhead
%

\textbf{Architecture\newline Completeness\newline25\%} &
Description is clear, complete, concise, and informative, resulting in an excellent and coherent understanding of the overall architecture and its major components. &
Description is clear, almost complete, and informative, resulting in a good and coherent understanding of the system's architecture and structure.	&
Description is mostly clear and informative, resulting in a good understanding of the system's architectural structure. &
Description is mostly clear and informative, though some architectural elements may be missing or underexplained. &
At times the description lacks clarity, leading to a vague or partial overview of the system's architecture. &
Description is unclear or incomplete, omitting important architectural elements or structure, leading to a poor understanding of the architecture. &
Description is confusing, severely incomplete, resulting in an incorrect or misleading understanding of the architecture. \\
\hline

\textbf{Architecture Clarity and\newline Consistency\newline20\%} &
Architectural structure is communicated with excellent clarity, logical flow, consistency and at an appropriate level of abstraction.  Relationships and res- ponsibilities between components~are~well explained and coherent. &
Structure is clearly presented and mostly consistent. Component responsibilities and relationships are explained well. &
Architecture is mostly clear and consistent, though some relationships or responsibilities may be weakly described.	Description is understandable but may lack cohesion, with minor inconsistencies or unclear relationships. &
Description is understandable~but~may lack cohesion, with minor inconsistencies or unclear relationships. &
Architectural explanation is somewhat disorganised or inconsistent, weakening the overall coherence. &
Explanation is unclear or inconsistent, making it difficult to follow architectural relationships. &
Explanation is highly inconsistent or incoherent, obscuring the system's architecture entirely. \\
\hline

\textbf{Design\newline ~~Diagrams\newline25\%} &
All diagrams are easy to comprehend, convey important information, and enhance the presentation. &
Most diagrams are easy to comprehend, convey important~in\-formation, and are used well in the presentation. &
Most diagrams are comprehensible, convey useful information, and are used well in the presentation. &
Most diagrams are comprehensible, convey useful information, and are connected to the presentation. &
Most diagrams are comprehensible, convey some useful information, and are mostly connected to the presentation. &
Some diagrams are incomprehensible, do not convey useful information, or are disconnected from the presentation. &
Most diagrams are incomprehensible, do not convey useful information, or are disconnected from the presentation. \\
\hline

\textbf{Presentation\newline 10\%} &
Presentation is well paced and delivered fluently. Information is logically sequenced, with clear objectives making it very easy to follow. &
Presentation is well paced and delivered clearly. Information is logically sequenced, with some clear objectives making it easy to follow. &
Presentation is mostly well paced and~de\-livered clearly. Information is logically sequenced, with signposting guiding audience through presentation. &
Presentation pace~is a little inconsistent or delivery is occasionally unclear. Information is logically sequenced allowing audience to follow presentation fairly well. &
Presentation pace~is inconsistent or delivery is sometimes unclear. Information is not always logically sequenced, distracting audience from presentation flow. &
Presentation pace~is inconsistent or delivery is unclear. Infor- mation is not logically sequenced, and planned progression was not clear to audience. &
Presentation pace~is inconsistent and~delivery is unclear. Infor- mation is poorly sequenced, confusing audience. \\
\hline

\end{xltabular}

\clearpage

\subsection*{Presentation \#2 Critique}

\fontsize{9}{11}\selectfont

\begin{xltabular}{\linewidth}{| P{1.8cm} | X | X | X | X | X | X | X |}
\hline
\multicolumn{1}{|c}{\multirow{2}{*}{\textbf{Criteria}}} &
  \multicolumn{7}{c|}{\textbf{Standard}} \\ \cline{2-8}
\multicolumn{1}{|c}{} &
  \multicolumn{1}{c|}{\textbf{Exceptional ~ (7)}} &
  \multicolumn{1}{c|}{\textbf{Advanced ~ (6)}} &
  \multicolumn{1}{c|}{\textbf{Proficient ~ (5)}} &
  \multicolumn{1}{c|}{\textbf{Functional ~ (4)}} &
  \multicolumn{1}{c|}{\textbf{Developing ~ (3)}} &
  \multicolumn{1}{c|}{\textbf{Little Evidence ~ (2)}} &
  \multicolumn{1}{c|}{\textbf{No Evidence ~ (1)}} \\ \hline
\endhead
%

\textbf{Depth\newline30\%} &
Provides a thorough, critical analysis of the architecture, addressing key strengths, weaknesses, and how well it meets the ASRs and quality attributes. The critique is insightful, balanced, and well-supported by evidence.&
Provides a comprehensive critique with clear analysis of the architecture's strengths, weaknesses, and how it addresses ASRs and quality attributes. Some evidence supports the critique. &
Critique is generally well-developed, covering major strengths and weaknesses, though it may lack some depth or specific evidence.&
Critique is adequate but lacks depth, with only superficial analysis of strengths, weaknesses, and ASRs. &
Critique is somewhat vague, with limited analysis of the architecture's strengths and weaknesses. &
Critique lacks meaningful analysis or focuses only on minor or irrelevant points. &
No meaningful critique is provided, or it fails to identify any strengths or weaknesses of the architecture. \\
\hline

\textbf{Relevance\newline25\%} &
Critique is closely aligned with the ASRs and quality attributes, offering a clear and detailed explanation of how well the architecture meets them. &
Critique is mostly aligned with ASRs and quality attributes, discussing their impact on the architecture effectively. &
Critique references ASRs and quality attributes, but the connection is not always clear or well-supported. &
Critique mentions ASRs and quality attributes, but the connection to the architecture is weak or unclear. &
Critique makes limited or superficial reference to ASRs or quality attributes. &
Critique mentions ASRs and quality attributes but fails to connect them to the architecture. &
Critique is entirely disconnected from the ASRs and quality attributes.\\
\hline

\textbf{Balanced Evaluation\newline15\%} &
Provides a well-balanced critique, discussing both strengths and weaknesses in a fair, objective, and constructive manner. &
Provides a fairly balanced critique, discussing both strengths and weaknesses, but may focus slightly more on one side.	&
Critique discusses strengths and weaknesses, but the evaluation may be unbalanced, focusing more on one aspect than the other. &
Critique covers strengths and weaknesses, but may not be sufficiently balanced or may favor one aspect too much. &
Critique lacks balance, focusing more on weaknesses or strengths, without giving adequate attention to the other side. &
Critique is unbalanced, only discussing strengths or weaknesses in detail with little consideration of the other side. &
Critique is entirely one-sided or overly negative without recognizing any positive aspects of the architecture. \\
\hline

\textbf{Presentation\newline 10\%} &
Presentation is well paced and delivered fluently. Information is logically sequenced, with clear objectives making it very easy to follow. &
Presentation is well paced and delivered clearly. Information is logically sequenced, with some clear objectives making it easy to follow. &
Presentation is mostly well paced and~de\-livered clearly. Information is logically sequenced, with signposting guiding audience through presentation. &
Presentation pace~is a little inconsistent or delivery is occasionally unclear. Information is logically sequenced allowing audience to follow presentation fairly well. &
Presentation pace~is inconsistent or delivery is sometimes unclear. Information is not always logically sequenced, distracting audience from presentation flow. &
Presentation pace~is inconsistent or delivery is unclear. Infor- mation is not logically sequenced, and planned progression was not clear to audience. &
Presentation pace~is inconsistent and~delivery is unclear. Infor- mation is poorly sequenced, confusing audience. \hline

\end{xltabular}

\clearpage

\subsection*{Presentation \#3 Detailed Design}

\fontsize{9}{11}\selectfont

\begin{xltabular}{\linewidth}{| P{1.8cm} | X | X | X | X | X | X | X |}
\hline
\multicolumn{1}{|c}{\multirow{2}{*}{\textbf{Criteria}}} &
  \multicolumn{7}{c|}{\textbf{Standard}} \\ \cline{2-8}
\multicolumn{1}{|c}{} &
  \multicolumn{1}{c|}{\textbf{Exceptional ~ (7)}} &
  \multicolumn{1}{c|}{\textbf{Advanced ~ (6)}} &
  \multicolumn{1}{c|}{\textbf{Proficient ~ (5)}} &
  \multicolumn{1}{c|}{\textbf{Functional ~ (4)}} &
  \multicolumn{1}{c|}{\textbf{Developing ~ (3)}} &
  \multicolumn{1}{c|}{\textbf{Little Evidence ~ (2)}} &
  \multicolumn{1}{c|}{\textbf{No Evidence ~ (1)}} \\ \hline
\endhead
%

\textbf{Selection of Design Focus\newline15\%} &
An important and significant part of the system was selected, showing excellent judgement. &
A relevant and fairly significant part of the system was selected, which reflects key design complexity or importance. &
A reasonable part of the system was selected to present in detail.	&
Design focus is acceptable, but may not show the most relevant or complex aspect of the system. &
Design focus is only partially appropriate.	&
Focus is weak or only marginally relevant to understanding the detailed design.	&
Design focus is inappropriate, trivial, or disconnected from the system. \\
\hline

\textbf{Design Clarity and\newline Completeness\newline30\%} &
Detailed design is presented clearly and comprehensively, with excellent coverage of component responsibilities and interactions. &
Detailed design is mostly clear and complete, effectively showing how components interact and function.	&
Design is generally clear, with most responsibilities and flows explained; minor gaps may exist. &
Design is presented in an understandable way, though some areas are underdeveloped or unclear.	&
Design presentation lacks detail or clarity in key parts, limiting understanding.	&
Design is hard to follow or significantly incomplete. &
Design is confusing, vague, or missing critical information. \\
\hline

\textbf{Design\newline ~~Diagrams\newline25\%} &
All diagrams are easy to comprehend, convey important information, and enhance the presentation. &
Most diagrams are easy to comprehend, convey important~in\-formation, and are used well in the presentation. &
Most diagrams are comprehensible, convey useful information, and are used well in the presentation. &
Most diagrams are comprehensible, convey useful information, and are connected to the presentation. &
Most diagrams are comprehensible, convey some useful information, and are mostly connected to the presentation. &
Some diagrams are incomprehensible, do not convey useful information, or are disconnected from the presentation. &
Most diagrams are incomprehensible, do not convey useful information, or are disconnected from the presentation. \\
\hline

\textbf{Presentation\newline 10\%} &
Presentation is well paced and delivered fluently. Information is logically sequenced, with clear objectives making it very easy to follow. &
Presentation is well paced and delivered clearly. Information is logically sequenced, with some clear objectives making it easy to follow. &
Presentation is mostly well paced and~de\-livered clearly. Information is logically sequenced, with signposting guiding audience through presentation. &
Presentation pace~is a little inconsistent or delivery is occasionally unclear. Information is logically sequenced allowing audience to follow presentation fairly well. &
Presentation pace~is inconsistent or delivery is sometimes unclear. Information is not always logically sequenced, distracting audience from presentation flow. &
Presentation pace~is inconsistent or delivery is unclear. Infor- mation is not logically sequenced, and planned progression was not clear to audience. &
Presentation pace~is inconsistent and~delivery is unclear. Infor- mation is poorly sequenced, confusing audience. \hline

\end{xltabular}


\clearpage

\subsection*{Presentations \#4 and \#5 Comparison}

\fontsize{9}{11}\selectfont

\begin{xltabular}{\linewidth}{| P{1.8cm} | X | X | X | X | X | X | X |}
\hline
\multicolumn{1}{|c}{\multirow{2}{*}{\textbf{Criteria}}} &
  \multicolumn{7}{c|}{\textbf{Standard}} \\ \cline{2-8}
\multicolumn{1}{|c}{} &
  \multicolumn{1}{c|}{\textbf{Exceptional ~ (7)}} &
  \multicolumn{1}{c|}{\textbf{Advanced ~ (6)}} &
  \multicolumn{1}{c|}{\textbf{Proficient ~ (5)}} &
  \multicolumn{1}{c|}{\textbf{Functional ~ (4)}} &
  \multicolumn{1}{c|}{\textbf{Developing ~ (3)}} &
  \multicolumn{1}{c|}{\textbf{Little Evidence ~ (2)}} &
  \multicolumn{1}{c|}{\textbf{No Evidence ~ (1)}} \\ \hline
\endhead
%

\textbf{Alternative Selection\newline15\%} &
Clearly identifies a relevant and credible alternative architecture, with strong justification for its suitability for the project. &
Identifies a relevant alternative architecture with a good, but not thorough, justification. &
Identifies a plausible alternative architecture but justification of its suitability is a little weak.	&
Identifies a plausible alternative architecture but with minimal justification or clarity of why it's viable. &
Identifies an alternative, but the choice may be weak or poorly explained. &
Identifies an alternative that is irrelevant or unclear. &
Does not identify any meaningful alternative architecture or design philosophy. \\
\hline

\textbf{Comparison\newline30\%} &
Provides a highly detailed and insightful comparison of the chosen architecture and alternative, covering key dimensions. Clearly explains which architecture is more suitable and why. &
Provides an informative and clear comparison, covering key dimensions, with good reasoning behind the preference for the chosen architecture. &
Provides a good comparison, touching on the main aspects, though the explanation may lack depth or full clarity in some areas. &
Comparison addresses key aspects, but it lacks depth in areas such as complexity or the impact on ASRs.	&
Comparison is basic and lacks clarity. &
Provides only superficial comparisons, missing key aspects of the architectures or failing to explain their impact on the system. &
Comparison is poorly developed or nonexistent, providing minimal or no insights into how the two architectures compare. \\
\hline

\textbf{Trade-off Analysis\newline25\%} &
Provides a thorough analysis of the trade-offs involved in choosing the alternative, detailing both its strengths and weaknesses, and how these trade-offs might impact the overall system.	&
Provides a strong analysis of trade-offs, with a clear explanation of how the alternative would affect the system's quality attributes.	&
Identifies major trade- offs but lacks a detailed explanation of how they would impact the project's quality attributes.	&
Provides a basic analysis of trade-offs, but lacks depth in understanding their potential impact on the project. &
Mentions trade-offs but provides limited insight into their impact on the overall system, or the trade-offs are unclear. &
Provides minimal analysis of trade-offs, with little connection to system goals or project needs. &
No trade-off analysis is provided, or it is wholly inadequate or unsubstantiated. \\
\hline

\textbf{Presentation\newline 10\%} &
Presentation is well paced and delivered fluently. Information is logically sequenced, with clear objectives making it very easy to follow. &
Presentation is well paced and delivered clearly. Information is logically sequenced, with some clear objectives making it easy to follow. &
Presentation is mostly well paced and~de\-livered clearly. Information is logically sequenced, with signposting guiding audience through presentation. &
Presentation pace~is a little inconsistent or delivery is occasionally unclear. Information is logically sequenced allowing audience to follow presentation fairly well. &
Presentation pace~is inconsistent or delivery is sometimes unclear. Information is not always logically sequenced, distracting audience from presentation flow. &
Presentation pace~is inconsistent or delivery is unclear. Infor- mation is not logically sequenced, and planned progression was not clear to audience. &
Presentation pace~is inconsistent and~delivery is unclear. Infor- mation is poorly sequenced, confusing audience. \hline

\end{xltabular}

\clearpage

\subsection*{Presentation \#6 Security}

\fontsize{9}{11}\selectfont

\begin{xltabular}{\linewidth}{| P{1.8cm} | X | X | X | X | X | X | X |}
\hline
\multicolumn{1}{|c}{\multirow{2}{*}{\textbf{Criteria}}} &
  \multicolumn{7}{c|}{\textbf{Standard}} \\ \cline{2-8}
\multicolumn{1}{|c}{} &
  \multicolumn{1}{c|}{\textbf{Exceptional ~ (7)}} &
  \multicolumn{1}{c|}{\textbf{Advanced ~ (6)}} &
  \multicolumn{1}{c|}{\textbf{Proficient ~ (5)}} &
  \multicolumn{1}{c|}{\textbf{Functional ~ (4)}} &
  \multicolumn{1}{c|}{\textbf{Developing ~ (3)}} &
  \multicolumn{1}{c|}{\textbf{Little Evidence ~ (2)}} &
  \multicolumn{1}{c|}{\textbf{No Evidence ~ (1)}} \\ \hline
\endhead
%

\textbf{Security\newline Threats\newline30\%} &
Clearly and comprehensively identifies all security threats specific to the system, with a deep understanding of their potential impact. &
Identifies key security threats and risks, with a good understanding of their potential impact. &
Identifies several security threats, though some may be less relevant or insufficiently detailed. &
Identifies a few key security threats but misses some major ones or provides insufficient detail. &
Identifies only a limited range of security threats, missing major threats that could impact the system. &
Provides an incomplete or unclear iden- tification of security threats, omitting cri- tical issues.	&
Fails to identify or improperly identifies the security threats. \\
\hline

\textbf{Mitigations\newline30\%} &
Thoroughly explains the security mechanisms used to mitigate the identified threats, and how they are integrated into the architecture. Mechanisms are clearly linked to specific threats.	&
Explains the security mechanisms effectively, and links them to the identified threats and risks, with minor gaps in explanation. &
Provides a good explanation of security mechanisms, but some parts lack clarity or details of how they address specific threats. &
Provides an explanation of security mechanisms, but with vague or incomplete descriptions of how they mitigate the risks.	&
Provides a minimal explanation of security mechanisms, leaving out key details or failing to fully connect them to identified threats.	&
The explanation of security mechanisms is unclear or disconnected from the identified threats, with many important aspects missing.	&
Does not explain the security mechanisms or fails to show how they address the identified threats. \\
\hline

\textbf{Remaining Challenges\newline10\%} &
Thoroughly identifies any remaining security challenges or risks in the architecture and suggests thoughtful, feasible improvements. &
Identifies remaining security challenges with a good explanation of potential future improvements and strategies to address them. &
Acknowledges some remaining security challenges but does not offer concrete or comprehensive strategies for improvement. &
Identifies some challenges but does not offer specific or actionable recommendations for future improvements. &
Mentions remaining security issues, but provides no or very weak suggestions for improvement. &
Superficial identification of remaining security challenges or improvement oppor- tunities. &
Fails to identify remaining security chal- lenges or improvement opportunities, or completely overlooks the topic. \\
\hline

\textbf{Presentation\newline 10\%} &
Presentation is well paced and delivered fluently. Information is logically sequenced, with clear objectives making it very easy to follow. &
Presentation is well paced and delivered clearly. Information is logically sequenced, with some clear objectives making it easy to follow. &
Presentation is mostly well paced and~de\-livered clearly. Information is logically sequenced, with signposting guiding audience through presentation. &
Presentation pace~is a little inconsistent or delivery is occasionally unclear. Information is logically sequenced allowing audience to follow presentation fairly well. &
Presentation pace~is inconsistent or delivery is sometimes unclear. Information is not always logically sequenced, distracting audience from presentation flow. &
Presentation pace~is inconsistent or delivery is unclear. Infor- mation is not logically sequenced, and planned progression was not clear to audience. &
Presentation pace~is inconsistent and~delivery is unclear. Infor- mation is poorly sequenced, confusing audience. \hline

\end{xltabular}

\end{landscape}


\end{document}