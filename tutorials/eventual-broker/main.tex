\documentclass{csse4400}

\usepackage{tikz}
\usetikzlibrary{positioning}
\usetikzlibrary{arrows}

\usepackage{float}

\usepackage{enumitem}

\usepackage{languages}

\title{Eventual Broker}
\author{Brae Webb}

\date{\week[tutorial]{7}}
\begin{document}

\maketitle

\section{Brief}

Now that we are over halfway through software architecutre,
we should have sufficient knowledge to start making a profit.
Instead of applying for positions as a software architect,
we've decided to use our knowledge to design 
Eventual Broker,
a financial trading system that monitors market data and triggers trades based on certain conditions,
such as price movements, news events, or other market data.

Eventual Broker aims to be a highly reliable and scalable system capable of processing a large volume of market data and trades in real-time.
You want to design a system that is extensible,
as you are able to consume more data sources,
you want your core trading logic to be able to process the new data stream.

\section{Requirements}

\begin{enumerate}
\item The system should be able to receive market data from multiple sources in real-time.
\item The system should be able to process the market data and trigger trades based on certain conditions, as specified by a trading logic unit(s).
\item The system should cope with large volumes of data and trades.
\item The system should be scalable to handle an increasing number and variety of data sources.
\item The system should be observable to appropriately monitor trading and the performance of historical trades.
\end{enumerate}

\section{Outline}

\subsection*{Introduction (5 minutes)}
Introduction to the brief,
what is does our input data stream look like,
what are the requirements,
where to start.

\subsection*{Planning (10 minutes)}
In small groups,
attempt to enumerate potential sources of input data.
Of course,
we will need feeds of informing us of changing in stock values.
What other sources of data might be relevant to give you an edge?
Discuss your existing \textsl{domain knowledge} with your peers.
What types of financial instruments will you trade (e.g. stocks, bonds, options, futures, etc.)?

\subsection*{Design (25 minutes)}

In your group,
design the components that your system requires.
Each component should have a brief description of the functionality that it provides the system.
You will also need to design the flow of information through your system.
This flow of information results in trades being performed.

Your final design can include more sources of information than originally planned.
Make sure that you are considering risk management strategies and observability.
How are the systems decisions being reported?
How are the results of these decisions evaluated?

\subsection*{Presentation (10 minutes)}

In the remaining time,
each group should present their proposed design.
This is an opportunity for discussion amoungst the class to point out limitations of the proposed system designs.

\section{Design Challenges}

\subsection*{Challenge 1: Duplicate Data}
The market data from different sources may overlap,
leading to duplicate data being processed by the system.
For example if you are monitoring the stock price of Apple,
you may receive data from multiple sources,
each of which may have a different delay.
How would you modify your design to prevent this?

\subsection*{Challenge 2: Regulation Compliance}
How can you design the system to comply with financial regulations,
such as those related to insider trading, market manipulation, or other requirements?
Can you use auditing, logging, or other techniques to track and monitor trading activities?

\subsection*{Challenge 3: Trading Manipulation}
Now that our system is not performing market manipulation,
we need to consider preventing others from manipulating our trading system \cite{bot-manipulation}.
For example,
if you are monitoring social networking sites,
how could you prevent one user or a few users manipulating positive tweets of a company to trigger buying their shares?

\bibliography{references}
\bibliographystyle{ieeetr}

\end{document}
